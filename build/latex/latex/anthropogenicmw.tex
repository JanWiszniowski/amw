%% Generated by Sphinx.
\def\sphinxdocclass{report}
\documentclass[letterpaper,10pt,english]{sphinxmanual}
\ifdefined\pdfpxdimen
   \let\sphinxpxdimen\pdfpxdimen\else\newdimen\sphinxpxdimen
\fi \sphinxpxdimen=.75bp\relax
\ifdefined\pdfimageresolution
    \pdfimageresolution= \numexpr \dimexpr1in\relax/\sphinxpxdimen\relax
\fi
%% let collapsible pdf bookmarks panel have high depth per default
\PassOptionsToPackage{bookmarksdepth=5}{hyperref}

\PassOptionsToPackage{booktabs}{sphinx}
\PassOptionsToPackage{colorrows}{sphinx}

\PassOptionsToPackage{warn}{textcomp}
\usepackage[utf8]{inputenc}
\ifdefined\DeclareUnicodeCharacter
% support both utf8 and utf8x syntaxes
  \ifdefined\DeclareUnicodeCharacterAsOptional
    \def\sphinxDUC#1{\DeclareUnicodeCharacter{"#1}}
  \else
    \let\sphinxDUC\DeclareUnicodeCharacter
  \fi
  \sphinxDUC{00A0}{\nobreakspace}
  \sphinxDUC{2500}{\sphinxunichar{2500}}
  \sphinxDUC{2502}{\sphinxunichar{2502}}
  \sphinxDUC{2514}{\sphinxunichar{2514}}
  \sphinxDUC{251C}{\sphinxunichar{251C}}
  \sphinxDUC{2572}{\textbackslash}
\fi
\usepackage{cmap}
\usepackage[T1]{fontenc}
\usepackage{amsmath,amssymb,amstext}
\usepackage{babel}



\usepackage{tgtermes}
\usepackage{tgheros}
\renewcommand{\ttdefault}{txtt}



\usepackage[Bjarne]{fncychap}
\usepackage{sphinx}

\fvset{fontsize=auto}
\usepackage{geometry}


% Include hyperref last.
\usepackage{hyperref}
% Fix anchor placement for figures with captions.
\usepackage{hypcap}% it must be loaded after hyperref.
% Set up styles of URL: it should be placed after hyperref.
\urlstyle{same}


\usepackage{sphinxmessages}
\setcounter{tocdepth}{3}
\setcounter{secnumdepth}{3}


\title{Anthropogenic mw}
\date{Mar 04, 2025}
\release{0.0.1}
\author{Jan Wiszniowski}
\newcommand{\sphinxlogo}{\vbox{}}
\renewcommand{\releasename}{Release}
\makeindex
\begin{document}

\ifdefined\shorthandoff
  \ifnum\catcode`\=\string=\active\shorthandoff{=}\fi
  \ifnum\catcode`\"=\active\shorthandoff{"}\fi
\fi

\pagestyle{empty}
\sphinxmaketitle
\pagestyle{plain}
\sphinxtableofcontents
\pagestyle{normal}
\phantomsection\label{\detokenize{index::doc}}


\sphinxAtStartPar
\sphinxstylestrong{Moment magnitude estimation of small local earthquakes}
\begin{quote}\begin{description}
\sphinxlineitem{Copyright}
\sphinxAtStartPar
2024\sphinxhyphen{}2025 Jan Wiszniowski \sphinxhref{mailto:jwisz@igf.edu.pl}{jwisz@igf.edu.pl}

\sphinxlineitem{Version}
\sphinxAtStartPar


\sphinxlineitem{Release}
\sphinxAtStartPar
0.0.1

\end{description}\end{quote}

\sphinxAtStartPar
The Anthropogenic Mw package is designed for the determination of the moment magnitude (Mw)
of small and local earthquakes, where stations are close to hypocenters.
Such situations often occur in anthropogenic events, and this algorithm was developed to calculate Mw
of mining\sphinxhyphen{}induced and reservoir\sphinxhyphen{}triggered earthquakes. Hence, the package name is \sphinxstyleemphasis{Anthropogenic Mw}.
However, it can also be used for natural events and is recommended for local ones.

\sphinxAtStartPar
The method of Mw computation based on spectral displacement amplitude is elaborated.
Mw is computed using a fitting of displacement spectra of seismic waves recorded at stations
to the simulated spectrum in the far field with the estimation of the noise.
As proposed, it allows for estimating Mw based on a single P or S wave spectra.
However, a combined spectrum of two waves together and spectrum simulation in intermediate and near fields
was applied to Mw estimation as an innovation.
The algorithm automatically estimates the station magnitude of small and close events.

\sphinxAtStartPar
Go to section {\hyperref[\detokenize{description:description}]{\sphinxcrossref{\DUrole{std,std-ref}{Theoretical Background}}}}
to get more information on how the code works.

\sphinxAtStartPar
Mw is written in Python and requires a working Python
environment and same Python packages to run (see {\hyperref[\detokenize{installation:installation}]{\sphinxcrossref{\DUrole{std,std-ref}{Installation}}}}).

\sphinxAtStartPar
The \sphinxstyleemphasis{Anthropogenic Mw} package primarily provides several functions and classes for estimating Mw,
but it also provides a few command line tools:
\begin{itemize}
\item {} 
\sphinxAtStartPar
\sphinxcode{\sphinxupquote{spectral\_Mw}}: Compute moment magnitudes from all events in the QuakeMl catalog
and add them to the catalog.

\item {} 
\sphinxAtStartPar
\sphinxcode{\sphinxupquote{view\_greens\_function}}: Direct modelling of P, S, and PS waves spectra in far, intermediate,
and near fields based on user\sphinxhyphen{}defined earthquake source parameters,
for various hypocentral distances and
plot the results in the frequency domain.

\end{itemize}

\sphinxstepscope


\chapter{Theoretical Background}
\label{\detokenize{description:theoretical-background}}\label{\detokenize{description:description}}\label{\detokenize{description::doc}}

\section{Overview}
\label{\detokenize{description:overview}}
\sphinxAtStartPar
Anthropogenic Mw inverts the P, S, or PS waves displacement spectral amplitude from
station recordings of a single event.

\sphinxAtStartPar
The source parameters are estimated at a station on a recording of the single wave \(c\) (P, S)
or both waves (PS) by minimisation of
\begin{equation*}
\begin{split}\underset{M_0,f_0} {\min} L^{\left(c\right)}= \|\widehat{U}^{(c)}\left( f|M_0,f_0\right),
\widetilde{U}^{\left(c\right)}\left( f \right)\|\end{split}
\end{equation*}
\sphinxAtStartPar
where \(\widehat{U}^{(c))}\left( f|M_0,f_0\right)\) is the simulated spectrum of the wave
and \(\widetilde{U}^{(c)}\left(f\right)\) is the displacement spectrum of the recorded wave.


\section{Signal and noise spectrum}
\label{\detokenize{description:signal-and-noise-spectrum}}
\sphinxAtStartPar
For each station, the code computes P, S, or PS waves displacement spectral amplitude
for each component \(x\) (e.g., Z, N, E), within a predefined time window.
\begin{equation*}
\begin{split}\widetilde{U}^{(c)}_x \left(f\right) =\frac{1}{2\pi f}
\left| \int_{t^{(c)}_1}^{t^{(c)}_2} v_x(t)w(t) e^{-2 j \pi f t} dt \right|\end{split}
\end{equation*}
\sphinxAtStartPar
where the exponent \((c)\) means that we are considering either P,
S, or both P and S waves, \(t^{(c)}_*\) are the start and end times of
the P, S, or both P and S waves time window, \(v_x(t)\) is the velocity time signal
for component \(x\), \(w(t)\) is the tapper, and \(f\) is the frequency.
The spectrum signal is scaled to the time signal according to Parseval’s theorem

\sphinxAtStartPar
The same thing is done for a noise time window. However, it is done for a few separate time windows of noise
(Fig. 1), which allows us to estimate the mean noise spectrum and the variation of the noise spectrum,
which are used to modify the simulated spectrum and to weigh the spectral fitting.

\begin{figure}[htbp]
\centering
\capstart

\noindent\sphinxincludegraphics[width=600\sphinxpxdimen]{{example_with_noise}.png}
\caption{Example three\sphinxhyphen{}component trace plot (displacement), showing noise and S+P wave
windows. Pick pointed as read at traces. The colors of traces are: Z \sphinxhyphen{} blue, NS \sphinxhyphen{} cyan, and EW \sphinxhyphen{} green.
Windows and tappers are presented as dotted lines.
The colors of spectra are: signal \sphinxhyphen{} black, pure source spectra at the station \sphinxhyphen{} green,
source spectra with noise correction at the station \sphinxhyphen{} red, noise \sphinxhyphen{} blue
(dotted lines mark the noise uncertainty)}\label{\detokenize{description:id7}}\end{figure}

\sphinxAtStartPar
The 3\sphinxhyphen{}D signal and noise spectra are the square root of the sum of component squares (e.g., Z, N, E) :
\begin{equation*}
\begin{split}\widetilde{U}^{(c)} \left(f\right) = \sqrt{
\left( \widetilde{U}^{(c)}_Z \left(f\right) \right)^2 +
\left( \widetilde{U}^{(c)}_N \left(f\right) \right)^2 +
\left( \widetilde{U}^{(c)}_E \left(f\right) \right)^2
}.\end{split}
\end{equation*}

\section{Source spectral model}
\label{\detokenize{description:source-spectral-model}}
\sphinxAtStartPar
Assuming no correlation of the signal and noise Pikoulis \sphinxstyleemphasis{et al.} {[}\hyperlink{cite.bibliography:id6}{2020}{]},
the amplitude spectrum of the P or S, or both PS waves displacement spectra
in far, intermediate, and near fields for fitting to the signal (Fig. 1)
are modelled as
\begin{equation*}
\begin{split}{\hat{U}}^{(c)}\left(f|M_0{,f}_0\right)=
\sqrt{\left[S\left(f|M_0{,f}_0\right)G^{(c)}\left(f\right)\right]^2
+\left[\hat{N}\left(f\right)\right]^2},\end{split}
\end{equation*}
\sphinxAtStartPar
where \(S\left(f|M_0{,f}_0\right)\) is the source function,
\(G^{(c)}\left(f\right)\) is the response term,
the product propagation terms in far, intermediate, and near fields,
inelastic (internal) attenuation, and site effect,
\(\hat{N}\left(f\right)\) is the noise estimated at a station including
the response of the recorder.

\sphinxAtStartPar
The Boatwright {[}\hyperlink{cite.bibliography:id2}{1978}{]}, Boatwright {[}\hyperlink{cite.bibliography:id3}{1980}{]} source frequency function is used
\begin{equation*}
\begin{split}S\left(f|M_0,f_0\right)=
{{\frac{1}{2\pi f}M}_0\left[{1+\left(\frac{f}{f_0}\right)}^{n\gamma}\right]}^\frac{-1}{\gamma}\end{split}
\end{equation*}
\sphinxAtStartPar
where constant values \(\gamma\) and \(n\) controls the sharpness of the corners of the spectrum.
Brune {[}\hyperlink{cite.bibliography:id4}{1970}{]}, Brune {[}\hyperlink{cite.bibliography:id5}{1971}{]} model is a particular case of Boatwright’s model
for \(n=2\) and \(\gamma=1\).
However, there are no contraindications to using other source models.

\sphinxAtStartPar
More detailed information about the response term is described in the {\hyperref[\detokenize{api_main:api-main}]{\sphinxcrossref{\DUrole{std,std-ref}{Main module}}}}.


\section{Inversion method}
\label{\detokenize{description:inversion-method}}\label{\detokenize{description:id6}}
\sphinxAtStartPar
The parameters determined from the spectral inversion are moment magnitude \(M_w\) and scalar moment  \(M_0\)
and corner frequency.

\sphinxAtStartPar
The inversion is performed by weighted spectra fitting.
Two distances were investigated:
\begin{enumerate}
\sphinxsetlistlabels{\arabic}{enumi}{enumii}{}{.}%
\item {} \begin{description}
\sphinxlineitem{The p\sphinxhyphen{}norm distance}\begin{equation*}
\begin{split}\left\| \textbf{x},\textbf{y} \right\|=
\left[\sum_{f=f_{low}}^{f_{high}}{\left|x\left(f\right)-y\left(f\right)\right|^p\cdot
w\left(f\right)}\right]^\frac{1}{p}\end{split}
\end{equation*}
\end{description}

\item {} \begin{description}
\sphinxlineitem{The logarithmic distance}\begin{equation*}
\begin{split}\left\| \textbf{x},\textbf{y} \right\|=
\left[\sum_{f=f_{low}}^{f_{high}}{\left| \log\left( x\left(f\right) \right)-
\log\left( y\left(f\right) \right)\right|^p\cdot
w\left(f\right)}\right]^\frac{1}{p}\end{split}
\end{equation*}
\end{description}

\end{enumerate}

\sphinxAtStartPar
The weight coefficients are functions of frequency, signal spectrum,
and noise spectrum estimators:
\begin{equation*}
\begin{split}w^{(c)}\left(f\right)=
w\left(f,\ {\widetilde{U}}^{(c)}\left(f\right),\hat{N}\left(f\right),\sigma_N\left(f\right)\right),\end{split}
\end{equation*}
\sphinxAtStartPar
where \(\sigma_N\left(f\right)\) is a standard deviation of the noise spectrum estimator.

\sphinxAtStartPar
Different inversion algorithms can be used,
but a simple grid search algorithm is  built into the program,
which is sufficiently fast and accurate.

\sphinxstepscope


\chapter{Getting Started}
\label{\detokenize{getting_started:getting-started}}\label{\detokenize{getting_started:id1}}\label{\detokenize{getting_started::doc}}
\begin{sphinxadmonition}{note}{Note:}
\sphinxAtStartPar
For the impatient

\sphinxAtStartPar
To run the example, call it \sphinxcode{\sphinxupquote{spectral\_Mw \sphinxhyphen{}c test.xml \sphinxhyphen{}s test.msd test.json}}
\end{sphinxadmonition}

\sphinxAtStartPar
Both Mw programs \sphinxcode{\sphinxupquote{spectral\_Mw}} and \sphinxcode{\sphinxupquote{view\_green\_fun}} require the configuration file
in the \sphinxhref{https://www.json.org/json-en.html}{JSON} format.
The configuration file name is the only calling parameter that is always required.
It is described in the {\hyperref[\detokenize{configuration:configuration}]{\sphinxcrossref{\DUrole{std,std-ref}{Configuration file}}}} section.
Seismic signal can be read from the MiniSEED file or downloaded from servers.


\section{Spectral Mw calculation}
\label{\detokenize{getting_started:spectral-mw-calculation}}
\sphinxAtStartPar
The spectral Mw is calculated by \sphinxcode{\sphinxupquote{spectral\_Mw}}


\subsection{Use case: external server}
\label{\detokenize{getting_started:use-case-external-server}}
\sphinxAtStartPar
You must modify the {\hyperref[\detokenize{configuration:configuration}]{\sphinxcrossref{\DUrole{std,std-ref}{Configuration file}}}} file and then run

\begin{sphinxVerbatim}[commandchars=\\\{\}]
\PYG{n}{spectral\PYGZus{}Mw} \PYG{o}{\PYGZhy{}}\PYG{n}{q} \PYG{n}{event}\PYG{o}{.}\PYG{n}{xml} \PYG{n}{configuration}\PYG{o}{.}\PYG{n}{json}\PYG{p}{,}
\end{sphinxVerbatim}

\sphinxAtStartPar
where event.xml is an example of the catalog file name
and configuration.json is an example of the configuration file name.


\subsubsection{FDSNWS}
\label{\detokenize{getting_started:fdsnws}}
\sphinxAtStartPar
If you have access to \sphinxhref{https://www.fdsn.org/webservices/}{FDSNWS} server configure at least:

\begin{sphinxVerbatim}[commandchars=\\\{\}]
\PYG{l+s+s2}{\PYGZdq{}}\PYG{l+s+s2}{stream}\PYG{l+s+s2}{\PYGZdq{}}\PYG{p}{:} \PYG{p}{\PYGZob{}}
    \PYG{l+s+s2}{\PYGZdq{}}\PYG{l+s+s2}{source}\PYG{l+s+s2}{\PYGZdq{}}\PYG{p}{:} \PYG{l+s+s2}{\PYGZdq{}}\PYG{l+s+s2}{fdsnws}\PYG{l+s+s2}{\PYGZdq{}}\PYG{p}{,}
    \PYG{l+s+s2}{\PYGZdq{}}\PYG{l+s+s2}{host}\PYG{l+s+s2}{\PYGZdq{}}\PYG{p}{:} \PYG{n}{host\PYGZus{}name}\PYG{p}{,}
    \PYG{l+s+s2}{\PYGZdq{}}\PYG{l+s+s2}{net}\PYG{l+s+s2}{\PYGZdq{}}\PYG{p}{:} \PYG{n}{net\PYGZus{}code}
\PYG{p}{\PYGZcb{}}\PYG{p}{,}
\end{sphinxVerbatim}


\subsubsection{ArcLink}
\label{\detokenize{getting_started:arclink}}
\sphinxAtStartPar
If you have access to \sphinxhref{https://www.seiscomp.de/seiscomp3/doc/applications/arclink.html}{ArcLink} server configure at least:

\begin{sphinxVerbatim}[commandchars=\\\{\}]
\PYG{l+s+s2}{\PYGZdq{}}\PYG{l+s+s2}{stream}\PYG{l+s+s2}{\PYGZdq{}}\PYG{p}{:} \PYG{p}{\PYGZob{}}
    \PYG{l+s+s2}{\PYGZdq{}}\PYG{l+s+s2}{source}\PYG{l+s+s2}{\PYGZdq{}}\PYG{p}{:} \PYG{l+s+s2}{\PYGZdq{}}\PYG{l+s+s2}{arclink}\PYG{l+s+s2}{\PYGZdq{}}\PYG{p}{,}
    \PYG{l+s+s2}{\PYGZdq{}}\PYG{l+s+s2}{host}\PYG{l+s+s2}{\PYGZdq{}}\PYG{p}{:} \PYG{n}{host\PYGZus{}name}\PYG{p}{,}
    \PYG{l+s+s2}{\PYGZdq{}}\PYG{l+s+s2}{port}\PYG{l+s+s2}{\PYGZdq{}}\PYG{p}{:} \PYG{l+s+s2}{\PYGZdq{}}\PYG{l+s+s2}{18001}\PYG{l+s+s2}{\PYGZdq{}}\PYG{p}{,}
    \PYG{l+s+s2}{\PYGZdq{}}\PYG{l+s+s2}{user}\PYG{l+s+s2}{\PYGZdq{}}\PYG{p}{:} \PYG{l+s+s2}{\PYGZdq{}}\PYG{l+s+s2}{anonymous@}\PYG{l+s+s2}{\PYGZdq{}}\PYG{p}{,}
    \PYG{l+s+s2}{\PYGZdq{}}\PYG{l+s+s2}{net}\PYG{l+s+s2}{\PYGZdq{}}\PYG{p}{:} \PYG{n}{net\PYGZus{}code}
\PYG{p}{\PYGZcb{}}\PYG{p}{,}
\end{sphinxVerbatim}


\subsection{Use case: files}
\label{\detokenize{getting_started:use-case-files}}\begin{enumerate}
\sphinxsetlistlabels{\arabic}{enumi}{enumii}{}{.}%
\item {} 
\sphinxAtStartPar
Prepare the the {\hyperref[\detokenize{configuration:configuration}]{\sphinxcrossref{\DUrole{std,std-ref}{Configuration file}}}} file

\item {} \begin{description}
\sphinxlineitem{Run \sphinxcode{\sphinxupquote{spectral\_Mw \sphinxhyphen{}s traces.mseed \sphinxhyphen{}q event1.xml configuration.json}},}
\sphinxAtStartPar
where seismic recordings are in \sphinxhref{http://ds.iris.edu/ds/nodes/dmc/data/formats/miniseed/}{miniSEED} format (e.g., \sphinxcode{\sphinxupquote{traces.mseed}}),
metadata are in \sphinxhref{http://docs.fdsn.org/projects/stationxml/en/latest/}{StationXML} format (metadata file name is in the configration file, e.g., \sphinxcode{\sphinxupquote{station.xml}})
and event information is in \sphinxhref{https://quake.ethz.ch/quakeml/}{QuakeML} format (e.g., \sphinxcode{\sphinxupquote{event.xml}}).

\end{description}

\end{enumerate}


\subsection{Command line arguments}
\label{\detokenize{getting_started:command-line-arguments}}
\sphinxAtStartPar
After successfully installing Anthropogenic Mw (see {\hyperref[\detokenize{installation:installation}]{\sphinxcrossref{\DUrole{std,std-ref}{Mw installation}}}}),
you can get help on the command line arguments used by each code by typing from
your terminal:

\begin{sphinxVerbatim}[commandchars=\\\{\}]
\PYG{n}{spectral\PYGZus{}Mw} \PYG{o}{\PYGZhy{}}\PYG{n}{h}
\end{sphinxVerbatim}


\section{Source spectra visualization}
\label{\detokenize{getting_started:source-spectra-visualization}}
\sphinxAtStartPar
Source spectra are plotted by \sphinxcode{\sphinxupquote{view\_green\_function}}.
Call

\begin{sphinxVerbatim}[commandchars=\\\{\}]
\PYG{n}{view\PYGZus{}green\PYGZus{}fun} \PYG{n}{configuration}\PYG{o}{.}\PYG{n}{json}
\end{sphinxVerbatim}

\sphinxstepscope


\chapter{Configuration file}
\label{\detokenize{configuration:configuration-file}}\label{\detokenize{configuration:configuration}}\label{\detokenize{configuration::doc}}
\sphinxAtStartPar
The configuration is kept in the Python dictionary,
where keys are case\sphinxhyphen{}sensitive strings and values depend on parameters.
They can be strings, float values, integer values, boolean values, sub\sphinxhyphen{}dictionaries, or lists.
Parameters definition can be required or optional. The required definitions must be in the configuration,
whereas optional need not. When definition of optional parameters is missing the default value is taken.
The information whether parameter is required or optional is with the parameter description.
In the case of optional parameter the default value is described.
The configuration file (example name: \sphinxcode{\sphinxupquote{config.json}}) is a file in JavaScript Object Notation (JSON)
Here is the example file:

\begin{sphinxVerbatim}[commandchars=\\\{\}]
\PYG{p}{\PYGZob{}}
  \PYG{l+s+s2}{\PYGZdq{}}\PYG{l+s+s2}{station\PYGZus{}parameters}\PYG{l+s+s2}{\PYGZdq{}} \PYG{p}{:} \PYG{p}{\PYGZob{}}
    \PYG{l+s+s2}{\PYGZdq{}}\PYG{l+s+s2}{any}\PYG{l+s+s2}{\PYGZdq{}} \PYG{p}{:} \PYG{p}{\PYGZob{}}
      \PYG{l+s+s2}{\PYGZdq{}}\PYG{l+s+s2}{phase\PYGZus{}parameters}\PYG{l+s+s2}{\PYGZdq{}}\PYG{p}{:} \PYG{p}{\PYGZob{}}
        \PYG{l+s+s2}{\PYGZdq{}}\PYG{l+s+s2}{P}\PYG{l+s+s2}{\PYGZdq{}} \PYG{p}{:} \PYG{p}{\PYGZob{}}
          \PYG{l+s+s2}{\PYGZdq{}}\PYG{l+s+s2}{Q\PYGZus{}0}\PYG{l+s+s2}{\PYGZdq{}} \PYG{p}{:} \PYG{l+m+mf}{800.0}\PYG{p}{,}
          \PYG{l+s+s2}{\PYGZdq{}}\PYG{l+s+s2}{Q\PYGZus{}theta}\PYG{l+s+s2}{\PYGZdq{}} \PYG{p}{:} \PYG{l+m+mf}{0.0}\PYG{p}{,}
          \PYG{l+s+s2}{\PYGZdq{}}\PYG{l+s+s2}{Q\PYGZus{}corner}\PYG{l+s+s2}{\PYGZdq{}} \PYG{p}{:} \PYG{l+m+mf}{0.0}\PYG{p}{,}
          \PYG{l+s+s2}{\PYGZdq{}}\PYG{l+s+s2}{kappa}\PYG{l+s+s2}{\PYGZdq{}} \PYG{p}{:} \PYG{l+m+mf}{0.0}\PYG{p}{,}
          \PYG{l+s+s2}{\PYGZdq{}}\PYG{l+s+s2}{distance\PYGZus{}exponent}\PYG{l+s+s2}{\PYGZdq{}} \PYG{p}{:} \PYG{l+m+mf}{1.0}\PYG{p}{,}
          \PYG{l+s+s2}{\PYGZdq{}}\PYG{l+s+s2}{low\PYGZus{}frequency}\PYG{l+s+s2}{\PYGZdq{}} \PYG{p}{:} \PYG{l+m+mf}{7.50000000000000010e\PYGZhy{}001}\PYG{p}{,}
          \PYG{l+s+s2}{\PYGZdq{}}\PYG{l+s+s2}{high\PYGZus{}frequency}\PYG{l+s+s2}{\PYGZdq{}} \PYG{p}{:} \PYG{l+m+mf}{3.00000000000000000e+001}\PYG{p}{,}
          \PYG{l+s+s2}{\PYGZdq{}}\PYG{l+s+s2}{noise\PYGZus{}bias}\PYG{l+s+s2}{\PYGZdq{}} \PYG{p}{:} \PYG{l+m+mf}{0.00000000000000000e+000}\PYG{p}{,}
          \PYG{l+s+s2}{\PYGZdq{}}\PYG{l+s+s2}{noise\PYGZus{}std\PYGZus{}bias}\PYG{l+s+s2}{\PYGZdq{}} \PYG{p}{:} \PYG{l+m+mf}{0.00000000000000000e+000}\PYG{p}{,}
          \PYG{l+s+s2}{\PYGZdq{}}\PYG{l+s+s2}{noise\PYGZus{}freq\PYGZus{}bias}\PYG{l+s+s2}{\PYGZdq{}} \PYG{p}{:} \PYG{l+m+mf}{0.00000000000000000e+000}
        \PYG{p}{\PYGZcb{}}\PYG{p}{,}
        \PYG{l+s+s2}{\PYGZdq{}}\PYG{l+s+s2}{S}\PYG{l+s+s2}{\PYGZdq{}} \PYG{p}{:} \PYG{p}{\PYGZob{}}
          \PYG{l+s+s2}{\PYGZdq{}}\PYG{l+s+s2}{Q\PYGZus{}0}\PYG{l+s+s2}{\PYGZdq{}}\PYG{p}{:} \PYG{l+m+mf}{400.0}\PYG{p}{,}
          \PYG{l+s+s2}{\PYGZdq{}}\PYG{l+s+s2}{Q\PYGZus{}theta}\PYG{l+s+s2}{\PYGZdq{}}\PYG{p}{:} \PYG{l+m+mf}{0.0}\PYG{p}{,}
          \PYG{l+s+s2}{\PYGZdq{}}\PYG{l+s+s2}{Q\PYGZus{}corner}\PYG{l+s+s2}{\PYGZdq{}}\PYG{p}{:} \PYG{l+m+mf}{0.0}\PYG{p}{,}
          \PYG{l+s+s2}{\PYGZdq{}}\PYG{l+s+s2}{kappa}\PYG{l+s+s2}{\PYGZdq{}}\PYG{p}{:} \PYG{l+m+mf}{0.0}\PYG{p}{,}
          \PYG{l+s+s2}{\PYGZdq{}}\PYG{l+s+s2}{distance\PYGZus{}exponent}\PYG{l+s+s2}{\PYGZdq{}}\PYG{p}{:} \PYG{l+m+mf}{1.0}\PYG{p}{,}
          \PYG{l+s+s2}{\PYGZdq{}}\PYG{l+s+s2}{low\PYGZus{}frequency}\PYG{l+s+s2}{\PYGZdq{}}\PYG{p}{:} \PYG{l+m+mf}{7.50000000000000010e\PYGZhy{}001}\PYG{p}{,}
          \PYG{l+s+s2}{\PYGZdq{}}\PYG{l+s+s2}{high\PYGZus{}frequency}\PYG{l+s+s2}{\PYGZdq{}}\PYG{p}{:} \PYG{l+m+mf}{3.00000000000000000e+001}\PYG{p}{,}
          \PYG{l+s+s2}{\PYGZdq{}}\PYG{l+s+s2}{noise\PYGZus{}bias}\PYG{l+s+s2}{\PYGZdq{}}\PYG{p}{:} \PYG{l+m+mf}{0.00000000000000000e+000}\PYG{p}{,}
          \PYG{l+s+s2}{\PYGZdq{}}\PYG{l+s+s2}{noise\PYGZus{}std\PYGZus{}bias}\PYG{l+s+s2}{\PYGZdq{}}\PYG{p}{:} \PYG{l+m+mf}{0.00000000000000000e+000}\PYG{p}{,}
          \PYG{l+s+s2}{\PYGZdq{}}\PYG{l+s+s2}{noise\PYGZus{}freq\PYGZus{}bias}\PYG{l+s+s2}{\PYGZdq{}}\PYG{p}{:} \PYG{l+m+mf}{0.00000000000000000e+000}
        \PYG{p}{\PYGZcb{}}
      \PYG{p}{\PYGZcb{}}\PYG{p}{,}
      \PYG{l+s+s2}{\PYGZdq{}}\PYG{l+s+s2}{far\PYGZus{}radial\PYGZus{}average\PYGZus{}radiation}\PYG{l+s+s2}{\PYGZdq{}} \PYG{p}{:} \PYG{l+m+mf}{0.52}\PYG{p}{,}
      \PYG{l+s+s2}{\PYGZdq{}}\PYG{l+s+s2}{far\PYGZus{}transversal\PYGZus{}average\PYGZus{}radiation}\PYG{l+s+s2}{\PYGZdq{}}\PYG{p}{:} \PYG{l+m+mf}{0.63}\PYG{p}{,}
      \PYG{l+s+s2}{\PYGZdq{}}\PYG{l+s+s2}{mw\PYGZus{}correction}\PYG{l+s+s2}{\PYGZdq{}} \PYG{p}{:} \PYG{l+m+mf}{0.00000000000000000e+000}\PYG{p}{,}
      \PYG{l+s+s2}{\PYGZdq{}}\PYG{l+s+s2}{consider\PYGZus{}intermediate\PYGZus{}field}\PYG{l+s+s2}{\PYGZdq{}}\PYG{p}{:} \PYG{n}{false}\PYG{p}{,}
      \PYG{l+s+s2}{\PYGZdq{}}\PYG{l+s+s2}{consider\PYGZus{}near\PYGZus{}field}\PYG{l+s+s2}{\PYGZdq{}}\PYG{p}{:} \PYG{n}{false}
    \PYG{p}{\PYGZcb{}}
  \PYG{p}{\PYGZcb{}}\PYG{p}{,}
  \PYG{l+s+s2}{\PYGZdq{}}\PYG{l+s+s2}{default\PYGZus{}rho}\PYG{l+s+s2}{\PYGZdq{}}\PYG{p}{:} \PYG{l+m+mf}{2700.0}\PYG{p}{,}
  \PYG{l+s+s2}{\PYGZdq{}}\PYG{l+s+s2}{default\PYGZus{}vp}\PYG{l+s+s2}{\PYGZdq{}}\PYG{p}{:} \PYG{l+m+mf}{5200.0}\PYG{p}{,}
  \PYG{l+s+s2}{\PYGZdq{}}\PYG{l+s+s2}{default\PYGZus{}vs}\PYG{l+s+s2}{\PYGZdq{}}\PYG{p}{:} \PYG{l+m+mf}{3000.0}\PYG{p}{,}
  \PYG{l+s+s2}{\PYGZdq{}}\PYG{l+s+s2}{method}\PYG{l+s+s2}{\PYGZdq{}}\PYG{p}{:} \PYG{l+s+s2}{\PYGZdq{}}\PYG{l+s+s2}{separate\PYGZus{}phases}\PYG{l+s+s2}{\PYGZdq{}}\PYG{p}{,}
  \PYG{l+s+s2}{\PYGZdq{}}\PYG{l+s+s2}{metric}\PYG{l+s+s2}{\PYGZdq{}}\PYG{p}{:} \PYG{l+s+s2}{\PYGZdq{}}\PYG{l+s+s2}{lin}\PYG{l+s+s2}{\PYGZdq{}}\PYG{p}{,}
  \PYG{l+s+s2}{\PYGZdq{}}\PYG{l+s+s2}{source\PYGZus{}model}\PYG{l+s+s2}{\PYGZdq{}}\PYG{p}{:} \PYG{l+s+s2}{\PYGZdq{}}\PYG{l+s+s2}{Brune}\PYG{l+s+s2}{\PYGZdq{}}\PYG{p}{,}
  \PYG{l+s+s2}{\PYGZdq{}}\PYG{l+s+s2}{taper}\PYG{l+s+s2}{\PYGZdq{}} \PYG{p}{:} \PYG{p}{\PYGZob{}}
    \PYG{l+s+s2}{\PYGZdq{}}\PYG{l+s+s2}{type}\PYG{l+s+s2}{\PYGZdq{}} \PYG{p}{:} \PYG{l+s+s2}{\PYGZdq{}}\PYG{l+s+s2}{cosine\PYGZus{}taper}\PYG{l+s+s2}{\PYGZdq{}}\PYG{p}{,}
    \PYG{l+s+s2}{\PYGZdq{}}\PYG{l+s+s2}{percentage}\PYG{l+s+s2}{\PYGZdq{}} \PYG{p}{:} \PYG{l+m+mi}{10}
  \PYG{p}{\PYGZcb{}}\PYG{p}{,}
  \PYG{l+s+s2}{\PYGZdq{}}\PYG{l+s+s2}{stream}\PYG{l+s+s2}{\PYGZdq{}}\PYG{p}{:} \PYG{p}{\PYGZob{}}
    \PYG{l+s+s2}{\PYGZdq{}}\PYG{l+s+s2}{source}\PYG{l+s+s2}{\PYGZdq{}}\PYG{p}{:} \PYG{l+s+s2}{\PYGZdq{}}\PYG{l+s+s2}{arclink}\PYG{l+s+s2}{\PYGZdq{}}\PYG{p}{,}
    \PYG{l+s+s2}{\PYGZdq{}}\PYG{l+s+s2}{host}\PYG{l+s+s2}{\PYGZdq{}}\PYG{p}{:} \PYG{l+s+s2}{\PYGZdq{}}\PYG{l+s+s2}{tytan.igf.edu.pl}\PYG{l+s+s2}{\PYGZdq{}}\PYG{p}{,}
    \PYG{l+s+s2}{\PYGZdq{}}\PYG{l+s+s2}{port}\PYG{l+s+s2}{\PYGZdq{}}\PYG{p}{:} \PYG{l+s+s2}{\PYGZdq{}}\PYG{l+s+s2}{18001}\PYG{l+s+s2}{\PYGZdq{}}\PYG{p}{,}
    \PYG{l+s+s2}{\PYGZdq{}}\PYG{l+s+s2}{user}\PYG{l+s+s2}{\PYGZdq{}}\PYG{p}{:} \PYG{l+s+s2}{\PYGZdq{}}\PYG{l+s+s2}{anonymous@igf.edu.pl}\PYG{l+s+s2}{\PYGZdq{}}\PYG{p}{,}
    \PYG{l+s+s2}{\PYGZdq{}}\PYG{l+s+s2}{timeout}\PYG{l+s+s2}{\PYGZdq{}}\PYG{p}{:} \PYG{l+m+mi}{300}\PYG{p}{,}
    \PYG{l+s+s2}{\PYGZdq{}}\PYG{l+s+s2}{net}\PYG{l+s+s2}{\PYGZdq{}}\PYG{p}{:} \PYG{l+s+s2}{\PYGZdq{}}\PYG{l+s+s2}{VN}\PYG{l+s+s2}{\PYGZdq{}}\PYG{p}{,}
    \PYG{l+s+s2}{\PYGZdq{}}\PYG{l+s+s2}{cache}\PYG{l+s+s2}{\PYGZdq{}} \PYG{p}{:} \PYG{l+s+s2}{\PYGZdq{}}\PYG{l+s+s2}{cache\PYGZus{}Mw}\PYG{l+s+s2}{\PYGZdq{}}
  \PYG{p}{\PYGZcb{}}\PYG{p}{,}
  \PYG{l+s+s2}{\PYGZdq{}}\PYG{l+s+s2}{optimization}\PYG{l+s+s2}{\PYGZdq{}}\PYG{p}{:} \PYG{p}{\PYGZob{}}
    \PYG{l+s+s2}{\PYGZdq{}}\PYG{l+s+s2}{method}\PYG{l+s+s2}{\PYGZdq{}}\PYG{p}{:} \PYG{l+s+s2}{\PYGZdq{}}\PYG{l+s+s2}{grid\PYGZus{}search}\PYG{l+s+s2}{\PYGZdq{}}\PYG{p}{,}
    \PYG{l+s+s2}{\PYGZdq{}}\PYG{l+s+s2}{mw}\PYG{l+s+s2}{\PYGZdq{}}\PYG{p}{:} \PYG{p}{[}\PYG{l+m+mf}{0.0}\PYG{p}{,} \PYG{l+m+mf}{5.01}\PYG{p}{,} \PYG{l+m+mf}{0.05}\PYG{p}{]}\PYG{p}{,}
    \PYG{l+s+s2}{\PYGZdq{}}\PYG{l+s+s2}{log\PYGZus{}f0}\PYG{l+s+s2}{\PYGZdq{}}\PYG{p}{:} \PYG{p}{[}\PYG{o}{\PYGZhy{}}\PYG{l+m+mf}{0.6}\PYG{p}{,} \PYG{l+m+mf}{1.51}\PYG{p}{,} \PYG{l+m+mf}{0.05}\PYG{p}{]}
  \PYG{p}{\PYGZcb{}}\PYG{p}{,}
  \PYG{l+s+s2}{\PYGZdq{}}\PYG{l+s+s2}{inventory}\PYG{l+s+s2}{\PYGZdq{}}\PYG{p}{:} \PYG{p}{\PYGZob{}}
    \PYG{l+s+s2}{\PYGZdq{}}\PYG{l+s+s2}{file\PYGZus{}name}\PYG{l+s+s2}{\PYGZdq{}}\PYG{p}{:} \PYG{l+s+s2}{\PYGZdq{}}\PYG{l+s+s2}{VN\PYGZus{}Stations.xml}\PYG{l+s+s2}{\PYGZdq{}}\PYG{p}{,}
    \PYG{l+s+s2}{\PYGZdq{}}\PYG{l+s+s2}{file\PYGZus{}format}\PYG{l+s+s2}{\PYGZdq{}}\PYG{p}{:} \PYG{l+s+s2}{\PYGZdq{}}\PYG{l+s+s2}{STATIONXML}\PYG{l+s+s2}{\PYGZdq{}}
  \PYG{p}{\PYGZcb{}}\PYG{p}{,}
  \PYG{l+s+s2}{\PYGZdq{}}\PYG{l+s+s2}{remove\PYGZus{}response}\PYG{l+s+s2}{\PYGZdq{}}\PYG{p}{:} \PYG{p}{\PYGZob{}}
    \PYG{l+s+s2}{\PYGZdq{}}\PYG{l+s+s2}{prefilter}\PYG{l+s+s2}{\PYGZdq{}}\PYG{p}{:} \PYG{p}{[}\PYG{l+m+mf}{0.01}\PYG{p}{,} \PYG{l+m+mf}{0.2}\PYG{p}{,} \PYG{l+m+mi}{45}\PYG{p}{,} \PYG{l+m+mi}{50}\PYG{p}{]}
  \PYG{p}{\PYGZcb{}}\PYG{p}{,}
  \PYG{l+s+s2}{\PYGZdq{}}\PYG{l+s+s2}{plot}\PYG{l+s+s2}{\PYGZdq{}}\PYG{p}{:} \PYG{p}{\PYGZob{}}
    \PYG{l+s+s2}{\PYGZdq{}}\PYG{l+s+s2}{draw\PYGZus{}the\PYGZus{}noise}\PYG{l+s+s2}{\PYGZdq{}}\PYG{p}{:} \PYG{n}{false}\PYG{p}{,}
    \PYG{l+s+s2}{\PYGZdq{}}\PYG{l+s+s2}{draw\PYGZus{}source\PYGZus{}spectrum\PYGZus{}without\PYGZus{}the\PYGZus{}noise}\PYG{l+s+s2}{\PYGZdq{}}\PYG{p}{:} \PYG{n}{true}\PYG{p}{,}
    \PYG{l+s+s2}{\PYGZdq{}}\PYG{l+s+s2}{own\PYGZus{}frequencies}\PYG{l+s+s2}{\PYGZdq{}}\PYG{p}{:} \PYG{p}{[}\PYG{l+m+mf}{0.1}\PYG{p}{,} \PYG{l+m+mi}{30}\PYG{p}{,} \PYG{l+m+mf}{0.1}\PYG{p}{]}\PYG{p}{,}
    \PYG{l+s+s2}{\PYGZdq{}}\PYG{l+s+s2}{view}\PYG{l+s+s2}{\PYGZdq{}}\PYG{p}{:} \PYG{l+s+s2}{\PYGZdq{}}\PYG{l+s+s2}{DISP}\PYG{l+s+s2}{\PYGZdq{}}\PYG{p}{,}
    \PYG{l+s+s2}{\PYGZdq{}}\PYG{l+s+s2}{how\PYGZus{}to\PYGZus{}show}\PYG{l+s+s2}{\PYGZdq{}}\PYG{p}{:} \PYG{l+s+s2}{\PYGZdq{}}\PYG{l+s+s2}{many figures inline}\PYG{l+s+s2}{\PYGZdq{}}
  \PYG{p}{\PYGZcb{}}
\PYG{p}{\PYGZcb{}}
\end{sphinxVerbatim}

\sphinxAtStartPar
Below is the description of parameters. Not all described bellow parameters are in the example,
not all parameters are required and some of them are optional.
If optional parameters are not defined, default values are assumed.
\begin{quote}\begin{description}
\sphinxlineitem{station\_parameters}
\sphinxAtStartPar
(dict) describe all parameters required to estimate the Mw
at the station. The value is the list of stations, where keys are station names
and vales are sub\sphinxhyphen{}dictionaries of parameters for the current station.
The magic station’s name is \sphinxcode{\sphinxupquote{any}}. When the station configuration is missing or
the specific station parameter is missing, it will be taken from the \sphinxcode{\sphinxupquote{any}} station.
(required)

\sphinxlineitem{velocity\_model}
\sphinxAtStartPar
(str) (optional, default value is “constant”). Now, only “constant” is available.
Other velocity models have not been implemented yet.

\sphinxlineitem{default\_rho}
\sphinxAtStartPar
(float) Default value of density {[}kg/m\textasciicircum{}3{]} (required),

\sphinxlineitem{default\_vp}
\sphinxAtStartPar
(float)  Default value of the P wave velocity {[}m/s{]} (required),

\sphinxlineitem{default\_vs}
\sphinxAtStartPar
(float) Default value of the S wave velocity {[}m/s{]} (required),

\sphinxlineitem{method}
\sphinxAtStartPar
(str) (required) the moment magnitude spectral estimation method.
Two options are available:
\begin{itemize}
\item {} 
\sphinxAtStartPar
“separate\_phases” \sphinxhyphen{} estimation of Mw base on P wave and/or S wave, and the common magnitude is
calculated based on \(Mw^{(P)}\) and \(Mw^{(S)}\)

\item {} 
\sphinxAtStartPar
“multiphase” \sphinxhyphen{} estimate Mw base signal covering both P and S wave.

\end{itemize}

\sphinxlineitem{metric}
\sphinxAtStartPar
(str) The metric distances used for spectra fitting.
Two metrics are available: p\_norm (lin) and log (See {\hyperref[\detokenize{description:inversion-method}]{\sphinxcrossref{\DUrole{std,std-ref}{Inversion method}}}}),
(optional, default value is “p\_norm”),

\sphinxlineitem{p\_value}
\sphinxAtStartPar
(float) The power of the values distances in the metric. (See {\hyperref[\detokenize{description:inversion-method}]{\sphinxcrossref{\DUrole{std,std-ref}{Inversion method}}}})
(optional, default value is 2.0)

\sphinxlineitem{source\_model}
\sphinxAtStartPar
(str) (optional, default value is “Brune”)

\sphinxlineitem{Boatwright gamma}
\sphinxAtStartPar
(float) (optional, default value is 1.0)

\sphinxlineitem{Boatwright n}
\sphinxAtStartPar
(float) (optional, default value is 2.0)

\sphinxlineitem{taper}
\sphinxAtStartPar
(dict) the signal {\hyperref[\detokenize{configuration:taper-parameters}]{\sphinxcrossref{\DUrole{std,std-ref}{Taper parameters}}}} for Fourier transform (required)

\sphinxlineitem{stream}
\sphinxAtStartPar
(dict) {\hyperref[\detokenize{configuration:stream-parameters}]{\sphinxcrossref{\DUrole{std,std-ref}{Stream parameters}}}} describing how to get streams for magnitude estimation (required)

\sphinxlineitem{optimization}
\sphinxAtStartPar
(dict) {\hyperref[\detokenize{configuration:optimization-parameters}]{\sphinxcrossref{\DUrole{std,std-ref}{Optimization parameters}}}} defining the minimization method of signal and source spectra
difference (required)

\sphinxlineitem{inventory}
\sphinxAtStartPar
(dict) (required) The dictionary of parameters defining how to get the inventory of all stations
(see {\hyperref[\detokenize{configuration:inventory-parameters}]{\sphinxcrossref{\DUrole{std,std-ref}{Inventory parameters}}}})

\sphinxlineitem{remove\_response}
\sphinxAtStartPar
(dict) Parameters required for stream preprocessing (required).
They are described in the {\hyperref[\detokenize{configuration:remove-response-parameters}]{\sphinxcrossref{\DUrole{std,std-ref}{Remove response parameters}}}} section.

\sphinxlineitem{plot}
\sphinxAtStartPar
(dict) define the results {\hyperref[\detokenize{configuration:plot-parameters}]{\sphinxcrossref{\DUrole{std,std-ref}{Plot parameters}}}}
(optional, if the position is not defined no plot is performed)

\sphinxlineitem{output\_file}
\sphinxAtStartPar
(str) The name of the output file in which the catalog with estimated magnitudes will be saved
(optional, default value is “output.xml”)

\sphinxlineitem{module}
\sphinxAtStartPar
(str) (optional, default value is “MinimizeInGrid”)

\sphinxlineitem{method}
\sphinxAtStartPar
(str) (optional, default value is “minimize”)

\sphinxlineitem{catalog\_file}
\sphinxAtStartPar
(str) The catalog file name (required unless the name is the call parameter).

\end{description}\end{quote}


\section{Not predefined parameters}
\label{\detokenize{configuration:not-predefined-parameters}}
\sphinxAtStartPar
Not predefined parameters are created during computation and should not be predefined.
\begin{quote}\begin{description}
\sphinxlineitem{Plotter}
\sphinxAtStartPar
(PlotMw) It keeps the plotter class object. Do not define it.

\end{description}\end{quote}


\section{Station parameters}
\label{\detokenize{configuration:station-parameters}}
\sphinxAtStartPar
Station parameters consts of
\begin{description}
\sphinxlineitem{\sphinxstylestrong{phase\_parameters:}}
\sphinxAtStartPar
(dict)
It describes all parameters required to estimate the Mw
at the phase. There could phases \sphinxcode{\sphinxupquote{P}}, \sphinxcode{\sphinxupquote{S}}, and \sphinxcode{\sphinxupquote{any}}.
The parameter is taken from the \sphinxcode{\sphinxupquote{any}} phase if the parameter is missing in \sphinxcode{\sphinxupquote{P}} or \sphinxcode{\sphinxupquote{S}} phases sub\sphinxhyphen{}dictionaries.
(see {\hyperref[\detokenize{configuration:phase-parameters}]{\sphinxcrossref{\DUrole{std,std-ref}{Phase parameters}}}})

\sphinxlineitem{\sphinxstylestrong{far\_radial\_average\_radiation:}}
\sphinxAtStartPar
(float)
The average radiation in the far field radial direction.
It is the same as P wave average radiation in the far field (optional, default value is 0.52)

\sphinxlineitem{\sphinxstylestrong{far\_transversal\_average\_radiation:}}
\sphinxAtStartPar
(float)
The average radiation in the far field transversal directions.
It is the same as S wave average radiation in the far field (optional, default value is 0.63)

\sphinxlineitem{\sphinxstylestrong{mw\_correction:}}
\sphinxAtStartPar
(float)
Empirical Mw correction at the particular station
(optional, default value is 1.0).
It consists of a correction value added to the estimated value before the saving.
It was shown in {[}\hyperlink{cite.bibliography:id7}{Wiejacz and Wiszniowski, 2006}{]} the correction is required in some cases.

\sphinxlineitem{\sphinxstylestrong{consider\_intermediate\_field:}}
\sphinxAtStartPar
(bool)
If it is true, the source spectrum calculation considers the intermediate field
(optional, default value is false)

\sphinxlineitem{\sphinxstylestrong{intermediate\_p\_radial\_average\_radiation:}}
\sphinxAtStartPar
(float)
The average P wave radiation in the intermediate field radial
direction (optional, default value is 4.0 * far\_radial\_average\_radiation)

\sphinxlineitem{\sphinxstylestrong{intermediate\_p\_transversal\_average\_radiation:}}
\sphinxAtStartPar
(float)
The average P wave radiation in the far field transversal
directions (optional, default value is \sphinxhyphen{}2.0 * far\_transversal\_average\_radiation)

\sphinxlineitem{\sphinxstylestrong{intermediate\_s\_radial\_average\_radiation:}}
\sphinxAtStartPar
(float)
The average S wave radiation in the intermediate field radial
direction (optional, default value is \sphinxhyphen{}3.0 * far\_radial\_average\_radiation)

\sphinxlineitem{\sphinxstylestrong{intermediate\_s\_transversal\_average\_radiation:}}
\sphinxAtStartPar
(float)
The average S wave radiation in the far field transversal
directions (optional, default value is 3.0 * far\_transversal\_average\_radiation)

\sphinxlineitem{\sphinxstylestrong{consider\_near\_field:}}
\sphinxAtStartPar
(bool)
If it is true,
the source spectrum calculation for common P and S waves considers the near field
(optional, default value is false)

\sphinxlineitem{\sphinxstylestrong{near\_radial\_average\_radiation:}}
\sphinxAtStartPar
(float)
The average radiation in the far field transversal
directions(optional, default value is 9.0 * far\_radial\_average\_radiation)

\sphinxlineitem{\sphinxstylestrong{near\_transversal\_average\_radiation:}}
\sphinxAtStartPar
(float)
The average S wave radiation in the far field transversal
directions (optional, default value is \sphinxhyphen{}6.0 * far\_transversal\_average\_radiation)

\sphinxlineitem{\sphinxstylestrong{weight:}}
\sphinxAtStartPar
(float)
The station weight for computation of the event magnitude part from the current station magnitude
(optional, default value is 1.0)

\end{description}


\subsection{Phase parameters}
\label{\detokenize{configuration:phase-parameters}}
\sphinxAtStartPar
Phase parameters (\sphinxcode{\sphinxupquote{phase\_parameters}}) consts of

\sphinxAtStartPar
The parameters below define the internal (inelastic) dumping by the formula
\begin{equation*}
\begin{split}A\left(f\right)=exp\left(\frac{-\pi Tf}{Q\left(f\right)}\right),\end{split}
\end{equation*}
\sphinxAtStartPar
where \(Q\left(f\right)\) is is given by
\begin{equation*}
\begin{split}Q\left(f\right)=Q_0\left(\frac{f_q+f}{f_q}\right)^\vartheta,\end{split}
\end{equation*}
\sphinxAtStartPar
or not considering cornel frequency
\begin{equation*}
\begin{split}Q\left(f\right)=Q_0\left( f \right)^\vartheta,\end{split}
\end{equation*}
\sphinxAtStartPar
where:
\begin{quote}\begin{description}
\sphinxlineitem{Q\_0}
\sphinxAtStartPar
(float) define the \(Q_0\) value (required)

\sphinxlineitem{Q\_theta}
\sphinxAtStartPar
(float) define the \(Q_{\theta}\) value (optional, default value is 0.0)

\sphinxlineitem{Q\_corner}
\sphinxAtStartPar
(float) define the \(f_q\) value (optional, default value is 0.0,
which means the second formula, not considering cornel frequency, is used)

\end{description}\end{quote}

\sphinxAtStartPar
The parameters below define the site near surface amplification and dumping according the formula
\begin{equation*}
\begin{split}R\left(f\right)=R_c \exp\left(-\pi \kappa f\right)\end{split}
\end{equation*}
\sphinxAtStartPar
where:
\begin{quote}\begin{description}
\sphinxlineitem{kappa}
\sphinxAtStartPar
(float) define the dumping \(\kappa\) value (optional, default value is 0.0)

\sphinxlineitem{surface\_correction}
\sphinxAtStartPar
(float) define the free surface amplification \(R_c\) value
(optional, default value is 1.0)

\end{description}\end{quote}

\sphinxAtStartPar
These parameters define the site amplification and dumping by the formula for the current phase
when the signal is calculated in the far field.
In the more fields. The signal correction is accordingly:
\begin{itemize}
\item {} 
\sphinxAtStartPar
P phase \sphinxhyphen{} the radial component of the signal

\item {} 
\sphinxAtStartPar
S phase \sphinxhyphen{} the transversal component of the signal

\end{itemize}
\begin{quote}\begin{description}
\sphinxlineitem{distance\_exponent}
\sphinxAtStartPar
(float) 1.0, \sphinxstylestrong{Not used in current solution}

\end{description}\end{quote}

\sphinxAtStartPar
The frequency limits are defined for phases, but in the case of the PS\sphinxhyphen{}wave method the
P wave values are considered.
\begin{quote}\begin{description}
\sphinxlineitem{low\_frequency}
\sphinxAtStartPar
(float) lowest frequency of spectra fitting (optional, default value is 0.5)

\sphinxlineitem{high\_frequency}
\sphinxAtStartPar
(float) highest frequency of spectra fitting (optional, default value is 20.0)

\end{description}\end{quote}

\sphinxAtStartPar
The parameters below modify (bias) the noise correction. We notice that slight noise bias
protects occasional estimation of wrong high magnitude and low cornel frequency in the
case of small signals and low\sphinxhyphen{}frequency noise
The noise is biased by the formula:
\begin{equation*}
\begin{split}\widehat{N} \left( f \right) =
\overline{N} \left( f \right)\left( 1+b_Nf^{b_f} \right) + b_{\sigma}\sigma_N\end{split}
\end{equation*}
\sphinxAtStartPar
where:
\begin{quote}\begin{description}
\sphinxlineitem{noise\_bias}
\sphinxAtStartPar
(float) define the \(b_N\) value (optional, default value is 0.0)

\sphinxlineitem{noise\_std\_bias}
\sphinxAtStartPar
(float) define the \(b_{\sigma}\) value (optional, default value is 0.0)

\sphinxlineitem{noise\_freq\_bias}
\sphinxAtStartPar
(float) (optional, default value is 0.0) define the \(b_ff\) value

\sphinxlineitem{weights}
\sphinxAtStartPar
(dict) the definition of fitting weights assessment (optional, default value is None)

\sphinxlineitem{window}
\sphinxAtStartPar
(dict) The signal window period for spectra computation
(optional, if missing, the window is set based on S \sphinxhyphen{} P time).
It contains two coefficients \(b_1\) and \(b_2\) defined by keys “b1” and “b2”.
The window time is calculated according to \(\tau = r^{b_1} / 10^{b_2}\)

\sphinxlineitem{length}
\sphinxAtStartPar
The minimal S\sphinxhyphen{}wave window (optional, default  = 2 s)

\sphinxlineitem{P\sphinxhyphen{}S}
\sphinxAtStartPar
The part of P\sphinxhyphen{}S time taken as P wave window (optional, default value is 0.9)

\end{description}\end{quote}


\subsubsection{Phase weights parameters}
\label{\detokenize{configuration:phase-weights-parameters}}\begin{quote}\begin{description}
\sphinxlineitem{use\_threshold}
\sphinxAtStartPar
(float) the binary weights threshold \(T_h\)
\begin{equation*}
\begin{split}w = \left\{ {\begin{matrix} 1:\Delta \le T_h \\
0:\Delta < T_h \end{matrix}} \right\}\end{split}
\end{equation*}
\sphinxAtStartPar
If the parameter use\_threshold is missing or None, weights take real values
\begin{equation*}
\begin{split}w = \left\{ \begin{matrix}
0 & : & \Delta \le 0 \\
\Delta & : & 0 < \Delta < 1\\
1 & : & \Delta \ge 1
\end{matrix} \right\}\end{split}
\end{equation*}
\sphinxAtStartPar
(optional, default is None)

\sphinxlineitem{use\_logarithm}
\sphinxAtStartPar
(bool) the \(\Delta\) calculation method.
If the “use\_logarithm” parameter exists and is true
\begin{equation*}
\begin{split}\Delta = \log\left(U \right) - \log\left( N \right)\end{split}
\end{equation*}
\sphinxAtStartPar
else
\begin{equation*}
\begin{split}\Delta = \frac{U-N}{U}\end{split}
\end{equation*}
\sphinxAtStartPar
where \(U\) is the seismic signal and \(N\) is the noise term
computed based on the mean value and standard deviation of noise
(optional, default value is false)

\sphinxlineitem{use\_std}
\sphinxAtStartPar
(float) the coefficient of the part of the standard deviation of noise
in the nose term computation. When the use\_std is set to \(s\) value
\begin{equation*}
\begin{split}N = \bar{N} + s\delta{N}\end{split}
\end{equation*}
\sphinxAtStartPar
else
\begin{equation*}
\begin{split}N = \bar{N}\end{split}
\end{equation*}
\sphinxAtStartPar
where \(\widehat{N}\) is the mean value of noise,
\(\delta{N}\) is the standard deviation of noise,
and \(s\) is the “use\_std” parameter
(optional, default is None)

\sphinxlineitem{use\_frequency}
\sphinxAtStartPar
(float) The option,
whether use frequencies in as fitting weighting (optional, default is None).
The parameter defines the value \(f_w\) in the formula
\begin{equation*}
\begin{split}w = w\left| f - f_m \right|^{f_w}\end{split}
\end{equation*}
\sphinxlineitem{use\_main\_frequency}
\sphinxAtStartPar
The option, whether use main frequency in the middle of testing band
for fitting weighting (float) (optional, default is 0).
The parameter defines the value \(f_m\)

\end{description}\end{quote}

\sphinxAtStartPar
\sphinxstylestrong{The phase P parameters are treated as both phase parameters in the case
of multiphase spectral Mw estimation.}


\section{Taper parameters}
\label{\detokenize{configuration:taper-parameters}}\begin{quote}\begin{description}
\sphinxlineitem{type}
\sphinxAtStartPar
(str) The taper type (required, only available type is now “cosine\_taper”)

\sphinxlineitem{percentage}
\sphinxAtStartPar
(float) Percentage of cosine taper (optional, default value is 10)

\sphinxlineitem{half\_cosine}
\sphinxAtStartPar
(bool) If it is true, the taper is a half cosine function,
otherwise, it is a quarter cosine function (optional, default value is true)

\end{description}\end{quote}


\section{Stream parameters}
\label{\detokenize{configuration:stream-parameters}}\begin{quote}\begin{description}
\sphinxlineitem{source}
\sphinxAtStartPar
(str) The web server source type (required, available options “arclink”, “fdsnws”)

\sphinxlineitem{host}
\sphinxAtStartPar
(str) Host name (required)

\sphinxlineitem{port}
\sphinxAtStartPar
(int) Server port number, (optional)

\sphinxlineitem{user}
\sphinxAtStartPar
(int) User name, (required for arclink)

\sphinxlineitem{timeout}
\sphinxAtStartPar
The waiting time for the server response (optional)

\sphinxlineitem{net}
\sphinxAtStartPar
(str) The network code (required if \sphinxtitleref{stations} parameter is missing)

\sphinxlineitem{loc}
\sphinxAtStartPar
(str) The location filter (optional)

\sphinxlineitem{chan}
\sphinxAtStartPar
(str) Channels filter (optional)

\sphinxlineitem{stations}
\sphinxAtStartPar
(list(str)) list of station names. When stations names are in the form “NN.SSSS”
where “NN” is the network code and “SSSS” is the station code.
The “net” parameter can be omitted.
If stations names are in the form “SSSS”, the “net” parameter must be defined.
It is possible to define in the list individual channels in the form “NN.SSSS.LL.CCC”
where “LL” is a location code (can be empty) and “CCC” is the channel code.

\sphinxlineitem{cache}
\sphinxAtStartPar
(str) the cache directory (optional, if missing data are not cached)

\end{description}\end{quote}


\section{Optimization parameters}
\label{\detokenize{configuration:optimization-parameters}}\begin{quote}\begin{description}
\sphinxlineitem{module}
\sphinxAtStartPar
(str) The module name containing the optimization function (optional, default is “MinimizeInGrid”),

\sphinxlineitem{method}
\sphinxAtStartPar
(str) The optimization function name (optional, default is “grid\_search”),

\sphinxlineitem{mw}
\sphinxAtStartPar
(list) This parameter is required for the grid search optimization.
It is the list of three values describing the grid whose magnitude was checked.
They are: initial value, upper limit, and the step e.g.  {[}0.0, 5.01, 0.05{]},

\sphinxlineitem{log\_f0}
\sphinxAtStartPar
(list) This is the parameter required for the grid search optimization.
It is the list of three values describing the grid in which logarithms were checked.
They are: the initial value, upper limit, and the step, e.g. {[}\sphinxhyphen{}0.6, 1.51, 0.05{]}

\end{description}\end{quote}


\section{Inventory parameters}
\label{\detokenize{configuration:inventory-parameters}}
\sphinxAtStartPar
The \sphinxtitleref{Inventory parameters} describe how to read station inventories.
\begin{quote}\begin{description}
\sphinxlineitem{file\_name}
\sphinxAtStartPar
The file name of the inventory file (optional, default value is “inventory.xml”).
When the file doesn’t exist, the program tries to download the inventory to the file
from the server defined in {\hyperref[\detokenize{configuration:stream-parameters}]{\sphinxcrossref{\DUrole{std,std-ref}{Stream parameters}}}},

\sphinxlineitem{file\_format}
\sphinxAtStartPar
The inventory format (optional, default value is “STATIONXML”).
It is not required when the inventory file exists

\end{description}\end{quote}


\section{Remove response parameters}
\label{\detokenize{configuration:remove-response-parameters}}
\sphinxAtStartPar
The remove response parameters are compatible with the ObsPy response removing
:water\_level: (int) (optional, default value is 128)
:prefilter: (list(4*float)) The prefilter coefficients.
:output: (str) (optional, default value is “VEL”)


\section{Plot parameters}
\label{\detokenize{configuration:plot-parameters}}\begin{description}
\sphinxlineitem{\sphinxstylestrong{do\_not\_draw:}}
\sphinxAtStartPar
(bool)
When this parameter exists and is true, results are not plotted.
This parameter allows turn off plotting without erasing plot configuration
(optional, default value is false)

\sphinxlineitem{\sphinxstylestrong{draw\_the\_signal\_spectrum:}}
\sphinxAtStartPar
(bool)
When this parameter exists and is false,
the signal spectrum is not plotted.
Usually, this parameter is missing and useless (optional, default value is true)

\sphinxlineitem{\sphinxstylestrong{draw\_the\_noise:}}
\sphinxAtStartPar
(bool)
When this parameter exists and is true,
the noise spectrum is plotted (optional, default value is false)

\sphinxlineitem{\sphinxstylestrong{draw\_the\_noise\_uncertainty:}}
\sphinxAtStartPar
(bool) When this parameter exists and is true,
The dotted linie describing the high noise spectrum uncertainty levels is plotted
(optional, default value is false)

\sphinxlineitem{\sphinxstylestrong{draw\_the\_noise\_correction:}}
\sphinxAtStartPar
(bool)
When this parameter exists and is true,
the noise correction value, which needn’t be the same as the noise, is plotted
(optional, default value is false)

\sphinxlineitem{\sphinxstylestrong{draw\_source\_spectrum\_without\_the\_noise:}}
\sphinxAtStartPar
(bool)
When this parameter exists and is true,
the source spectra corrected by Green function but by noise is plotted in green
(optional, default value is false)

\sphinxlineitem{\sphinxstylestrong{draw\_source\_spectrum\_with\_the\_noise:}}
\sphinxAtStartPar
(bool)
When this parameter is missing or is true,
the final source spectrum is plotted
(optional, default value is true)

\sphinxlineitem{\sphinxstylestrong{own\_frequencies:}}
\sphinxAtStartPar
(list(float, float, int))
Used to define spectra frequency for viewing the source function simulation.
consists of low frequency, high frequency, and the number of frequency points, e.g. {[}0.1, 30, 25{]}
(required in the \sphinxtitleref{test\_greens\_function})

\sphinxlineitem{\sphinxstylestrong{view:}}
\sphinxAtStartPar
(str)
Defines the result plotting method
(optional, allowed values “DISP” \sphinxhyphen{} displacement or “VEL” = velocity, default value is “VEL”)

\sphinxlineitem{\sphinxstylestrong{how\_to\_show:}}
\sphinxAtStartPar
(str)
Defines the result plotting method. Three methods can be used
\begin{itemize}
\item {} 
\sphinxAtStartPar
“single figure” \sphinxhyphen{} all stations are plotted in one figure. Each row is one station,

\item {} 
\sphinxAtStartPar
“many figures” \sphinxhyphen{} each station is plotted in separate figures \sphinxhyphen{} signals at the top

\end{itemize}

\end{description}

\sphinxstepscope


\chapter{Input/Output data}
\label{\detokenize{io_data:input-output-data}}\label{\detokenize{io_data:io-data}}\label{\detokenize{io_data::doc}}
\sphinxAtStartPar
General input data are
* ObsPy catalogs
* ObsPy streams
* ObsPy inventory

\sphinxAtStartPar
All data can be read from files or downloaded from seismic servers.
The downloaded depends on the server type,
but preferred are \sphinxhref{http://ds.iris.edu/ds/nodes/dmc/data/formats/miniseed/}{MiniSEED} for waveforms and \sphinxhref{http://docs.fdsn.org/projects/stationxml/en/latest/}{StationXML} for the inventory
and \sphinxhref{https://quake.ethz.ch/quakeml/}{QuakeML} for the catalogue.

\sphinxAtStartPar
Two types of servers are defined
* FDSN web server
* ArcLink server
The FDSN web server communication is supported by ObsPy library.
The ArcLink server is obsolete in the ObsPy, therefore it required an additional library
from the older ObsPy version. Communication with the server is performed in the cache mode
(see Downloading data from servers)


\section{Downloading data from servers}
\label{\detokenize{io_data:downloading-data-from-servers}}
\sphinxAtStartPar
Downloading data from servers is performed in the cache mode
(see Downloading data from servers).
It means that the downloaded are stored in the disk cache, and next downloading
takes data from the file in the cache. It allows faster downloading in repeated calculations
or taking the data in the cache to the computer, which cannot access the server.
The downloaded data are preprocessed before storing on the cache disk. It speedup rereading,
but in the case of changing preprocessing, data must be reloaded.
The cache consists of downloaded stream or inventory files
and the file containing a description of what was downloaded.

\sphinxAtStartPar
Here is the example downloading file:

\begin{sphinxVerbatim}[commandchars=\\\{\}]
\PYG{p}{\PYGZob{}}
    \PYG{l+s+s2}{\PYGZdq{}}\PYG{l+s+s2}{smi:igf.edu.pl/LGCD\PYGZus{}CIBIS\PYGZus{}800586\PYGZus{}PL.KWLC}\PYG{l+s+s2}{\PYGZdq{}}\PYG{p}{:} \PYG{p}{\PYGZob{}}
        \PYG{l+s+s2}{\PYGZdq{}}\PYG{l+s+s2}{source}\PYG{l+s+s2}{\PYGZdq{}}\PYG{p}{:} \PYG{p}{\PYGZob{}}
            \PYG{l+s+s2}{\PYGZdq{}}\PYG{l+s+s2}{source}\PYG{l+s+s2}{\PYGZdq{}}\PYG{p}{:} \PYG{l+s+s2}{\PYGZdq{}}\PYG{l+s+s2}{arclink}\PYG{l+s+s2}{\PYGZdq{}}\PYG{p}{,}
            \PYG{l+s+s2}{\PYGZdq{}}\PYG{l+s+s2}{host}\PYG{l+s+s2}{\PYGZdq{}}\PYG{p}{:} \PYG{l+s+s2}{\PYGZdq{}}\PYG{l+s+s2}{tytan.igf.edu.pl}\PYG{l+s+s2}{\PYGZdq{}}\PYG{p}{,}
            \PYG{l+s+s2}{\PYGZdq{}}\PYG{l+s+s2}{port}\PYG{l+s+s2}{\PYGZdq{}}\PYG{p}{:} \PYG{l+s+s2}{\PYGZdq{}}\PYG{l+s+s2}{18001}\PYG{l+s+s2}{\PYGZdq{}}\PYG{p}{,}
            \PYG{l+s+s2}{\PYGZdq{}}\PYG{l+s+s2}{user}\PYG{l+s+s2}{\PYGZdq{}}\PYG{p}{:} \PYG{l+s+s2}{\PYGZdq{}}\PYG{l+s+s2}{anonymous@igf.edu.pl}\PYG{l+s+s2}{\PYGZdq{}}\PYG{p}{,}
            \PYG{l+s+s2}{\PYGZdq{}}\PYG{l+s+s2}{timeout}\PYG{l+s+s2}{\PYGZdq{}}\PYG{p}{:} \PYG{l+m+mi}{300}\PYG{p}{,}
            \PYG{l+s+s2}{\PYGZdq{}}\PYG{l+s+s2}{net}\PYG{l+s+s2}{\PYGZdq{}}\PYG{p}{:} \PYG{l+s+s2}{\PYGZdq{}}\PYG{l+s+s2}{PL}\PYG{l+s+s2}{\PYGZdq{}}\PYG{p}{,}
            \PYG{l+s+s2}{\PYGZdq{}}\PYG{l+s+s2}{cache}\PYG{l+s+s2}{\PYGZdq{}}\PYG{p}{:} \PYG{l+s+s2}{\PYGZdq{}}\PYG{l+s+s2}{cache\PYGZus{}Mw}\PYG{l+s+s2}{\PYGZdq{}}
        \PYG{p}{\PYGZcb{}}\PYG{p}{,}
        \PYG{l+s+s2}{\PYGZdq{}}\PYG{l+s+s2}{stations}\PYG{l+s+s2}{\PYGZdq{}}\PYG{p}{:} \PYG{p}{[}
            \PYG{l+s+s2}{\PYGZdq{}}\PYG{l+s+s2}{PL.KWLC}\PYG{l+s+s2}{\PYGZdq{}}
        \PYG{p}{]}\PYG{p}{,}
        \PYG{l+s+s2}{\PYGZdq{}}\PYG{l+s+s2}{begin\PYGZus{}time}\PYG{l+s+s2}{\PYGZdq{}}\PYG{p}{:} \PYG{l+s+s2}{\PYGZdq{}}\PYG{l+s+s2}{2015\PYGZhy{}08\PYGZhy{}19T01:06:10.915125}\PYG{l+s+s2}{\PYGZdq{}}\PYG{p}{,}
        \PYG{l+s+s2}{\PYGZdq{}}\PYG{l+s+s2}{end\PYGZus{}time}\PYG{l+s+s2}{\PYGZdq{}}\PYG{p}{:} \PYG{l+s+s2}{\PYGZdq{}}\PYG{l+s+s2}{2015\PYGZhy{}08\PYGZhy{}19T01:06:17.349625}\PYG{l+s+s2}{\PYGZdq{}}\PYG{p}{,}
        \PYG{l+s+s2}{\PYGZdq{}}\PYG{l+s+s2}{file\PYGZus{}name}\PYG{l+s+s2}{\PYGZdq{}}\PYG{p}{:} \PYG{l+s+s2}{\PYGZdq{}}\PYG{l+s+s2}{cache\PYGZus{}Mw/96559972\PYGZhy{}cee2\PYGZhy{}4da9\PYGZhy{}a30e\PYGZhy{}d997dd2736b2.msd}\PYG{l+s+s2}{\PYGZdq{}}\PYG{p}{,}
        \PYG{l+s+s2}{\PYGZdq{}}\PYG{l+s+s2}{preprocess\PYGZus{}name}\PYG{l+s+s2}{\PYGZdq{}}\PYG{p}{:} \PYG{l+s+s2}{\PYGZdq{}}\PYG{l+s+s2}{Mw\PYGZus{}preprocessing\PYGZus{}1}\PYG{l+s+s2}{\PYGZdq{}}\PYG{p}{,}
        \PYG{l+s+s2}{\PYGZdq{}}\PYG{l+s+s2}{processing}\PYG{l+s+s2}{\PYGZdq{}}\PYG{p}{:} \PYG{p}{\PYGZob{}}
            \PYG{l+s+s2}{\PYGZdq{}}\PYG{l+s+s2}{PL.KWLC..EHZ}\PYG{l+s+s2}{\PYGZdq{}}\PYG{p}{:} \PYG{p}{[}
                \PYG{l+s+s2}{\PYGZdq{}}\PYG{l+s+s2}{ObsPy 1.4.1: trim(endtime=UTCDateTime(2015, 8, 19, 1, 6, 17, 344734)::fill\PYGZus{}value=None::nearest\PYGZus{}sample=True::pad=False::starttime=UTCDateTime(2015, 8, 19, 1, 6, 10, 914734))}\PYG{l+s+s2}{\PYGZdq{}}\PYG{p}{,}
                \PYG{l+s+s2}{\PYGZdq{}}\PYG{l+s+s2}{ObsPy 1.4.1: remove\PYGZus{}response(fig=None::inventory=\PYGZlt{}obspy.core.inventory.inventory.Inventory object at 0x000001C9B8CEF2B0\PYGZgt{}::output=}\PYG{l+s+s2}{\PYGZsq{}}\PYG{l+s+s2}{VEL}\PYG{l+s+s2}{\PYGZsq{}}\PYG{l+s+s2}{::plot=False::pre\PYGZus{}filt=[0.01, 0.05, 45, 50]::taper=True::taper\PYGZus{}fraction=0.05::water\PYGZus{}level=128::zero\PYGZus{}mean=True)}\PYG{l+s+s2}{\PYGZdq{}}
            \PYG{p}{]}\PYG{p}{,}
            \PYG{l+s+s2}{\PYGZdq{}}\PYG{l+s+s2}{PL.KWLC..EHN}\PYG{l+s+s2}{\PYGZdq{}}\PYG{p}{:} \PYG{p}{[}
                \PYG{l+s+s2}{\PYGZdq{}}\PYG{l+s+s2}{ObsPy 1.4.1: trim(endtime=UTCDateTime(2015, 8, 19, 1, 6, 17, 344734)::fill\PYGZus{}value=None::nearest\PYGZus{}sample=True::pad=False::starttime=UTCDateTime(2015, 8, 19, 1, 6, 10, 914734))}\PYG{l+s+s2}{\PYGZdq{}}\PYG{p}{,}
                \PYG{l+s+s2}{\PYGZdq{}}\PYG{l+s+s2}{ObsPy 1.4.1: remove\PYGZus{}response(fig=None::inventory=\PYGZlt{}obspy.core.inventory.inventory.Inventory object at 0x000001C9B8CEF2B0\PYGZgt{}::output=}\PYG{l+s+s2}{\PYGZsq{}}\PYG{l+s+s2}{VEL}\PYG{l+s+s2}{\PYGZsq{}}\PYG{l+s+s2}{::plot=False::pre\PYGZus{}filt=[0.01, 0.05, 45, 50]::taper=True::taper\PYGZus{}fraction=0.05::water\PYGZus{}level=128::zero\PYGZus{}mean=True)}\PYG{l+s+s2}{\PYGZdq{}}
            \PYG{p}{]}\PYG{p}{,}
            \PYG{l+s+s2}{\PYGZdq{}}\PYG{l+s+s2}{PL.KWLC..EHE}\PYG{l+s+s2}{\PYGZdq{}}\PYG{p}{:} \PYG{p}{[}
                \PYG{l+s+s2}{\PYGZdq{}}\PYG{l+s+s2}{ObsPy 1.4.1: trim(endtime=UTCDateTime(2015, 8, 19, 1, 6, 17, 344734)::fill\PYGZus{}value=None::nearest\PYGZus{}sample=True::pad=False::starttime=UTCDateTime(2015, 8, 19, 1, 6, 10, 914734))}\PYG{l+s+s2}{\PYGZdq{}}\PYG{p}{,}
                \PYG{l+s+s2}{\PYGZdq{}}\PYG{l+s+s2}{ObsPy 1.4.1: remove\PYGZus{}response(fig=None::inventory=\PYGZlt{}obspy.core.inventory.inventory.Inventory object at 0x000001C9B8CEF2B0\PYGZgt{}::output=}\PYG{l+s+s2}{\PYGZsq{}}\PYG{l+s+s2}{VEL}\PYG{l+s+s2}{\PYGZsq{}}\PYG{l+s+s2}{::plot=False::pre\PYGZus{}filt=[0.01, 0.05, 45, 50]::taper=True::taper\PYGZus{}fraction=0.05::water\PYGZus{}level=128::zero\PYGZus{}mean=True)}\PYG{l+s+s2}{\PYGZdq{}}
            \PYG{p}{]}
        \PYG{p}{\PYGZcb{}}
    \PYG{p}{\PYGZcb{}}\PYG{p}{,}
    \PYG{l+s+s2}{\PYGZdq{}}\PYG{l+s+s2}{smi:igf.edu.pl/LGCD\PYGZus{}CIBIS\PYGZus{}800586\PYGZus{}PL.MOSK}\PYG{l+s+s2}{\PYGZdq{}}\PYG{p}{:} \PYG{p}{\PYGZob{}}
        \PYG{l+s+s2}{\PYGZdq{}}\PYG{l+s+s2}{source}\PYG{l+s+s2}{\PYGZdq{}}\PYG{p}{:} \PYG{p}{\PYGZob{}}
            \PYG{l+s+s2}{\PYGZdq{}}\PYG{l+s+s2}{source}\PYG{l+s+s2}{\PYGZdq{}}\PYG{p}{:} \PYG{l+s+s2}{\PYGZdq{}}\PYG{l+s+s2}{arclink}\PYG{l+s+s2}{\PYGZdq{}}\PYG{p}{,}
            \PYG{l+s+s2}{\PYGZdq{}}\PYG{l+s+s2}{host}\PYG{l+s+s2}{\PYGZdq{}}\PYG{p}{:} \PYG{l+s+s2}{\PYGZdq{}}\PYG{l+s+s2}{tytan.igf.edu.pl}\PYG{l+s+s2}{\PYGZdq{}}\PYG{p}{,}
            \PYG{l+s+s2}{\PYGZdq{}}\PYG{l+s+s2}{port}\PYG{l+s+s2}{\PYGZdq{}}\PYG{p}{:} \PYG{l+s+s2}{\PYGZdq{}}\PYG{l+s+s2}{18001}\PYG{l+s+s2}{\PYGZdq{}}\PYG{p}{,}
            \PYG{l+s+s2}{\PYGZdq{}}\PYG{l+s+s2}{user}\PYG{l+s+s2}{\PYGZdq{}}\PYG{p}{:} \PYG{l+s+s2}{\PYGZdq{}}\PYG{l+s+s2}{anonymous@igf.edu.pl}\PYG{l+s+s2}{\PYGZdq{}}\PYG{p}{,}
            \PYG{l+s+s2}{\PYGZdq{}}\PYG{l+s+s2}{timeout}\PYG{l+s+s2}{\PYGZdq{}}\PYG{p}{:} \PYG{l+m+mi}{300}\PYG{p}{,}
            \PYG{l+s+s2}{\PYGZdq{}}\PYG{l+s+s2}{net}\PYG{l+s+s2}{\PYGZdq{}}\PYG{p}{:} \PYG{l+s+s2}{\PYGZdq{}}\PYG{l+s+s2}{PL}\PYG{l+s+s2}{\PYGZdq{}}\PYG{p}{,}
            \PYG{l+s+s2}{\PYGZdq{}}\PYG{l+s+s2}{cache}\PYG{l+s+s2}{\PYGZdq{}}\PYG{p}{:} \PYG{l+s+s2}{\PYGZdq{}}\PYG{l+s+s2}{cache\PYGZus{}Mw}\PYG{l+s+s2}{\PYGZdq{}}
        \PYG{p}{\PYGZcb{}}\PYG{p}{,}
        \PYG{l+s+s2}{\PYGZdq{}}\PYG{l+s+s2}{stations}\PYG{l+s+s2}{\PYGZdq{}}\PYG{p}{:} \PYG{p}{[}
            \PYG{l+s+s2}{\PYGZdq{}}\PYG{l+s+s2}{PL.MOSK}\PYG{l+s+s2}{\PYGZdq{}}
        \PYG{p}{]}\PYG{p}{,}
        \PYG{l+s+s2}{\PYGZdq{}}\PYG{l+s+s2}{begin\PYGZus{}time}\PYG{l+s+s2}{\PYGZdq{}}\PYG{p}{:} \PYG{l+s+s2}{\PYGZdq{}}\PYG{l+s+s2}{2015\PYGZhy{}08\PYGZhy{}19T01:06:13.726250}\PYG{l+s+s2}{\PYGZdq{}}\PYG{p}{,}
        \PYG{l+s+s2}{\PYGZdq{}}\PYG{l+s+s2}{end\PYGZus{}time}\PYG{l+s+s2}{\PYGZdq{}}\PYG{p}{:} \PYG{l+s+s2}{\PYGZdq{}}\PYG{l+s+s2}{2015\PYGZhy{}08\PYGZhy{}19T01:06:16.183250}\PYG{l+s+s2}{\PYGZdq{}}\PYG{p}{,}
        \PYG{l+s+s2}{\PYGZdq{}}\PYG{l+s+s2}{file\PYGZus{}name}\PYG{l+s+s2}{\PYGZdq{}}\PYG{p}{:} \PYG{l+s+s2}{\PYGZdq{}}\PYG{l+s+s2}{cache\PYGZus{}Mw/4ca4c2fd\PYGZhy{}9131\PYGZhy{}46b7\PYGZhy{}83a2\PYGZhy{}d4874d09102a.msd}\PYG{l+s+s2}{\PYGZdq{}}\PYG{p}{,}
        \PYG{l+s+s2}{\PYGZdq{}}\PYG{l+s+s2}{preprocess\PYGZus{}name}\PYG{l+s+s2}{\PYGZdq{}}\PYG{p}{:} \PYG{l+s+s2}{\PYGZdq{}}\PYG{l+s+s2}{Mw\PYGZus{}preprocessing\PYGZus{}1}\PYG{l+s+s2}{\PYGZdq{}}\PYG{p}{,}
        \PYG{l+s+s2}{\PYGZdq{}}\PYG{l+s+s2}{processing}\PYG{l+s+s2}{\PYGZdq{}}\PYG{p}{:} \PYG{p}{\PYGZob{}}
            \PYG{l+s+s2}{\PYGZdq{}}\PYG{l+s+s2}{PL.MOSK..EHN}\PYG{l+s+s2}{\PYGZdq{}}\PYG{p}{:} \PYG{p}{[}
                \PYG{l+s+s2}{\PYGZdq{}}\PYG{l+s+s2}{ObsPy 1.4.1: trim(endtime=UTCDateTime(2015, 8, 19, 1, 6, 16, 186531)::fill\PYGZus{}value=None::nearest\PYGZus{}sample=True::pad=False::starttime=UTCDateTime(2015, 8, 19, 1, 6, 13, 726531))}\PYG{l+s+s2}{\PYGZdq{}}\PYG{p}{,}
                \PYG{l+s+s2}{\PYGZdq{}}\PYG{l+s+s2}{ObsPy 1.4.1: remove\PYGZus{}response(fig=None::inventory=\PYGZlt{}obspy.core.inventory.inventory.Inventory object at 0x000001C9B8CEF2B0\PYGZgt{}::output=}\PYG{l+s+s2}{\PYGZsq{}}\PYG{l+s+s2}{VEL}\PYG{l+s+s2}{\PYGZsq{}}\PYG{l+s+s2}{::plot=False::pre\PYGZus{}filt=[0.01, 0.05, 45, 50]::taper=True::taper\PYGZus{}fraction=0.05::water\PYGZus{}level=128::zero\PYGZus{}mean=True)}\PYG{l+s+s2}{\PYGZdq{}}
            \PYG{p}{]}\PYG{p}{,}
            \PYG{l+s+s2}{\PYGZdq{}}\PYG{l+s+s2}{PL.MOSK..EHZ}\PYG{l+s+s2}{\PYGZdq{}}\PYG{p}{:} \PYG{p}{[}
                \PYG{l+s+s2}{\PYGZdq{}}\PYG{l+s+s2}{ObsPy 1.4.1: trim(endtime=UTCDateTime(2015, 8, 19, 1, 6, 16, 186531)::fill\PYGZus{}value=None::nearest\PYGZus{}sample=True::pad=False::starttime=UTCDateTime(2015, 8, 19, 1, 6, 13, 726531))}\PYG{l+s+s2}{\PYGZdq{}}\PYG{p}{,}
                \PYG{l+s+s2}{\PYGZdq{}}\PYG{l+s+s2}{ObsPy 1.4.1: remove\PYGZus{}response(fig=None::inventory=\PYGZlt{}obspy.core.inventory.inventory.Inventory object at 0x000001C9B8CEF2B0\PYGZgt{}::output=}\PYG{l+s+s2}{\PYGZsq{}}\PYG{l+s+s2}{VEL}\PYG{l+s+s2}{\PYGZsq{}}\PYG{l+s+s2}{::plot=False::pre\PYGZus{}filt=[0.01, 0.05, 45, 50]::taper=True::taper\PYGZus{}fraction=0.05::water\PYGZus{}level=128::zero\PYGZus{}mean=True)}\PYG{l+s+s2}{\PYGZdq{}}
            \PYG{p}{]}\PYG{p}{,}
            \PYG{l+s+s2}{\PYGZdq{}}\PYG{l+s+s2}{PL.MOSK..EHE}\PYG{l+s+s2}{\PYGZdq{}}\PYG{p}{:} \PYG{p}{[}
                \PYG{l+s+s2}{\PYGZdq{}}\PYG{l+s+s2}{ObsPy 1.4.1: trim(endtime=UTCDateTime(2015, 8, 19, 1, 6, 16, 186531)::fill\PYGZus{}value=None::nearest\PYGZus{}sample=True::pad=False::starttime=UTCDateTime(2015, 8, 19, 1, 6, 13, 726531))}\PYG{l+s+s2}{\PYGZdq{}}\PYG{p}{,}
                \PYG{l+s+s2}{\PYGZdq{}}\PYG{l+s+s2}{ObsPy 1.4.1: remove\PYGZus{}response(fig=None::inventory=\PYGZlt{}obspy.core.inventory.inventory.Inventory object at 0x000001C9B8CEF2B0\PYGZgt{}::output=}\PYG{l+s+s2}{\PYGZsq{}}\PYG{l+s+s2}{VEL}\PYG{l+s+s2}{\PYGZsq{}}\PYG{l+s+s2}{::plot=False::pre\PYGZus{}filt=[0.01, 0.05, 45, 50]::taper=True::taper\PYGZus{}fraction=0.05::water\PYGZus{}level=128::zero\PYGZus{}mean=True)}\PYG{l+s+s2}{\PYGZdq{}}
            \PYG{p}{]}
        \PYG{p}{\PYGZcb{}}
    \PYG{p}{\PYGZcb{}}
\PYG{p}{\PYGZcb{}}
\end{sphinxVerbatim}

\sphinxAtStartPar
The file in JSON format contains a dictionary,
where the dictionary key is the downloading event ID
and the dictionary value is the directory describing
the downloading parameters:
\begin{quote}\begin{description}
\sphinxlineitem{source}
\sphinxAtStartPar
(str) The server parameters,

\sphinxlineitem{station}
\sphinxAtStartPar
(list(str)) The list of station names. It must be the list even if one station is defined.
\sphinxstylestrong{WARNING! The list contains all requested stations, not only downloaded.},

\sphinxlineitem{begin\_time}
\sphinxAtStartPar
(\sphinxhref{https://docs.obspy.org/packages/autogen/obspy.core.utcdatetime.UTCDateTime.html}{UTCDateTime}) The request begin time,

\sphinxlineitem{end\_time}
\sphinxAtStartPar
(\sphinxhref{https://docs.obspy.org/packages/autogen/obspy.core.utcdatetime.UTCDateTime.html}{UTCDateTime}) The request end time,

\sphinxlineitem{file\_name}
\sphinxAtStartPar
(str) The name of the stream file,

\sphinxlineitem{preprocess\_name}
\sphinxAtStartPar
(str) The name of the preprocessing method,

\sphinxlineitem{processing}
\sphinxAtStartPar
(dict(str,list(str))) The preprocessing description \sphinxhyphen{} copy of the trace class
processing instance.

\end{description}\end{quote}


\section{File formats}
\label{\detokenize{io_data:file-formats}}
\sphinxAtStartPar
All the \sphinxhref{https://docs.obspy.org/packages/autogen/obspy.core.stream.read.html}{stream and inventory formats supported by
ObsPy}
can be read from files or they are downloaded from seismic servers. The downloaded depends
on the server type.

\sphinxAtStartPar
Preferred are \sphinxhref{http://ds.iris.edu/ds/nodes/dmc/data/formats/miniseed/}{MiniSEED} for waveforms and \sphinxhref{http://docs.fdsn.org/projects/stationxml/en/latest/}{StationXML} for inventory

\sphinxAtStartPar
ObsPy catalog are read from the \sphinxhref{https://quake.ethz.ch/quakeml/}{QuakeML} files


\section{Output file}
\label{\detokenize{io_data:output-file}}
\sphinxAtStartPar
The output file is the seismic catalogue supplemented with new moment magnitudes
in the \sphinxhref{https://quake.ethz.ch/quakeml/}{QuakeML} format

\sphinxstepscope


\chapter{Mw installation}
\label{\detokenize{installation:mw-installation}}\label{\detokenize{installation:installation}}\label{\detokenize{installation::doc}}
\sphinxAtStartPar
Magnitude Mw requires at least Python 3.7. All the required dependencies
will be downloaded and installed during the setup process.


\section{Installing the latest release}
\label{\detokenize{installation:installing-the-latest-release}}
\sphinxAtStartPar
To keep Anthropogenic Mw updated, run:

\begin{sphinxVerbatim}[commandchars=\\\{\}]
\PYG{n}{pip} \PYG{n}{install} \PYG{o}{\PYGZhy{}}\PYG{o}{\PYGZhy{}}\PYG{n}{upgrade} \PYG{n}{amw}
\end{sphinxVerbatim}

\sphinxAtStartPar
from within your environment.


\subsection{Using pip and PyPI}
\label{\detokenize{installation:using-pip-and-pypi}}
\sphinxAtStartPar
The latest release of Anthropogenic Mw is available on the \sphinxhref{https://pypi.org/project/amw/}{Python Package
Index}.

\sphinxAtStartPar
You can install it easily through \sphinxcode{\sphinxupquote{pip}}:

\begin{sphinxVerbatim}[commandchars=\\\{\}]
\PYG{n}{pip} \PYG{n}{install} \PYG{n}{amw}
\end{sphinxVerbatim}

\sphinxAtStartPar
To upgrade from a previously installed version:

\begin{sphinxVerbatim}[commandchars=\\\{\}]
\PYG{n}{pip} \PYG{n}{install} \PYG{o}{\PYGZhy{}}\PYG{o}{\PYGZhy{}}\PYG{n}{upgrade} \PYG{n}{amw}
\end{sphinxVerbatim}


\section{Installing a developer snapshot}
\label{\detokenize{installation:installing-a-developer-snapshot}}
\sphinxAtStartPar
If you need a recent feature that is not in the latest release (see the
“unreleased” section in {\hyperref[\detokenize{changelog:changelog}]{\sphinxcrossref{\DUrole{std,std-ref}{Changelogs}}}}),
you want to use the more recent development snapshot from the \sphinxhref{https://github.com/JanWiszniowski/amw}{Anthropogenic Mw
GitHub repository}.


\subsection{Using pip}
\label{\detokenize{installation:using-pip}}
\sphinxAtStartPar
The easiest way to install the most recent development snapshot is to download
and install it through \sphinxcode{\sphinxupquote{pip}}, using its builtin \sphinxcode{\sphinxupquote{git}} client:

\begin{sphinxVerbatim}[commandchars=\\\{\}]
\PYG{n}{pip} \PYG{n}{install} \PYG{n}{git}\PYG{o}{+}\PYG{n}{https}\PYG{p}{:}\PYG{o}{/}\PYG{o}{/}\PYG{n}{github}\PYG{o}{.}\PYG{n}{com}\PYG{o}{/}\PYG{n}{JanWiszniowski}\PYG{o}{/}\PYG{n}{amw}\PYG{o}{.}\PYG{n}{git}
\end{sphinxVerbatim}

\sphinxAtStartPar
Run this command again, from time to time, to keep Anthropogenic Mw updated with
the development version.


\subsection{Cloning the Anthropogenic Mw and Mw GitHub repository}
\label{\detokenize{installation:cloning-the-anthropogenic-mw-and-mw-github-repository}}
\sphinxAtStartPar
If you want to take a look at the source code (and possibly modify it 😉),
clone the project using \sphinxcode{\sphinxupquote{git}}:

\begin{sphinxVerbatim}[commandchars=\\\{\}]
\PYG{n}{git} \PYG{n}{clone} \PYG{n}{https}\PYG{p}{:}\PYG{o}{/}\PYG{o}{/}\PYG{n}{github}\PYG{o}{.}\PYG{n}{com}\PYG{o}{/}\PYG{n}{JanWiszniowski}\PYG{o}{/}\PYG{n}{amw}\PYG{o}{.}\PYG{n}{git}
\end{sphinxVerbatim}

\sphinxAtStartPar
or using SSH:

\begin{sphinxVerbatim}[commandchars=\\\{\}]
\PYG{n}{git} \PYG{n}{clone} \PYG{n}{git}\PYG{n+nd}{@github}\PYG{o}{.}\PYG{n}{com}\PYG{p}{:}\PYG{n}{SeismicSource}\PYG{o}{/}\PYG{n}{amw}\PYG{o}{.}\PYG{n}{git}
\end{sphinxVerbatim}

\sphinxAtStartPar
(avoid using the “Download ZIP” option from the green “Code” button, since
version number is lost).

\sphinxAtStartPar
Then, go into the \sphinxcode{\sphinxupquote{amw}} main directory and install the code in
“editable mode” by running:

\begin{sphinxVerbatim}[commandchars=\\\{\}]
\PYG{n}{pip} \PYG{n}{install} \PYG{o}{\PYGZhy{}}\PYG{n}{e} \PYG{o}{.}
\end{sphinxVerbatim}

\sphinxAtStartPar
You can keep your local Anthropogenic Mw repository updated by running \sphinxcode{\sphinxupquote{git pull}}
from time to time. Thanks to \sphinxcode{\sphinxupquote{pip}}’s “editable mode”, you don’t need to
reinstall Anthropogenic Mw after each update.

\sphinxstepscope


\chapter{Getting Help}
\label{\detokenize{getting_help:getting-help}}\label{\detokenize{getting_help:id1}}\label{\detokenize{getting_help::doc}}

\section{🙏 I need help}
\label{\detokenize{getting_help:i-need-help}}
\sphinxAtStartPar
Join the Anthropogenic Mw
\sphinxhref{https://github.com/JanWiszniowski/amw/discussions}{Discussions}
and feel free to ask!


\section{🐞 I found a bug}
\label{\detokenize{getting_help:i-found-a-bug}}
\sphinxAtStartPar
Please open an \sphinxhref{https://github.com/JanWiszniowski/amw/issues}{Issue}.

\sphinxstepscope


\chapter{Mw contributing}
\label{\detokenize{contributing:mw-contributing}}\label{\detokenize{contributing:contributing}}\label{\detokenize{contributing::doc}}
\sphinxAtStartPar
Anthropogenic Mw estimation development happens on
\sphinxhref{https://github.com/JanWiszniowski/amw}{GitHub}.

\sphinxAtStartPar
This is a new solution. So, I’m very open to contributions: if you have new ideas, please open an
\sphinxhref{https://github.com/JanWiszniowski/amw/issues}{Issue}.
Don’t hesitate to send me pull requests with new features and/or bugfixes!

\sphinxstepscope


\chapter{How to Cite}
\label{\detokenize{citing:how-to-cite}}\label{\detokenize{citing:id1}}\label{\detokenize{citing::doc}}
\sphinxAtStartPar
No citation jet

\sphinxstepscope


\chapter{Anthropogenic Mw API}
\label{\detokenize{api:anthropogenic-mw-api}}\label{\detokenize{api::doc}}
\sphinxAtStartPar
Mw has a modular structure. Each module corresponds to a specific
function or class of functions.

\sphinxstepscope


\section{Main module}
\label{\detokenize{api_main:main-module}}\label{\detokenize{api_main:api-main}}\label{\detokenize{api_main::doc}}
\sphinxAtStartPar
The main module, \sphinxcode{\sphinxupquote{spectral\_mw.py}}, contains functions that perform all the operations needed to count Mw magnitudes.
These include finding peaks, taking waveforms, counting spectra, etc.
Mw main module functions are presented below, following the alphabetic order.
The module \sphinxcode{\sphinxupquote{spectral\_mw.py}} can be called directly with the configuration file in JsonFormat
and the catalogue file in the QuakeML XML format as parameters

\sphinxAtStartPar
E.g: \sphinxcode{\sphinxupquote{python3 spectral\_mw.py config.json catalog.xml}}


\subsection{spectral\_mw}
\label{\detokenize{api_main:module-amw.mw.spectral_mw}}\label{\detokenize{api_main:spectral-mw}}\index{module@\spxentry{module}!amw.mw.spectral\_mw@\spxentry{amw.mw.spectral\_mw}}\index{amw.mw.spectral\_mw@\spxentry{amw.mw.spectral\_mw}!module@\spxentry{module}}

\subsubsection{Spectral magnitude estimation for all event in the catalog}
\label{\detokenize{api_main:spectral-magnitude-estimation-for-all-event-in-the-catalog}}\index{MwStreamPreprocessing (class in amw.mw.spectral\_mw)@\spxentry{MwStreamPreprocessing}\spxextra{class in amw.mw.spectral\_mw}}

\begin{fulllineitems}
\phantomsection\label{\detokenize{api_main:amw.mw.spectral_mw.MwStreamPreprocessing}}
\pysigstartsignatures
\pysiglinewithargsret{\sphinxbfcode{\sphinxupquote{class\DUrole{w}{ }}}\sphinxcode{\sphinxupquote{amw.mw.spectral\_mw.}}\sphinxbfcode{\sphinxupquote{MwStreamPreprocessing}}}{\sphinxparam{\DUrole{n}{configuration}}\sphinxparamcomma \sphinxparam{\DUrole{n}{inventory}}}{}
\pysigstopsignatures
\end{fulllineitems}

\index{catalog\_moment\_magnitudes() (in module amw.mw.spectral\_mw)@\spxentry{catalog\_moment\_magnitudes()}\spxextra{in module amw.mw.spectral\_mw}}

\begin{fulllineitems}
\phantomsection\label{\detokenize{api_main:amw.mw.spectral_mw.catalog_moment_magnitudes}}
\pysigstartsignatures
\pysiglinewithargsret{\sphinxcode{\sphinxupquote{amw.mw.spectral\_mw.}}\sphinxbfcode{\sphinxupquote{catalog\_moment\_magnitudes}}}{\sphinxparam{\DUrole{n}{catalog}}\sphinxparamcomma \sphinxparam{\DUrole{n}{configuration}}}{}
\pysigstopsignatures
\sphinxAtStartPar
Function estimate the mw magnitude for all events having origin and picks according the configuration.
As a result the function modifies the catalog and adds mw magnitude to events
\begin{quote}\begin{description}
\sphinxlineitem{Parameters}\begin{itemize}
\item {} 
\sphinxAtStartPar
\sphinxstyleliteralstrong{\sphinxupquote{catalog}} (\sphinxstyleliteralemphasis{\sphinxupquote{ObsPy Catalog}}) \textendash{} The seismic catalog of event the magnitude will be estimated.
The estimated magnitudes will be added to the catalog magnitude.

\item {} 
\sphinxAtStartPar
\sphinxstyleliteralstrong{\sphinxupquote{configuration}} (\sphinxstyleliteralemphasis{\sphinxupquote{dict}}) \textendash{} The configuration container of all parameters dictionary required for the program to work.

\end{itemize}

\sphinxlineitem{Returns}
\sphinxAtStartPar
None

\end{description}\end{quote}

\end{fulllineitems}

\index{station\_moment\_magnitude() (in module amw.mw.spectral\_mw)@\spxentry{station\_moment\_magnitude()}\spxextra{in module amw.mw.spectral\_mw}}

\begin{fulllineitems}
\phantomsection\label{\detokenize{api_main:amw.mw.spectral_mw.station_moment_magnitude}}
\pysigstartsignatures
\pysiglinewithargsret{\sphinxcode{\sphinxupquote{amw.mw.spectral\_mw.}}\sphinxbfcode{\sphinxupquote{station\_moment\_magnitude}}}{\sphinxparam{\DUrole{n}{station\_name}}\sphinxparamcomma \sphinxparam{\DUrole{n}{picks}}\sphinxparamcomma \sphinxparam{\DUrole{n}{event}}\sphinxparamcomma \sphinxparam{\DUrole{n}{origin}}\sphinxparamcomma \sphinxparam{\DUrole{n}{inventory}}\sphinxparamcomma \sphinxparam{\DUrole{n}{configuration}}}{}
\pysigstopsignatures
\sphinxAtStartPar
Function estimate the mw magnitude for one events having origin and picks
according the configuration.
\begin{quote}\begin{description}
\sphinxlineitem{Parameters}\begin{itemize}
\item {} 
\sphinxAtStartPar
\sphinxstyleliteralstrong{\sphinxupquote{station\_name}} \textendash{} The station name in the form ‘NN.SSSS’, where NN is the network code and SSSS is the station\_name code.

\item {} 
\sphinxAtStartPar
\sphinxstyleliteralstrong{\sphinxupquote{station\_name}} \textendash{} str

\item {} 
\sphinxAtStartPar
\sphinxstyleliteralstrong{\sphinxupquote{picks}} (\sphinxstyleliteralemphasis{\sphinxupquote{ObsPy Pick}}) \textendash{} The list of two picks: P and S. The P wave\_name is first the S wave\_name is second.
If the wave\_name is missing there should be None value. At list one wave\_name is required,
but wo picks P and S are recommended. At least one wave\_name must be given. If the P or S wave\_name is missing,
the function tries to determine it based on the earthquake time at the focus and the remaining wave\_name time.

\item {} 
\sphinxAtStartPar
\sphinxstyleliteralstrong{\sphinxupquote{event}} (\sphinxstyleliteralemphasis{\sphinxupquote{ObsPy Event}}) \textendash{} The event object

\item {} 
\sphinxAtStartPar
\sphinxstyleliteralstrong{\sphinxupquote{origin}} (\sphinxstyleliteralemphasis{\sphinxupquote{ObsPy Origin}}) \textendash{} The event origin.

\item {} 
\sphinxAtStartPar
\sphinxstyleliteralstrong{\sphinxupquote{inventory}} (\sphinxstyleliteralemphasis{\sphinxupquote{ObsPy Inventory}}) \textendash{} The inventory Object. It must contain the station\_name inventory

\item {} 
\sphinxAtStartPar
\sphinxstyleliteralstrong{\sphinxupquote{configuration}} (\sphinxstyleliteralemphasis{\sphinxupquote{dict}}) \textendash{} The configuration container of all parameters dictionary required for the program to work.

\end{itemize}

\sphinxlineitem{Returns}
\sphinxAtStartPar
The station magnitude object or None, if the magnitude can not be estimated.

\sphinxlineitem{Return type}
\sphinxAtStartPar
ObsPy.StationMagnitude

\end{description}\end{quote}

\end{fulllineitems}



\subsection{test\_greens\_function}
\label{\detokenize{api_main:module-amw.mw.test_greens_function}}\label{\detokenize{api_main:test-greens-function}}\index{module@\spxentry{module}!amw.mw.test\_greens\_function@\spxentry{amw.mw.test\_greens\_function}}\index{amw.mw.test\_greens\_function@\spxentry{amw.mw.test\_greens\_function}!module@\spxentry{module}}

\subsubsection{The plot of the source spectra}
\label{\detokenize{api_main:the-plot-of-the-source-spectra}}
\sphinxAtStartPar
Program plot the source spectra including the source term and theoretical Green function

\sphinxstepscope


\section{Support modules}
\label{\detokenize{api_support:module-amw.mw.estimation}}\label{\detokenize{api_support:support-modules}}\label{\detokenize{api_support::doc}}\index{module@\spxentry{module}!amw.mw.estimation@\spxentry{amw.mw.estimation}}\index{amw.mw.estimation@\spxentry{amw.mw.estimation}!module@\spxentry{module}}

\subsection{The general spectral magnitude estimation}
\label{\detokenize{api_support:the-general-spectral-magnitude-estimation}}\index{estimate\_mw() (in module amw.mw.estimation)@\spxentry{estimate\_mw()}\spxextra{in module amw.mw.estimation}}

\begin{fulllineitems}
\phantomsection\label{\detokenize{api_support:amw.mw.estimation.estimate_mw}}
\pysigstartsignatures
\pysiglinewithargsret{\sphinxcode{\sphinxupquote{amw.mw.estimation.}}\sphinxbfcode{\sphinxupquote{estimate\_mw}}}{\sphinxparam{\DUrole{n}{signal}}\sphinxparamcomma \sphinxparam{\DUrole{n}{begin\_signal}}\sphinxparamcomma \sphinxparam{\DUrole{n}{picks}}\sphinxparamcomma \sphinxparam{\DUrole{n}{origin}}\sphinxparamcomma \sphinxparam{\DUrole{n}{station\_inventory}}\sphinxparamcomma \sphinxparam{\DUrole{n}{configuration}}}{}
\pysigstopsignatures
\sphinxAtStartPar
Function estimate\_mw estimates either single phase or cumulated phases spectral moment magnitude.
\begin{quote}\begin{description}
\sphinxlineitem{Parameters}\begin{itemize}
\item {} 
\sphinxAtStartPar
\sphinxstyleliteralstrong{\sphinxupquote{signal}} (\sphinxstyleliteralemphasis{\sphinxupquote{ObsPy Stream}}) \textendash{} The signal is the 3D seismic displacement stream, which must cover both the P wave, the S wave,
and the noise before the P onset.

\item {} 
\sphinxAtStartPar
\sphinxstyleliteralstrong{\sphinxupquote{signal}} \textendash{} The signal is the 3D seismic displacement stream, which must cover both the P wave, the S wave,
and the noise before the P onset.

\item {} 
\sphinxAtStartPar
\sphinxstyleliteralstrong{\sphinxupquote{begin\_signal}} (\sphinxstyleliteralemphasis{\sphinxupquote{ObsPy UTCDateTime}}) \textendash{} The first phase time (usually it is P wave) required to select noise before seismic waves onset

\item {} 
\sphinxAtStartPar
\sphinxstyleliteralstrong{\sphinxupquote{picks}} (\sphinxstyleliteralemphasis{\sphinxupquote{list}}\sphinxstyleliteralemphasis{\sphinxupquote{(}}\sphinxstyleliteralemphasis{\sphinxupquote{ObsPy Pick}}\sphinxstyleliteralemphasis{\sphinxupquote{)}}) \textendash{} A list of picks of waves is used for moment magnitude estimation.
It can consist of a single pick P or S, then the magnitude estimation method from the single wave is used,
or two picks P and S for the magnitude estimation based on both waves.

\item {} 
\sphinxAtStartPar
\sphinxstyleliteralstrong{\sphinxupquote{origin}} (\sphinxstyleliteralemphasis{\sphinxupquote{ObsPy Origin}}) \textendash{} The event origin.

\item {} 
\sphinxAtStartPar
\sphinxstyleliteralstrong{\sphinxupquote{station\_inventory}} (\sphinxstyleliteralemphasis{\sphinxupquote{ObsPy Inventory}}) \textendash{} The inventory of the station that the signal was picked on

\item {} 
\sphinxAtStartPar
\sphinxstyleliteralstrong{\sphinxupquote{configuration}} (\sphinxstyleliteralemphasis{\sphinxupquote{dict}}) \textendash{} The configuration container of all parameters dictionary required for the program to work.

\end{itemize}

\sphinxlineitem{Returns}
\sphinxAtStartPar
mw : Estimated moment magnitude
f0 : Source function corner frequency
m0 : Scalar moment
time\_window : The assessed time window of P and S waves

\sphinxlineitem{Return type}
\sphinxAtStartPar
tuple

\end{description}\end{quote}
\begin{description}
\sphinxlineitem{Uses classes :}
\sphinxAtStartPar
DefaultParameters
MwFunctionParameters

\sphinxlineitem{Uses functions :}
\sphinxAtStartPar
get\_source\_par :
get\_margin :
get\_station\_name :
get\_phase\_window
get\_spectrum
get\_noise\_spectrum
get\_minimization\_method
minimization\_method

\end{description}

\end{fulllineitems}

\index{module@\spxentry{module}!amw.mw.parameters@\spxentry{amw.mw.parameters}}\index{amw.mw.parameters@\spxentry{amw.mw.parameters}!module@\spxentry{module}}

\subsection{Functions and classes for preparation of mw estimation procedure parameters}
\label{\detokenize{api_support:functions-and-classes-for-preparation-of-mw-estimation-procedure-parameters}}\label{\detokenize{api_support:module-amw.mw.parameters}}\index{LogMetric (class in amw.mw.parameters)@\spxentry{LogMetric}\spxextra{class in amw.mw.parameters}}

\begin{fulllineitems}
\phantomsection\label{\detokenize{api_support:amw.mw.parameters.LogMetric}}
\pysigstartsignatures
\pysiglinewithargsret{\sphinxbfcode{\sphinxupquote{class\DUrole{w}{ }}}\sphinxcode{\sphinxupquote{amw.mw.parameters.}}\sphinxbfcode{\sphinxupquote{LogMetric}}}{\sphinxparam{\DUrole{n}{p}}\sphinxparamcomma \sphinxparam{\DUrole{n}{weights}\DUrole{o}{=}\DUrole{default_value}{None}}}{}
\pysigstopsignatures
\sphinxAtStartPar
The logarithmic metric class. It computes the distance
\begin{equation*}
\begin{split}\left\| \textbf{x},\textbf{y} \right\|=
\left[\sum_{f=f_{low}}^{f_{high}}{\left| \log\left( x\left(f\right) \right)-
\log\left( y\left(f\right) \right)\right|^p\cdot
w\left(f\right)}\right]^\frac{1}{p}\end{split}
\end{equation*}\begin{quote}\begin{description}
\sphinxlineitem{Parameters}\begin{itemize}
\item {} 
\sphinxAtStartPar
\sphinxstyleliteralstrong{\sphinxupquote{p}} \textendash{} The power of the p\sphinxhyphen{}norm metric class

\item {} 
\sphinxAtStartPar
\sphinxstyleliteralstrong{\sphinxupquote{weights}} (\sphinxstyleliteralemphasis{\sphinxupquote{np.array}}\sphinxstyleliteralemphasis{\sphinxupquote{(}}\sphinxstyleliteralemphasis{\sphinxupquote{float}}\sphinxstyleliteralemphasis{\sphinxupquote{)}}) \textendash{} Weights for spectra frequency comparison. The size must be the same as spectra.
Optional parameter. If missing no weight are applied.

\end{itemize}

\end{description}\end{quote}

\end{fulllineitems}

\index{MwFunctionParameters (class in amw.mw.parameters)@\spxentry{MwFunctionParameters}\spxextra{class in amw.mw.parameters}}

\begin{fulllineitems}
\phantomsection\label{\detokenize{api_support:amw.mw.parameters.MwFunctionParameters}}
\pysigstartsignatures
\pysiglinewithargsret{\sphinxbfcode{\sphinxupquote{class\DUrole{w}{ }}}\sphinxcode{\sphinxupquote{amw.mw.parameters.}}\sphinxbfcode{\sphinxupquote{MwFunctionParameters}}}{\sphinxparam{\DUrole{n}{picks}}\sphinxparamcomma \sphinxparam{\DUrole{n}{station\_name}}\sphinxparamcomma \sphinxparam{\DUrole{n}{signal\_spec}}\sphinxparamcomma \sphinxparam{\DUrole{n}{noise\_spec}}\sphinxparamcomma \sphinxparam{\DUrole{n}{noise\_sd}}\sphinxparamcomma \sphinxparam{\DUrole{n}{freq}}\sphinxparamcomma \sphinxparam{\DUrole{n}{source\_parameters}}\sphinxparamcomma \sphinxparam{\DUrole{n}{station\_inventory}}\sphinxparamcomma \sphinxparam{\DUrole{n}{configuration}}}{}
\pysigstopsignatures
\sphinxAtStartPar
The class keeps all parameters used to estimate the mw and its object is used as optimised function.
\begin{quote}\begin{description}
\sphinxlineitem{Parameters}\begin{itemize}
\item {} 
\sphinxAtStartPar
\sphinxstyleliteralstrong{\sphinxupquote{picks}} (\sphinxstyleliteralemphasis{\sphinxupquote{list}}\sphinxstyleliteralemphasis{\sphinxupquote{(}}\sphinxstyleliteralemphasis{\sphinxupquote{ObsPy.Pick}}\sphinxstyleliteralemphasis{\sphinxupquote{)}}) \textendash{} List of two picks P and S. The P pick must be firsts .

\item {} 
\sphinxAtStartPar
\sphinxstyleliteralstrong{\sphinxupquote{station\_name}} (\sphinxstyleliteralemphasis{\sphinxupquote{str}}) \textendash{} The station name in the form required to find the configuration for the station

\item {} 
\sphinxAtStartPar
\sphinxstyleliteralstrong{\sphinxupquote{signal\_spec}} (\sphinxstyleliteralemphasis{\sphinxupquote{numpy.array}}\sphinxstyleliteralemphasis{\sphinxupquote{(}}\sphinxstyleliteralemphasis{\sphinxupquote{float}}\sphinxstyleliteralemphasis{\sphinxupquote{)}}) \textendash{} The signal spectrum

\item {} 
\sphinxAtStartPar
\sphinxstyleliteralstrong{\sphinxupquote{noise\_spec}} (\sphinxstyleliteralemphasis{\sphinxupquote{numpy.array}}\sphinxstyleliteralemphasis{\sphinxupquote{(}}\sphinxstyleliteralemphasis{\sphinxupquote{float}}\sphinxstyleliteralemphasis{\sphinxupquote{)}}) \textendash{} The noise mean spectrum

\item {} 
\sphinxAtStartPar
\sphinxstyleliteralstrong{\sphinxupquote{noise\_sd}} (\sphinxstyleliteralemphasis{\sphinxupquote{numpy.array}}\sphinxstyleliteralemphasis{\sphinxupquote{(}}\sphinxstyleliteralemphasis{\sphinxupquote{float}}\sphinxstyleliteralemphasis{\sphinxupquote{)}}) \textendash{} The standard deviation of noise spectrum

\item {} 
\sphinxAtStartPar
\sphinxstyleliteralstrong{\sphinxupquote{freq}} (\sphinxstyleliteralemphasis{\sphinxupquote{numpy.array}}\sphinxstyleliteralemphasis{\sphinxupquote{(}}\sphinxstyleliteralemphasis{\sphinxupquote{float}}\sphinxstyleliteralemphasis{\sphinxupquote{)}}) \textendash{} The frequencies that the spectra are compared

\item {} 
\sphinxAtStartPar
\sphinxstyleliteralstrong{\sphinxupquote{source\_parameters}} (\sphinxstyleliteralemphasis{\sphinxupquote{SourceParameters}}) \textendash{} The seismic source parameters required for of mw estimation procedure

\item {} 
\sphinxAtStartPar
\sphinxstyleliteralstrong{\sphinxupquote{station\_inventory}} (\sphinxstyleliteralemphasis{\sphinxupquote{ObsPy Inventory}}) \textendash{} The inventory of the station that the signal was picked and mw is estimated

\item {} 
\sphinxAtStartPar
\sphinxstyleliteralstrong{\sphinxupquote{configuration}} (\sphinxstyleliteralemphasis{\sphinxupquote{dict}}) \textendash{} The configuration container of all parameters dictionary required for the program to work.

\end{itemize}

\end{description}\end{quote}

\sphinxAtStartPar
\sphinxstylestrong{Warning! signal\_spectrum, noise\_spectrum, noise\_sd, frequencies must have the same size}

\end{fulllineitems}

\index{PNormMetric (class in amw.mw.parameters)@\spxentry{PNormMetric}\spxextra{class in amw.mw.parameters}}

\begin{fulllineitems}
\phantomsection\label{\detokenize{api_support:amw.mw.parameters.PNormMetric}}
\pysigstartsignatures
\pysiglinewithargsret{\sphinxbfcode{\sphinxupquote{class\DUrole{w}{ }}}\sphinxcode{\sphinxupquote{amw.mw.parameters.}}\sphinxbfcode{\sphinxupquote{PNormMetric}}}{\sphinxparam{\DUrole{n}{p}\DUrole{p}{:}\DUrole{w}{ }\DUrole{n}{float}}\sphinxparamcomma \sphinxparam{\DUrole{n}{weights}\DUrole{p}{:}\DUrole{w}{ }\DUrole{n}{array\DUrole{w}{ }\DUrole{p}{|}\DUrole{w}{ }None}\DUrole{w}{ }\DUrole{o}{=}\DUrole{w}{ }\DUrole{default_value}{None}}}{}
\pysigstopsignatures
\sphinxAtStartPar
The p\sphinxhyphen{}norm metric class. It computes the distance
\begin{equation*}
\begin{split}\left\| \textbf{x},\textbf{y} \right\|=
\left[\sum_{f=f_{low}}^{f_{high}}{\left|x\left(f\right)-y\left(f\right)\right|^p\cdot
w\left(f\right)}\right]^\frac{1}{p}\end{split}
\end{equation*}\begin{quote}\begin{description}
\sphinxlineitem{Parameters}\begin{itemize}
\item {} 
\sphinxAtStartPar
\sphinxstyleliteralstrong{\sphinxupquote{p}} \textendash{} The power of the p\sphinxhyphen{}norm metric class

\item {} 
\sphinxAtStartPar
\sphinxstyleliteralstrong{\sphinxupquote{weights}} (\sphinxstyleliteralemphasis{\sphinxupquote{np.array}}\sphinxstyleliteralemphasis{\sphinxupquote{(}}\sphinxstyleliteralemphasis{\sphinxupquote{float}}\sphinxstyleliteralemphasis{\sphinxupquote{)}}) \textendash{} Weights for spectra frequency comparison. The size must be the same as spectra.
Optional parameter. If missing no weight are applied.

\end{itemize}

\end{description}\end{quote}

\end{fulllineitems}

\index{get\_correction() (in module amw.mw.parameters)@\spxentry{get\_correction()}\spxextra{in module amw.mw.parameters}}

\begin{fulllineitems}
\phantomsection\label{\detokenize{api_support:amw.mw.parameters.get_correction}}
\pysigstartsignatures
\pysiglinewithargsret{\sphinxcode{\sphinxupquote{amw.mw.parameters.}}\sphinxbfcode{\sphinxupquote{get\_correction}}}{\sphinxparam{\DUrole{n}{phase\_name}}\sphinxparamcomma \sphinxparam{\DUrole{n}{station\_parameters}}\sphinxparamcomma \sphinxparam{\DUrole{n}{frequencies}}\sphinxparamcomma \sphinxparam{\DUrole{n}{travel\_time}}}{}
\pysigstopsignatures
\sphinxAtStartPar
Calculate the internal dumping, near surface amplification and frequency dumping in frequency domain
\begin{quote}\begin{description}
\sphinxlineitem{Parameters}\begin{itemize}
\item {} 
\sphinxAtStartPar
\sphinxstyleliteralstrong{\sphinxupquote{phase\_name}} (\sphinxstyleliteralemphasis{\sphinxupquote{str}}) \textendash{} The name of the phase (‘P or ‘S’). In the case of ‘P’ the internal dumping, etc. are calculating
for radial component of the signal, which is the P wave in far field. In the case of ‘S’ the internal dumping,
etc. is calculating for transversal component of the signal, which is the S wave in far field.

\item {} 
\sphinxAtStartPar
\sphinxstyleliteralstrong{\sphinxupquote{station\_parameters}} (\sphinxstyleliteralemphasis{\sphinxupquote{DefaultParameters}}) \textendash{} The station parameter

\item {} 
\sphinxAtStartPar
\sphinxstyleliteralstrong{\sphinxupquote{frequencies}} (\sphinxstyleliteralemphasis{\sphinxupquote{numpy.array}}\sphinxstyleliteralemphasis{\sphinxupquote{(}}\sphinxstyleliteralemphasis{\sphinxupquote{float}}\sphinxstyleliteralemphasis{\sphinxupquote{)}}) \textendash{} Frequencies for which the correction is calculated

\item {} 
\sphinxAtStartPar
\sphinxstyleliteralstrong{\sphinxupquote{travel\_time}} (\sphinxstyleliteralemphasis{\sphinxupquote{float}}) \textendash{} The signal travel time

\end{itemize}

\sphinxlineitem{Returns}
\sphinxAtStartPar
The correction in frequency domain

\sphinxlineitem{Return type}
\sphinxAtStartPar
numpy.array(float)

\end{description}\end{quote}

\end{fulllineitems}

\index{get\_far\_response() (in module amw.mw.parameters)@\spxentry{get\_far\_response()}\spxextra{in module amw.mw.parameters}}

\begin{fulllineitems}
\phantomsection\label{\detokenize{api_support:amw.mw.parameters.get_far_response}}
\pysigstartsignatures
\pysiglinewithargsret{\sphinxcode{\sphinxupquote{amw.mw.parameters.}}\sphinxbfcode{\sphinxupquote{get\_far\_response}}}{\sphinxparam{\DUrole{n}{travel\_time}}\sphinxparamcomma \sphinxparam{\DUrole{n}{rho}}\sphinxparamcomma \sphinxparam{\DUrole{n}{r}}\sphinxparamcomma \sphinxparam{\DUrole{n}{fault\_v}}\sphinxparamcomma \sphinxparam{\DUrole{n}{omega}}}{}
\pysigstopsignatures
\sphinxAtStartPar
The get\_far\_response function calculates the Green function in the far field \(G^{\left(c\right)far}\)
except the internal dumping and surface effect for phase P or S marked as \(\left(c\right)\) according to:
\begin{equation*}
\begin{split}G^{\left(c\right)far}=\frac{j\omega \exp\left(-j\omega T_c\right)}{4\pi\rho rv_c^3},\end{split}
\end{equation*}
\sphinxAtStartPar
where
\(c\) is the wave name (P or S),
\(\omega\) is the circular frequency, \(\omega = 2j\pi f\),
\(r\) is the hypocentral distance,
\(T_c\) is the phase travel time,
\(v_c\) is the phase velocity at the source,
\(\rho\) is the density at the source.
\begin{quote}\begin{description}
\sphinxlineitem{Parameters}\begin{itemize}
\item {} 
\sphinxAtStartPar
\sphinxstyleliteralstrong{\sphinxupquote{travel\_time}} (\sphinxstyleliteralemphasis{\sphinxupquote{float}}) \textendash{} The phase travel time

\item {} 
\sphinxAtStartPar
\sphinxstyleliteralstrong{\sphinxupquote{rho}} (\sphinxstyleliteralemphasis{\sphinxupquote{float}}) \textendash{} The density at the source {[}kg/m\textasciicircum{}3{]}

\item {} 
\sphinxAtStartPar
\sphinxstyleliteralstrong{\sphinxupquote{r}} (\sphinxstyleliteralemphasis{\sphinxupquote{float}}) \textendash{} The hipocentral distance {[}m{]}

\item {} 
\sphinxAtStartPar
\sphinxstyleliteralstrong{\sphinxupquote{fault\_v}} \textendash{} The phase velocity at the source {[}m/s{]}

\item {} 
\sphinxAtStartPar
\sphinxstyleliteralstrong{\sphinxupquote{fault\_v}} \textendash{} float

\item {} 
\sphinxAtStartPar
\sphinxstyleliteralstrong{\sphinxupquote{omega}} (\sphinxstyleliteralemphasis{\sphinxupquote{numpy.array}}\sphinxstyleliteralemphasis{\sphinxupquote{(}}\sphinxstyleliteralemphasis{\sphinxupquote{complex}}\sphinxstyleliteralemphasis{\sphinxupquote{)}}) \textendash{} The circular frequencies, for which the response is counted. \(\omega = 2j\pi f\)

\end{itemize}

\sphinxlineitem{Returns}
\sphinxAtStartPar
The far field part Green one phase function

\sphinxlineitem{Return type}
\sphinxAtStartPar
numpy.array(complex)

\end{description}\end{quote}

\end{fulllineitems}

\index{get\_intermediate\_response() (in module amw.mw.parameters)@\spxentry{get\_intermediate\_response()}\spxextra{in module amw.mw.parameters}}

\begin{fulllineitems}
\phantomsection\label{\detokenize{api_support:amw.mw.parameters.get_intermediate_response}}
\pysigstartsignatures
\pysiglinewithargsret{\sphinxcode{\sphinxupquote{amw.mw.parameters.}}\sphinxbfcode{\sphinxupquote{get\_intermediate\_response}}}{\sphinxparam{\DUrole{n}{travel\_time}}\sphinxparamcomma \sphinxparam{\DUrole{n}{rho}}\sphinxparamcomma \sphinxparam{\DUrole{n}{r}}\sphinxparamcomma \sphinxparam{\DUrole{n}{fault\_v}}\sphinxparamcomma \sphinxparam{\DUrole{n}{omega}}}{}
\pysigstopsignatures
\sphinxAtStartPar
The get\_intermediate\_response function calculates the radial or transversal component of Green function
for phase P or S in the intermediate field \(G_x^{\left(c\right)inter}\)
except the internal dumping and surface effect for phase P or S marked as \(\left(c\right)\) according to:
\begin{equation*}
\begin{split}G_x^{\left(c\right)inter}=\frac{exp\left(-j\omega T_c\right)}{4\pi\rho r^2 v_c^2},\end{split}
\end{equation*}
\sphinxAtStartPar
where
\(c\) is the wave name (P or S),
\(\omega\) is the circular frequency, \(\omega = 2j\pi f\),
\(r\) is the hipocentral distance,
\(T_c\) is the phase travel time,
\(v_c\) is the phase velocity at the source,
\(\rho\) is the density at the source
\(x\) describes the signal component (radial or transversal).
\begin{quote}\begin{description}
\sphinxlineitem{Parameters}\begin{itemize}
\item {} 
\sphinxAtStartPar
\sphinxstyleliteralstrong{\sphinxupquote{travel\_time}} (\sphinxstyleliteralemphasis{\sphinxupquote{float}}) \textendash{} The phase travel time

\item {} 
\sphinxAtStartPar
\sphinxstyleliteralstrong{\sphinxupquote{rho}} (\sphinxstyleliteralemphasis{\sphinxupquote{float}}) \textendash{} The density at the source {[}kg/m\textasciicircum{}3{]}

\item {} 
\sphinxAtStartPar
\sphinxstyleliteralstrong{\sphinxupquote{r}} (\sphinxstyleliteralemphasis{\sphinxupquote{float}}) \textendash{} The hipocentral distance {[}m{]}

\item {} 
\sphinxAtStartPar
\sphinxstyleliteralstrong{\sphinxupquote{fault\_v}} (\sphinxstyleliteralemphasis{\sphinxupquote{float}}) \textendash{} The phase velocity at the source {[}m/s{]}

\item {} 
\sphinxAtStartPar
\sphinxstyleliteralstrong{\sphinxupquote{omega}} (\sphinxstyleliteralemphasis{\sphinxupquote{numpy.array}}\sphinxstyleliteralemphasis{\sphinxupquote{(}}\sphinxstyleliteralemphasis{\sphinxupquote{complex}}\sphinxstyleliteralemphasis{\sphinxupquote{)}}) \textendash{} The circular frequencies, for which the response is counted. \(\omega = 2j\pi f\)

\end{itemize}

\sphinxlineitem{Returns}
\sphinxAtStartPar
The intermediate field part Green function of one phase

\sphinxlineitem{Return type}
\sphinxAtStartPar
numpy array(complex)

\end{description}\end{quote}

\end{fulllineitems}

\index{get\_near\_response() (in module amw.mw.parameters)@\spxentry{get\_near\_response()}\spxextra{in module amw.mw.parameters}}

\begin{fulllineitems}
\phantomsection\label{\detokenize{api_support:amw.mw.parameters.get_near_response}}
\pysigstartsignatures
\pysiglinewithargsret{\sphinxcode{\sphinxupquote{amw.mw.parameters.}}\sphinxbfcode{\sphinxupquote{get\_near\_response}}}{\sphinxparam{\DUrole{n}{picks}}\sphinxparamcomma \sphinxparam{\DUrole{n}{source\_parameters}}\sphinxparamcomma \sphinxparam{\DUrole{n}{station\_inventory}}\sphinxparamcomma \sphinxparam{\DUrole{n}{omega}}}{}
\pysigstopsignatures
\sphinxAtStartPar
The get\_near\_response function calculates the common (radial and) transversal component of Green function
in the intermediate field except the internal dumping and surface according to:
\begin{equation*}
\begin{split}G^{near}=\frac{\left(\omega T_P+1\right)exp\left(-\omega T_P\right)
-\left(\omega T_S+1\right)exp\left(-\omega T_S\right)}{4\omega ^2\rho r^4}\end{split}
\end{equation*}
\sphinxAtStartPar
where \(\omega\) is the circular frequency, \(\omega = 2j\pi f\), \(r\) is the hipocentral distance,
\(T_P\) is the P phase travel time, \(T_S\) is the P phase travel time, \(\rho\) is the density
at the source
\begin{quote}\begin{description}
\sphinxlineitem{Parameters}\begin{itemize}
\item {} 
\sphinxAtStartPar
\sphinxstyleliteralstrong{\sphinxupquote{picks}} (\sphinxstyleliteralemphasis{\sphinxupquote{list}}\sphinxstyleliteralemphasis{\sphinxupquote{(}}\sphinxstyleliteralemphasis{\sphinxupquote{ObsPy Pick}}\sphinxstyleliteralemphasis{\sphinxupquote{)}}) \textendash{} A list of picks of waves in the near field.
It must consist of two pick P or S and P must the first.

\item {} 
\sphinxAtStartPar
\sphinxstyleliteralstrong{\sphinxupquote{source\_parameters}} (\sphinxstyleliteralemphasis{\sphinxupquote{SourceParameters}}) \textendash{} The seismic source parameters required for of mw estimation procedure

\item {} 
\sphinxAtStartPar
\sphinxstyleliteralstrong{\sphinxupquote{station\_inventory}} (\sphinxstyleliteralemphasis{\sphinxupquote{ObsPy Inventory}}) \textendash{} The inventory of the station that the signal was picked and mw is estimated

\item {} 
\sphinxAtStartPar
\sphinxstyleliteralstrong{\sphinxupquote{omega}} (\sphinxstyleliteralemphasis{\sphinxupquote{numpy.array}}\sphinxstyleliteralemphasis{\sphinxupquote{(}}\sphinxstyleliteralemphasis{\sphinxupquote{complex}}\sphinxstyleliteralemphasis{\sphinxupquote{)}}) \textendash{} The circular frequencies, for which the response is counted. \(\omega = 2j\pi f\)

\end{itemize}

\sphinxlineitem{Returns}
\sphinxAtStartPar
The near field part Green function

\sphinxlineitem{Return type}
\sphinxAtStartPar
numpy.array(complex)

\end{description}\end{quote}

\end{fulllineitems}

\index{get\_phase\_response() (in module amw.mw.parameters)@\spxentry{get\_phase\_response()}\spxextra{in module amw.mw.parameters}}

\begin{fulllineitems}
\phantomsection\label{\detokenize{api_support:amw.mw.parameters.get_phase_response}}
\pysigstartsignatures
\pysiglinewithargsret{\sphinxcode{\sphinxupquote{amw.mw.parameters.}}\sphinxbfcode{\sphinxupquote{get\_phase\_response}}}{\sphinxparam{\DUrole{n}{pick}}\sphinxparamcomma \sphinxparam{\DUrole{n}{source\_parameters}}\sphinxparamcomma \sphinxparam{\DUrole{n}{station\_inventory}}\sphinxparamcomma \sphinxparam{\DUrole{n}{station\_parameters}}\sphinxparamcomma \sphinxparam{\DUrole{n}{frequencies}}}{}
\pysigstopsignatures
\sphinxAtStartPar
Function get\_phase\_response computes radial and transversal frequency responses of the seismic phase
at the station to the source frequency function:
\begin{equation*}
\begin{split}G^{\left(c\right)}\left(f\right)A^{\left(c\right)}\left(f\right)R\left(f\right)I\left(f\right),\end{split}
\end{equation*}
\sphinxAtStartPar
where \(\left(c\right)\) is the phase name
\begin{description}
\sphinxlineitem{The response consists of:}\begin{itemize}
\item {} 
\sphinxAtStartPar
Green function \(G^{\left(c\right)}\), which for P wave is defined as

\end{itemize}

\end{description}
\begin{align*}\!\begin{aligned}
G_R^{\left(P\right)}\left(f\right)= G_R^{\left(P\right)far}\left(f\right)
+G_R^{\left(P\right)inter}\left(f\right)\\
G_T^{\left(P\right)}\left(f\right) = G_T^{\left(P\right)inter}\left(f\right)\\
\end{aligned}\end{align*}
\sphinxAtStartPar
where \(G_R^{\left(P\right)far} = G^{\left(P\right)far}R_R^{far}\)
is the radial component of the P wave in the far field,
\(G_R^{\left(P\right)inter} = G^{\left(P\right)inter}R_R^{inter}\)
is the radial component of the P wave in the intermediate field,
and \(G_T^{\left(P\right)inter} = G^{\left(P\right)inter}R_T^{inter}\)
is the transversal component of the P wave in intermediate field.
\(R_R^{far}= R^{(P)}\) is the P wave average radiation coefficient in the far field.
For S wave, it is defined as
\begin{align*}\!\begin{aligned}
G_R^{(S)}\left(f\right) = G_R^{\left(S\right)inter}\left(f\right)\\
G_T^{(S)}\left(f\right) = G_T^{\left(S\right)far}\left(f\right)+G_T^{\left(S\right)inter}\left(f\right)\\
\end{aligned}\end{align*}
\sphinxAtStartPar
where \(G_R^{\left(S\right)inter} = G^{\left(S\right)inter}R_R^{inter}\)
is the radial component of the S wave in intermediate field,
\(G_T^{\left(S\right)far} = G^{\left(S\right)far}R_T^{far}\)
is the transversal component of the S wave in the far field,
and \(G_T^{\left(S\right)inter} = G^{\left(S\right)inter}R_T^{inter}\)
is the transversal component of the S wave in the intermediate field.
\(R_T^{far}= R^{(S)}\) is the S wave average radiation coefficient in the far field.

\sphinxAtStartPar
Inelastic (internal) dumping \(A^{\left(c\right)}\) is defined as
\begin{equation*}
\begin{split}A\left(f\right)=exp\left(\frac{-\pi T_cf}{Q^{\left(c\right)}\left(f\right)}\right),\end{split}
\end{equation*}
\sphinxAtStartPar
where
\begin{equation*}
\begin{split}Q^{\left(c\right)}\left(f\right)=Q_0^{\left(c\right)}\left(\frac{f_q+f}{f_q}\right)^\vartheta;\end{split}
\end{equation*}
\sphinxAtStartPar
or
\begin{equation*}
\begin{split}Q\left(f\right)=Q_0^{\left(c\right)}f^\vartheta;\end{split}
\end{equation*}
\sphinxAtStartPar
The near\sphinxhyphen{}surface losses and free surface amplification is assumed by
\begin{equation*}
\begin{split}R\left(f\right)=R_c\exp\left(-\pi \kappa f\right).\end{split}
\end{equation*}
\sphinxAtStartPar
The instrument response is \(I\left(f\right)\).
\begin{quote}\begin{description}
\sphinxlineitem{Parameters}\begin{itemize}
\item {} 
\sphinxAtStartPar
\sphinxstyleliteralstrong{\sphinxupquote{pick}} (\sphinxstyleliteralemphasis{\sphinxupquote{str}}) \textendash{} The P or S pick name

\item {} 
\sphinxAtStartPar
\sphinxstyleliteralstrong{\sphinxupquote{source\_parameters}} (\sphinxstyleliteralemphasis{\sphinxupquote{SourceParameters}}) \textendash{} The seismic source parameters required for of mw estimation procedure

\item {} 
\sphinxAtStartPar
\sphinxstyleliteralstrong{\sphinxupquote{station\_inventory}} (\sphinxstyleliteralemphasis{\sphinxupquote{ObsPy Inventory}}) \textendash{} The inventory of the station that the signal was picked and mw is estimated

\item {} 
\sphinxAtStartPar
\sphinxstyleliteralstrong{\sphinxupquote{station\_parameters}} \textendash{} The reference to the station\_name (or default)

\item {} 
\sphinxAtStartPar
\sphinxstyleliteralstrong{\sphinxupquote{frequencies}} (\sphinxstyleliteralemphasis{\sphinxupquote{numpy.array}}\sphinxstyleliteralemphasis{\sphinxupquote{(}}\sphinxstyleliteralemphasis{\sphinxupquote{float}}\sphinxstyleliteralemphasis{\sphinxupquote{)}}) \textendash{} The frequencies, for which the response is counted

\end{itemize}

\sphinxlineitem{Returns}
\sphinxAtStartPar
Tuple of two numpy arrays of complex radial and transversal response in frequency domain.

\sphinxlineitem{Return type}
\sphinxAtStartPar
tuple(numpy.array(float), numpy.array(float))

\end{description}\end{quote}

\end{fulllineitems}

\index{get\_phases\_response() (in module amw.mw.parameters)@\spxentry{get\_phases\_response()}\spxextra{in module amw.mw.parameters}}

\begin{fulllineitems}
\phantomsection\label{\detokenize{api_support:amw.mw.parameters.get_phases_response}}
\pysigstartsignatures
\pysiglinewithargsret{\sphinxcode{\sphinxupquote{amw.mw.parameters.}}\sphinxbfcode{\sphinxupquote{get\_phases\_response}}}{\sphinxparam{\DUrole{n}{picks}}\sphinxparamcomma \sphinxparam{\DUrole{n}{source\_parameters}}\sphinxparamcomma \sphinxparam{\DUrole{n}{station\_inventory}}\sphinxparamcomma \sphinxparam{\DUrole{n}{station\_parameters}}\sphinxparamcomma \sphinxparam{\DUrole{n}{frequencies}}}{}
\pysigstopsignatures
\sphinxAtStartPar
Function get\_phases\_response computes radial and transversal frequency responses
of the cumulated seismic phases P and S at the station to the source frequency function:
\begin{align*}\!\begin{aligned}
G_R^{(P+S)}\left(f\right)=G_R^{(P)far}\left(f\right)+G_R^{\left(P\right)inter}\left(f\right)
+G_R^{\left(S\right)inter}\left(f\right)+G_R^{near}\left(f\right)\\
G_T^{(P+S)}\left(f\right)=G_T^{\left(P\right)inter}\left(f\right)+G_T^{(S)far}\left(f\right)
+G_T^{\left(S\right)inter}\left(f\right)+G_T^{near}\left(f\right),\\
\end{aligned}\end{align*}
\sphinxAtStartPar
where
\(G_T^{near} = G^{near}R_T^{near}\)
is the transversal component in the near field
and \(G_R^{near} = G^{near}R_R^{near}\)
is the radial component in the near field.
The remaining components are defined in the get\_phase\_response function.
\begin{quote}\begin{description}
\sphinxlineitem{Parameters}\begin{itemize}
\item {} 
\sphinxAtStartPar
\sphinxstyleliteralstrong{\sphinxupquote{picks}} \textendash{} List of two P and S picks

\item {} 
\sphinxAtStartPar
\sphinxstyleliteralstrong{\sphinxupquote{source\_parameters}} (\sphinxstyleliteralemphasis{\sphinxupquote{SourceParameters}}) \textendash{} The seismic source parameters required for of mw estimation procedure

\item {} 
\sphinxAtStartPar
\sphinxstyleliteralstrong{\sphinxupquote{station\_inventory}} (\sphinxstyleliteralemphasis{\sphinxupquote{ObsPy Inventory}}) \textendash{} The inventory of the station that the signal was picked and mw is estimated

\item {} 
\sphinxAtStartPar
\sphinxstyleliteralstrong{\sphinxupquote{station\_parameters}} \textendash{} The reference to the station\_name (or default)

\item {} 
\sphinxAtStartPar
\sphinxstyleliteralstrong{\sphinxupquote{frequencies}} \textendash{} The frequencies, for which the response is counted

\end{itemize}

\sphinxlineitem{Returns}
\sphinxAtStartPar
Tuple of two numpy arrays of complex radial and transversal response in frequency domain.

\sphinxlineitem{Return type}
\sphinxAtStartPar
tuple(numpy.array(float), numpy.array(float))

\end{description}\end{quote}

\end{fulllineitems}

\index{get\_travel\_time() (in module amw.mw.parameters)@\spxentry{get\_travel\_time()}\spxextra{in module amw.mw.parameters}}

\begin{fulllineitems}
\phantomsection\label{\detokenize{api_support:amw.mw.parameters.get_travel_time}}
\pysigstartsignatures
\pysiglinewithargsret{\sphinxcode{\sphinxupquote{amw.mw.parameters.}}\sphinxbfcode{\sphinxupquote{get\_travel\_time}}}{\sphinxparam{\DUrole{n}{pick}}\sphinxparamcomma \sphinxparam{\DUrole{n}{source\_parameters}}\sphinxparamcomma \sphinxparam{\DUrole{n}{station\_inventory}}\sphinxparamcomma \sphinxparam{\DUrole{n}{use\_arrivals}\DUrole{o}{=}\DUrole{default_value}{False}}}{}
\pysigstopsignatures
\sphinxAtStartPar
Calculate the travel time of the picket wave from the source to the station,
hipocentral distance to the station, and phase velocity at the fault
\begin{quote}\begin{description}
\sphinxlineitem{Parameters}\begin{itemize}
\item {} 
\sphinxAtStartPar
\sphinxstyleliteralstrong{\sphinxupquote{use\_arrivals}} (\sphinxstyleliteralemphasis{\sphinxupquote{bool}}) \textendash{} Option whether to get travel time from origin arrivals.
If it is false thr travel time is computed from source and station coordinates.

\item {} 
\sphinxAtStartPar
\sphinxstyleliteralstrong{\sphinxupquote{pick}} (\sphinxstyleliteralemphasis{\sphinxupquote{ObsPy Pick}}) \textendash{} The pick of the wave that the travel time is assessed

\item {} 
\sphinxAtStartPar
\sphinxstyleliteralstrong{\sphinxupquote{source\_parameters}} (\sphinxstyleliteralemphasis{\sphinxupquote{SourceParameters}}) \textendash{} The seismic source parameters required for of mw estimation procedure

\item {} 
\sphinxAtStartPar
\sphinxstyleliteralstrong{\sphinxupquote{station\_inventory}} (\sphinxstyleliteralemphasis{\sphinxupquote{ObsPy Inventory}}) \textendash{} The inventory of the station that the signal was picked and mw is estimated

\end{itemize}

\sphinxlineitem{Returns}
\sphinxAtStartPar
The phase travel time {[}s{]}, hipocentral distance {[}m{]}, phase velocity at the fault {[}m/s{]}.

\sphinxlineitem{Return type}
\sphinxAtStartPar
tuple(float, float, float)

\end{description}\end{quote}

\end{fulllineitems}

\index{module@\spxentry{module}!amw.mw.plot@\spxentry{amw.mw.plot}}\index{amw.mw.plot@\spxentry{amw.mw.plot}!module@\spxentry{module}}

\subsection{The plot functions}
\label{\detokenize{api_support:the-plot-functions}}\label{\detokenize{api_support:module-amw.mw.plot}}\index{PlotMw (class in amw.mw.plot)@\spxentry{PlotMw}\spxextra{class in amw.mw.plot}}

\begin{fulllineitems}
\phantomsection\label{\detokenize{api_support:amw.mw.plot.PlotMw}}
\pysigstartsignatures
\pysiglinewithargsret{\sphinxbfcode{\sphinxupquote{class\DUrole{w}{ }}}\sphinxcode{\sphinxupquote{amw.mw.plot.}}\sphinxbfcode{\sphinxupquote{PlotMw}}}{\sphinxparam{\DUrole{n}{configuration}}}{}
\pysigstopsignatures
\sphinxAtStartPar
Class used to plot results of spectral magnitude estimation
\begin{quote}\begin{description}
\sphinxlineitem{Parameters}
\sphinxAtStartPar
\sphinxstyleliteralstrong{\sphinxupquote{configuration}} (\sphinxstyleliteralemphasis{\sphinxupquote{dict}}) \textendash{} The full local mw configuration. The class uses the ‘plot’ subdirectory
and ‘method’ describing the magnitude estimation method for preparing appropriate figure.

\end{description}\end{quote}
\index{format\_plot() (amw.mw.plot.PlotMw method)@\spxentry{format\_plot()}\spxextra{amw.mw.plot.PlotMw method}}

\begin{fulllineitems}
\phantomsection\label{\detokenize{api_support:amw.mw.plot.PlotMw.format_plot}}
\pysigstartsignatures
\pysiglinewithargsret{\sphinxbfcode{\sphinxupquote{format\_plot}}}{\sphinxparam{\DUrole{n}{what\_to\_show}}}{}
\pysigstopsignatures
\sphinxAtStartPar
Does nothing. Prepared to reformat final view
\begin{quote}\begin{description}
\sphinxlineitem{Parameters}
\sphinxAtStartPar
\sphinxstyleliteralstrong{\sphinxupquote{what\_to\_show}} (\sphinxstyleliteralemphasis{\sphinxupquote{str}}) \textendash{} Condition what to show. I can be only values ‘single figure’, ‘many figures’ or
‘many figures inline’. Format is performed if the parameter equals with the ‘what to show’ ‘plot’
parameter in the configuration.

\end{description}\end{quote}

\end{fulllineitems}

\index{plot\_results() (amw.mw.plot.PlotMw method)@\spxentry{plot\_results()}\spxextra{amw.mw.plot.PlotMw method}}

\begin{fulllineitems}
\phantomsection\label{\detokenize{api_support:amw.mw.plot.PlotMw.plot_results}}
\pysigstartsignatures
\pysiglinewithargsret{\sphinxbfcode{\sphinxupquote{plot\_results}}}{\sphinxparam{\DUrole{n}{m0}}\sphinxparamcomma \sphinxparam{\DUrole{n}{f0}}\sphinxparamcomma \sphinxparam{\DUrole{n}{function\_parameters}}}{}
\pysigstopsignatures
\sphinxAtStartPar
Plots spectra of estimation results. It can plot:
\begin{itemize}
\item {} 
\sphinxAtStartPar
The signal spectrum,

\item {} 
\sphinxAtStartPar
The source spectrum at the station

\item {} 
\sphinxAtStartPar
The mean value of the noise spectrum

\item {} 
\sphinxAtStartPar
The uncertainty range of the noise spectrum

\end{itemize}
\begin{quote}\begin{description}
\sphinxlineitem{Parameters}\begin{itemize}
\item {} 
\sphinxAtStartPar
\sphinxstyleliteralstrong{\sphinxupquote{m0}} (\sphinxstyleliteralemphasis{\sphinxupquote{float}}) \textendash{} Scalar moment magnitude \(M_0\)

\item {} 
\sphinxAtStartPar
\sphinxstyleliteralstrong{\sphinxupquote{f0}} (\sphinxstyleliteralemphasis{\sphinxupquote{float}}) \textendash{} Cornel focal function frequency \(f_0\)

\item {} 
\sphinxAtStartPar
\sphinxstyleliteralstrong{\sphinxupquote{function\_parameters}} ({\hyperref[\detokenize{api_support:amw.mw.parameters.MwFunctionParameters}]{\sphinxcrossref{\sphinxstyleliteralemphasis{\sphinxupquote{MwFunctionParameters}}}}}) \textendash{} All parameters used to estimate the mw

\end{itemize}

\end{description}\end{quote}

\end{fulllineitems}

\index{plot\_seismogram() (amw.mw.plot.PlotMw method)@\spxentry{plot\_seismogram()}\spxextra{amw.mw.plot.PlotMw method}}

\begin{fulllineitems}
\phantomsection\label{\detokenize{api_support:amw.mw.plot.PlotMw.plot_seismogram}}
\pysigstartsignatures
\pysiglinewithargsret{\sphinxbfcode{\sphinxupquote{plot\_seismogram}}}{\sphinxparam{\DUrole{n}{picks}}\sphinxparamcomma \sphinxparam{\DUrole{n}{traces}}\sphinxparamcomma \sphinxparam{\DUrole{n}{stream}}\sphinxparamcomma \sphinxparam{\DUrole{n}{n\_noises}}}{}
\pysigstopsignatures
\sphinxAtStartPar
Plots seismogram with picks and taper window
\begin{quote}\begin{description}
\sphinxlineitem{Parameters}\begin{itemize}
\item {} 
\sphinxAtStartPar
\sphinxstyleliteralstrong{\sphinxupquote{picks}}

\item {} 
\sphinxAtStartPar
\sphinxstyleliteralstrong{\sphinxupquote{traces}}

\item {} 
\sphinxAtStartPar
\sphinxstyleliteralstrong{\sphinxupquote{stream}}

\item {} 
\sphinxAtStartPar
\sphinxstyleliteralstrong{\sphinxupquote{n\_noises}}

\end{itemize}

\end{description}\end{quote}

\end{fulllineitems}

\index{set\_plot() (amw.mw.plot.PlotMw method)@\spxentry{set\_plot()}\spxextra{amw.mw.plot.PlotMw method}}

\begin{fulllineitems}
\phantomsection\label{\detokenize{api_support:amw.mw.plot.PlotMw.set_plot}}
\pysigstartsignatures
\pysiglinewithargsret{\sphinxbfcode{\sphinxupquote{set\_plot}}}{\sphinxparam{\DUrole{n}{station\_name}}\sphinxparamcomma \sphinxparam{\DUrole{n}{pick\_name}\DUrole{o}{=}\DUrole{default_value}{None}}}{}
\pysigstopsignatures
\sphinxAtStartPar
The method sets the plot area for station and optional the string.
In the case of magnitude estimation on single phase the \sphinxcode{\sphinxupquote{pick}} must be the phase name \sphinxcode{\sphinxupquote{P}} or \sphinxcode{\sphinxupquote{S}}.
In the case of magnitude estimation on many phases together the \sphinxcode{\sphinxupquote{pick}} must be None or omitted.
\begin{quote}\begin{description}
\sphinxlineitem{Parameters}\begin{itemize}
\item {} 
\sphinxAtStartPar
\sphinxstyleliteralstrong{\sphinxupquote{station\_name}} (\sphinxstyleliteralemphasis{\sphinxupquote{str}}) \textendash{} The station name in the form “NN.SSSS” where NN is the network
code and SSSS is the station code

\item {} 
\sphinxAtStartPar
\sphinxstyleliteralstrong{\sphinxupquote{pick\_name}} (\sphinxstyleliteralemphasis{\sphinxupquote{str}}) \textendash{} The pick name. It must be \sphinxcode{\sphinxupquote{P}} or \sphinxcode{\sphinxupquote{S}} for single phase mw or None for many phases mw

\end{itemize}

\end{description}\end{quote}

\end{fulllineitems}

\index{show\_plot() (amw.mw.plot.PlotMw method)@\spxentry{show\_plot()}\spxextra{amw.mw.plot.PlotMw method}}

\begin{fulllineitems}
\phantomsection\label{\detokenize{api_support:amw.mw.plot.PlotMw.show_plot}}
\pysigstartsignatures
\pysiglinewithargsret{\sphinxbfcode{\sphinxupquote{show\_plot}}}{\sphinxparam{\DUrole{n}{what\_to\_show}}}{}
\pysigstopsignatures
\sphinxAtStartPar
Visualize the plot and save figure to the file.
\begin{quote}\begin{description}
\sphinxlineitem{Parameters}
\sphinxAtStartPar
\sphinxstyleliteralstrong{\sphinxupquote{what\_to\_show}} (\sphinxstyleliteralemphasis{\sphinxupquote{str}}) \textendash{} Condition what to show. I can be only values ‘single figure’, ‘many figures’ or
‘many figures inline’. Plot is shown if the parameter equals with the ‘what to show’ ‘plot’
parameter in the configuration.

\end{description}\end{quote}

\end{fulllineitems}

\index{start\_plot() (amw.mw.plot.PlotMw method)@\spxentry{start\_plot()}\spxextra{amw.mw.plot.PlotMw method}}

\begin{fulllineitems}
\phantomsection\label{\detokenize{api_support:amw.mw.plot.PlotMw.start_plot}}
\pysigstartsignatures
\pysiglinewithargsret{\sphinxbfcode{\sphinxupquote{start\_plot}}}{\sphinxparam{\DUrole{n}{stations}}}{}
\pysigstopsignatures
\sphinxAtStartPar
The method configures the plots, defines figures and axis. The way of plotting depends
on the \sphinxcode{\sphinxupquote{plot}}, \sphinxcode{\sphinxupquote{how\_to\_show}} parameter in the configuration file.
There are three possibilities:
\begin{itemize}
\item {} 
\sphinxAtStartPar
\sphinxcode{\sphinxupquote{single figure}} \sphinxhyphen{} results of magnitude estimation of all station are in plot one figure,
station under station. On the left is seismogram and the left are spectra,

\item {} 
\sphinxAtStartPar
\sphinxcode{\sphinxupquote{many figures}} \sphinxhyphen{} results of magnitude estimation of all station are in one figure.
On the left is the seismogram and the bottom are spectra,

\item {} 
\sphinxAtStartPar
\sphinxcode{\sphinxupquote{many figures inline}} \sphinxhyphen{} results of magnitude estimation of all station are in one figure.
On the left is the seismogram and the left are spectra,

\end{itemize}
\begin{quote}\begin{description}
\sphinxlineitem{Parameters}
\sphinxAtStartPar
\sphinxstyleliteralstrong{\sphinxupquote{stations}} (\sphinxstyleliteralemphasis{\sphinxupquote{list}}\sphinxstyleliteralemphasis{\sphinxupquote{(}}\sphinxstyleliteralemphasis{\sphinxupquote{str}}\sphinxstyleliteralemphasis{\sphinxupquote{)}}) \textendash{} List of station names

\end{description}\end{quote}

\end{fulllineitems}


\end{fulllineitems}

\index{module@\spxentry{module}!amw.mw.single\_phase\_mw@\spxentry{amw.mw.single\_phase\_mw}}\index{amw.mw.single\_phase\_mw@\spxentry{amw.mw.single\_phase\_mw}!module@\spxentry{module}}

\subsection{Single phase spectral magnitude estimation}
\label{\detokenize{api_support:single-phase-spectral-magnitude-estimation}}\label{\detokenize{api_support:module-amw.mw.single_phase_mw}}\index{estimate\_single\_phase\_mw() (in module amw.mw.single\_phase\_mw)@\spxentry{estimate\_single\_phase\_mw()}\spxextra{in module amw.mw.single\_phase\_mw}}

\begin{fulllineitems}
\phantomsection\label{\detokenize{api_support:amw.mw.single_phase_mw.estimate_single_phase_mw}}
\pysigstartsignatures
\pysiglinewithargsret{\sphinxcode{\sphinxupquote{amw.mw.single\_phase\_mw.}}\sphinxbfcode{\sphinxupquote{estimate\_single\_phase\_mw}}}{\sphinxparam{\DUrole{n}{signal}}\sphinxparamcomma \sphinxparam{\DUrole{n}{pick\_name}}\sphinxparamcomma \sphinxparam{\DUrole{n}{picks}}\sphinxparamcomma \sphinxparam{\DUrole{n}{origin}}\sphinxparamcomma \sphinxparam{\DUrole{n}{station\_inventory}}\sphinxparamcomma \sphinxparam{\DUrole{n}{configuration}}}{}
\pysigstopsignatures
\sphinxAtStartPar
Estimates spectral moment magnitude on the single phase P or S
\begin{quote}\begin{description}
\sphinxlineitem{Parameters}\begin{itemize}
\item {} 
\sphinxAtStartPar
\sphinxstyleliteralstrong{\sphinxupquote{signal}} (\sphinxstyleliteralemphasis{\sphinxupquote{ObsPy.Stream}}) \textendash{} The signal is the 3D seismic displacement stream, which must cover both the P wave, the S wave,
and the noise before the P onset.

\item {} 
\sphinxAtStartPar
\sphinxstyleliteralstrong{\sphinxupquote{pick\_name}} (\sphinxstyleliteralemphasis{\sphinxupquote{str}}) \textendash{} The name of the pick ‘P’ or ‘S’

\item {} 
\sphinxAtStartPar
\sphinxstyleliteralstrong{\sphinxupquote{picks}} (\sphinxstyleliteralemphasis{\sphinxupquote{list}}\sphinxstyleliteralemphasis{\sphinxupquote{(}}\sphinxstyleliteralemphasis{\sphinxupquote{ObsPy.Pick}}\sphinxstyleliteralemphasis{\sphinxupquote{)}}) \textendash{} Two picks P and S are recommended. At least one wave\_name must be given.
If the P or S wave\_name is missing,
the function tries to determine it based on the earthquake time at the focus and the remaining wave\_name time.

\item {} 
\sphinxAtStartPar
\sphinxstyleliteralstrong{\sphinxupquote{origin}} (\sphinxstyleliteralemphasis{\sphinxupquote{ObsPy.Origin}}) \textendash{} The event origin.

\item {} 
\sphinxAtStartPar
\sphinxstyleliteralstrong{\sphinxupquote{station\_inventory}} (\sphinxstyleliteralemphasis{\sphinxupquote{ObsPy.Inventory}}) \textendash{} The inventory of the station that the signal was picked on

\item {} 
\sphinxAtStartPar
\sphinxstyleliteralstrong{\sphinxupquote{configuration}} (\sphinxstyleliteralemphasis{\sphinxupquote{dict}}) \textendash{} The configuration container of all parameters dictionary required for the program to work.

\item {} 
\sphinxAtStartPar
\sphinxstyleliteralstrong{\sphinxupquote{inventory}} (\sphinxstyleliteralemphasis{\sphinxupquote{ObsPy.Inventory}}) \textendash{} The inventory of all stations and channels

\end{itemize}

\sphinxlineitem{Returns}
\sphinxAtStartPar
mw : Estimated moment magnitude
f0 : Source function corner frequency
m0 : Scalar moment
time\_window : The assessed time window of P and S waves

\sphinxlineitem{Return type}
\sphinxAtStartPar
tuple

\end{description}\end{quote}

\end{fulllineitems}

\index{module@\spxentry{module}!amw.mw.double\_phase\_mw@\spxentry{amw.mw.double\_phase\_mw}}\index{amw.mw.double\_phase\_mw@\spxentry{amw.mw.double\_phase\_mw}!module@\spxentry{module}}

\subsection{Cumulated P and S phases spectral magnitude estimation}
\label{\detokenize{api_support:cumulated-p-and-s-phases-spectral-magnitude-estimation}}\label{\detokenize{api_support:module-amw.mw.double_phase_mw}}\index{estimate\_double\_phase\_mw() (in module amw.mw.double\_phase\_mw)@\spxentry{estimate\_double\_phase\_mw()}\spxextra{in module amw.mw.double\_phase\_mw}}

\begin{fulllineitems}
\phantomsection\label{\detokenize{api_support:amw.mw.double_phase_mw.estimate_double_phase_mw}}
\pysigstartsignatures
\pysiglinewithargsret{\sphinxcode{\sphinxupquote{amw.mw.double\_phase\_mw.}}\sphinxbfcode{\sphinxupquote{estimate\_double\_phase\_mw}}}{\sphinxparam{\DUrole{n}{signal}}\sphinxparamcomma \sphinxparam{\DUrole{n}{picks}}\sphinxparamcomma \sphinxparam{\DUrole{n}{origin}}\sphinxparamcomma \sphinxparam{\DUrole{n}{station\_inventory}}\sphinxparamcomma \sphinxparam{\DUrole{n}{configuration}}}{}
\pysigstopsignatures
\sphinxAtStartPar
Estimates the moment magnitude on the signal covering both phases P and S together.
\begin{quote}\begin{description}
\sphinxlineitem{Parameters}\begin{itemize}
\item {} 
\sphinxAtStartPar
\sphinxstyleliteralstrong{\sphinxupquote{signal}} (\sphinxstyleliteralemphasis{\sphinxupquote{ObsPy Stream}}) \textendash{} The signal is the 3D seismic displacement stream, which must cover both the P wave, the S wave,
and the noise before the P onset.

\item {} 
\sphinxAtStartPar
\sphinxstyleliteralstrong{\sphinxupquote{picks}} (\sphinxstyleliteralemphasis{\sphinxupquote{list}}\sphinxstyleliteralemphasis{\sphinxupquote{(}}\sphinxstyleliteralemphasis{\sphinxupquote{ObsPy.Pick}}\sphinxstyleliteralemphasis{\sphinxupquote{)}}) \textendash{} The list of two picks: P and S. The P wave\_name is first the S wave\_name is second.
If the wave\_name is missing there should be None value. At list one wave\_name is required,
but two picks P and S are recommended. At least one wave\_name must be given. If the P or S wave\_name is missing,
the function tries to determine it based on the earthquake time at the focus and the remaining wave\_name time.

\item {} 
\sphinxAtStartPar
\sphinxstyleliteralstrong{\sphinxupquote{origin}} (\sphinxstyleliteralemphasis{\sphinxupquote{ObsPy Origin}}) \textendash{} The event origin.

\item {} 
\sphinxAtStartPar
\sphinxstyleliteralstrong{\sphinxupquote{station\_inventory}} (\sphinxstyleliteralemphasis{\sphinxupquote{ObsPy.Inventory}}) \textendash{} The inventory of the station that the signal was picked on

\item {} 
\sphinxAtStartPar
\sphinxstyleliteralstrong{\sphinxupquote{configuration}} (\sphinxstyleliteralemphasis{\sphinxupquote{dict}}) \textendash{} The configuration container of all parameters dictionary required for the program to work.

\item {} 
\sphinxAtStartPar
\sphinxstyleliteralstrong{\sphinxupquote{inventory}} (\sphinxstyleliteralemphasis{\sphinxupquote{ObsPy.Inventory}}) \textendash{} The inventory of all stations and channels

\end{itemize}

\sphinxlineitem{Returns}
\sphinxAtStartPar
mw : Estimated moment magnitude
f0 : Source function corner frequency
m0 : Scalar moment
time\_window : The assessed time window of P and S waves

\end{description}\end{quote}
\begin{description}
\sphinxlineitem{Uses functions :}
\sphinxAtStartPar
get\_theoretical\_s
get\_theoretical\_p
estimate\_mw

\end{description}

\end{fulllineitems}

\index{module@\spxentry{module}!amw.mw.source\_models@\spxentry{amw.mw.source\_models}}\index{amw.mw.source\_models@\spxentry{amw.mw.source\_models}!module@\spxentry{module}}

\subsection{Seismic source models in frequency domain}
\label{\detokenize{api_support:seismic-source-models-in-frequency-domain}}\label{\detokenize{api_support:module-amw.mw.source_models}}\index{BoatwrightSourceModel (class in amw.mw.source\_models)@\spxentry{BoatwrightSourceModel}\spxextra{class in amw.mw.source\_models}}

\begin{fulllineitems}
\phantomsection\label{\detokenize{api_support:amw.mw.source_models.BoatwrightSourceModel}}
\pysigstartsignatures
\pysiglinewithargsret{\sphinxbfcode{\sphinxupquote{class\DUrole{w}{ }}}\sphinxcode{\sphinxupquote{amw.mw.source\_models.}}\sphinxbfcode{\sphinxupquote{BoatwrightSourceModel}}}{\sphinxparam{\DUrole{n}{frequencies}}\sphinxparamcomma \sphinxparam{\DUrole{n}{gamma}\DUrole{o}{=}\DUrole{default_value}{1.0}}\sphinxparamcomma \sphinxparam{\DUrole{n}{n}\DUrole{o}{=}\DUrole{default_value}{2.0}}}{}
\pysigstopsignatures
\sphinxAtStartPar
The Boatwright (1978; 1980) seismic source model is:
\begin{equation*}
\begin{split}S\left(f|M_0,f_0\right)=
{\frac{1}{2\pi f}M_0\left[{1+\left(\frac{f}{f_0}\right)}^{n\gamma}\right]}^\frac{-1}{\gamma},\end{split}
\end{equation*}
\sphinxAtStartPar
where \(M_0\) is a scalar moment and \(f_0\) is a cornel frequency. Constant values \(\gamma\)
and \(n\) controls the sharpness of the corners of the spectrum. For \(\gamma = 1\) and \(n = 2\),
it is Brune (1970; 1971) source model:
\begin{equation*}
\begin{split}S\left(f|M_0,f_0\right)={\frac{1}{2\pi f}M_0\left[{1+\left(\frac{f}{f_0}\right)}^2\right]}^{-1}\end{split}
\end{equation*}\begin{quote}\begin{description}
\sphinxlineitem{Parameters}\begin{itemize}
\item {} 
\sphinxAtStartPar
\sphinxstyleliteralstrong{\sphinxupquote{frequencies}} (\sphinxstyleliteralemphasis{\sphinxupquote{numpy array}}) \textendash{} The frequencies values, for w spectral function values will be computed

\item {} 
\sphinxAtStartPar
\sphinxstyleliteralstrong{\sphinxupquote{gamma}}

\item {} 
\sphinxAtStartPar
\sphinxstyleliteralstrong{\sphinxupquote{n}}

\end{itemize}

\end{description}\end{quote}

\sphinxAtStartPar
Default parameters gamma=1 and n=2 are for Brunea source model

\sphinxAtStartPar
References:
\begin{itemize}
\item {} 
\sphinxAtStartPar
Boatwright, J. (1978). Detailed spectral analysis of two small New York State earthquakes,
Bull. Seism. Soc. Am. 68 (4), 1117\textendash{}1131. \sphinxurl{https://doi.org/10.1785/BSSA0680041117}

\item {} 
\sphinxAtStartPar
Boatwright, J. (1980). A spectral theory for circular seismic sources; simple estimates of source dimension,
dynamic stress drop, and radiated seismic energy,
Bull. Seism. Soc. Am. 70 (1), 1\textendash{}27. \sphinxurl{https://doi.org/10.1785/BSSA0700010001}

\item {} 
\sphinxAtStartPar
Brune, J. N. (1970). Tectonic stress and the spectra of seismic shear waves from earthquakes,
J. Geophys. Res. 75, 4997\sphinxhyphen{}5009. \sphinxurl{https://doi.org/10.1029/JB075i026p04997}

\item {} 
\sphinxAtStartPar
Brune, J.N. (1971) Correction {[}to “Tectonic Stress and the Spectra of Seismic Shear Waves from Earthquakes”{]},
J. Geophys. Res. 76, 5002. \sphinxurl{http://dx.doi.org/10.1029/JB076i020p05002}

\end{itemize}

\end{fulllineitems}

\index{BruneSourceModel (class in amw.mw.source\_models)@\spxentry{BruneSourceModel}\spxextra{class in amw.mw.source\_models}}

\begin{fulllineitems}
\phantomsection\label{\detokenize{api_support:amw.mw.source_models.BruneSourceModel}}
\pysigstartsignatures
\pysiglinewithargsret{\sphinxbfcode{\sphinxupquote{class\DUrole{w}{ }}}\sphinxcode{\sphinxupquote{amw.mw.source\_models.}}\sphinxbfcode{\sphinxupquote{BruneSourceModel}}}{\sphinxparam{\DUrole{n}{frequencies}}}{}
\pysigstopsignatures
\sphinxAtStartPar
Brune (1970; 1971) seismic source model is:
\begin{equation*}
\begin{split}S\left(f|M_0,f_0\right)={\frac{1}{2\pi f}M_0\left[{1+\left(\frac{f}{f_0}\right)}^2\right]}^{-1},\end{split}
\end{equation*}
\sphinxAtStartPar
where \(M_0\) is a scalar moment and \(f_0\) is a corner frequency.
\begin{quote}\begin{description}
\sphinxlineitem{Parameters}
\sphinxAtStartPar
\sphinxstyleliteralstrong{\sphinxupquote{frequencies}} (\sphinxstyleliteralemphasis{\sphinxupquote{numpy array}}) \textendash{} The frequencies values, for w spectral function values will be computed

\end{description}\end{quote}

\end{fulllineitems}

\index{HaskellSourceModel (class in amw.mw.source\_models)@\spxentry{HaskellSourceModel}\spxextra{class in amw.mw.source\_models}}

\begin{fulllineitems}
\phantomsection\label{\detokenize{api_support:amw.mw.source_models.HaskellSourceModel}}
\pysigstartsignatures
\pysiglinewithargsret{\sphinxbfcode{\sphinxupquote{class\DUrole{w}{ }}}\sphinxcode{\sphinxupquote{amw.mw.source\_models.}}\sphinxbfcode{\sphinxupquote{HaskellSourceModel}}}{\sphinxparam{\DUrole{n}{frequencies}}\sphinxparamcomma \sphinxparam{\DUrole{n}{gamma}\DUrole{o}{=}\DUrole{default_value}{1.0}}\sphinxparamcomma \sphinxparam{\DUrole{n}{n}\DUrole{o}{=}\DUrole{default_value}{2.0}}}{}
\pysigstopsignatures\begin{quote}

\sphinxAtStartPar
Haskell (1964) seismic source model is:
\end{quote}
\begin{equation*}
\begin{split}S\left(f|M_0,f_0\right)=\frac{1}{2\pi f}M_0\text{sinc}\frac{f}{f0},\end{split}
\end{equation*}
\sphinxAtStartPar
where \(M_0\) is a scalar moment and \(f_0\) is a corner frequency.
\begin{quote}\begin{description}
\sphinxlineitem{Parameters}
\sphinxAtStartPar
\sphinxstyleliteralstrong{\sphinxupquote{frequencies}} (\sphinxstyleliteralemphasis{\sphinxupquote{numpy array}}) \textendash{} The frequencies values, for w spectral function values will be computed

\end{description}\end{quote}

\end{fulllineitems}

\index{module@\spxentry{module}!amw.mw.MinimizeInGrid@\spxentry{amw.mw.MinimizeInGrid}}\index{amw.mw.MinimizeInGrid@\spxentry{amw.mw.MinimizeInGrid}!module@\spxentry{module}}

\subsection{The function minimization by checking all solutions in a grid}
\label{\detokenize{api_support:the-function-minimization-by-checking-all-solutions-in-a-grid}}\label{\detokenize{api_support:module-amw.mw.MinimizeInGrid}}\index{grid\_search() (in module amw.mw.MinimizeInGrid)@\spxentry{grid\_search()}\spxextra{in module amw.mw.MinimizeInGrid}}

\begin{fulllineitems}
\phantomsection\label{\detokenize{api_support:amw.mw.MinimizeInGrid.grid_search}}
\pysigstartsignatures
\pysiglinewithargsret{\sphinxcode{\sphinxupquote{amw.mw.MinimizeInGrid.}}\sphinxbfcode{\sphinxupquote{grid\_search}}}{\sphinxparam{\DUrole{n}{function}}\sphinxparamcomma \sphinxparam{\DUrole{n}{x}}\sphinxparamcomma \sphinxparam{\DUrole{n}{args}\DUrole{o}{=}\DUrole{default_value}{None}}}{}
\pysigstopsignatures
\sphinxAtStartPar
It minimizes the function by checking all solutions in a grid.
\begin{quote}\begin{description}
\sphinxlineitem{Parameters}\begin{itemize}
\item {} 
\sphinxAtStartPar
\sphinxstyleliteralstrong{\sphinxupquote{function}} (\sphinxstyleliteralemphasis{\sphinxupquote{func}}) \textendash{} The minimized function

\item {} 
\sphinxAtStartPar
\sphinxstyleliteralstrong{\sphinxupquote{x}} (\sphinxstyleliteralemphasis{\sphinxupquote{list}}) \textendash{} Initial values (not used, only for compatibility)

\item {} 
\sphinxAtStartPar
\sphinxstyleliteralstrong{\sphinxupquote{args}} (\sphinxstyleliteralemphasis{\sphinxupquote{list}}) \textendash{} The minimized function arguments. It must be the configuration dictionary.

\end{itemize}

\sphinxlineitem{Returns}
\sphinxAtStartPar
The minimization result

\sphinxlineitem{Return type}
\sphinxAtStartPar
RetClass

\end{description}\end{quote}

\end{fulllineitems}


\sphinxstepscope


\section{Core modules}
\label{\detokenize{api_core:core-modules}}\label{\detokenize{api_core::doc}}
\sphinxAtStartPar
Core modules are used in the Anthropogenic Mw package,
but are designated for more general use, therefor the are described separately.

\sphinxAtStartPar
The arclink\_client.py is located the core utils.
Because it is now obsolete but still used in some seismic networks,
it is taken from the older version of ObsPy and put in the core subdirectory.
\index{module@\spxentry{module}!amw.core.signal\_utils@\spxentry{amw.core.signal\_utils}}\index{amw.core.signal\_utils@\spxentry{amw.core.signal\_utils}!module@\spxentry{module}}

\subsection{The waveform and inventory manipulation}
\label{\detokenize{api_core:the-waveform-and-inventory-manipulation}}\label{\detokenize{api_core:module-amw.core.signal_utils}}\index{Cache (class in amw.core.signal\_utils)@\spxentry{Cache}\spxextra{class in amw.core.signal\_utils}}

\begin{fulllineitems}
\phantomsection\label{\detokenize{api_core:amw.core.signal_utils.Cache}}
\pysigstartsignatures
\pysiglinewithargsret{\sphinxbfcode{\sphinxupquote{class\DUrole{w}{ }}}\sphinxcode{\sphinxupquote{amw.core.signal\_utils.}}\sphinxbfcode{\sphinxupquote{Cache}}}{\sphinxparam{\DUrole{n}{configuration}}\sphinxparamcomma \sphinxparam{\DUrole{n}{file\_name}}}{}
\pysigstopsignatures
\sphinxAtStartPar
The cache class for manipulating the cache metadata
\begin{quote}\begin{description}
\sphinxlineitem{Parameters}\begin{itemize}
\item {} 
\sphinxAtStartPar
\sphinxstyleliteralstrong{\sphinxupquote{configuration}} (\sphinxstyleliteralemphasis{\sphinxupquote{dict}}) \textendash{} The container of general seismic processing configuration.
The required parameter is a cache path kept in the ‘cache’.

\item {} 
\sphinxAtStartPar
\sphinxstyleliteralstrong{\sphinxupquote{file\_name}} (\sphinxstyleliteralemphasis{\sphinxupquote{str}}) \textendash{} The cache metadata file name

\end{itemize}

\end{description}\end{quote}
\index{backup() (amw.core.signal\_utils.Cache method)@\spxentry{backup()}\spxextra{amw.core.signal\_utils.Cache method}}

\begin{fulllineitems}
\phantomsection\label{\detokenize{api_core:amw.core.signal_utils.Cache.backup}}
\pysigstartsignatures
\pysiglinewithargsret{\sphinxbfcode{\sphinxupquote{backup}}}{}{}
\pysigstopsignatures
\sphinxAtStartPar
Backs up the cache metadata. Saves to the JSON file.
\begin{quote}\begin{description}
\sphinxlineitem{Returns}
\sphinxAtStartPar
None

\end{description}\end{quote}

\end{fulllineitems}


\end{fulllineitems}

\index{SignalException@\spxentry{SignalException}}

\begin{fulllineitems}
\phantomsection\label{\detokenize{api_core:amw.core.signal_utils.SignalException}}
\pysigstartsignatures
\pysiglinewithargsret{\sphinxbfcode{\sphinxupquote{exception\DUrole{w}{ }}}\sphinxcode{\sphinxupquote{amw.core.signal\_utils.}}\sphinxbfcode{\sphinxupquote{SignalException}}}{\sphinxparam{\DUrole{n}{message}\DUrole{o}{=}\DUrole{default_value}{\textquotesingle{}other\textquotesingle{}}}}{}
\pysigstopsignatures
\end{fulllineitems}

\index{StreamLoader (class in amw.core.signal\_utils)@\spxentry{StreamLoader}\spxextra{class in amw.core.signal\_utils}}

\begin{fulllineitems}
\phantomsection\label{\detokenize{api_core:amw.core.signal_utils.StreamLoader}}
\pysigstartsignatures
\pysiglinewithargsret{\sphinxbfcode{\sphinxupquote{class\DUrole{w}{ }}}\sphinxcode{\sphinxupquote{amw.core.signal\_utils.}}\sphinxbfcode{\sphinxupquote{StreamLoader}}}{\sphinxparam{\DUrole{n}{configuration}}\sphinxparamcomma \sphinxparam{\DUrole{n}{preprocess}\DUrole{o}{=}\DUrole{default_value}{None}}}{}
\pysigstopsignatures
\sphinxAtStartPar
The stream loader loads seismic waveforms from servers ArcLink or FDSNWS
and process data initially. The loaded and processed data can be kept on local disc
in the cache directory for increase the reloading speed.
\begin{quote}\begin{description}
\sphinxlineitem{Parameters}\begin{itemize}
\item {} 
\sphinxAtStartPar
\sphinxstyleliteralstrong{\sphinxupquote{configuration}} (\sphinxstyleliteralemphasis{\sphinxupquote{dict}}) \textendash{} The container of general seismic processing configuration.
The required parameters are kept in the ‘stream’ sub\sphinxhyphen{}dictionary:

\item {} 
\sphinxAtStartPar
\sphinxstyleliteralstrong{\sphinxupquote{preprocess}} ({\hyperref[\detokenize{api_core:amw.core.signal_utils.StreamPreprocessing}]{\sphinxcrossref{\sphinxstyleliteralemphasis{\sphinxupquote{StreamPreprocessing}}}}})

\end{itemize}

\end{description}\end{quote}

\sphinxAtStartPar
\sphinxstylestrong{The parameters present in the ‘stream’ sub\sphinxhyphen{}dictionary:}
\begin{quote}\begin{description}
\sphinxlineitem{Source}
\sphinxAtStartPar
The waveforms source. Available options are ‘arclink’ or ‘fdsnws’ (required)

\sphinxlineitem{Host}
\sphinxAtStartPar
The server host

\sphinxlineitem{Port}
\sphinxAtStartPar
The server port

\sphinxlineitem{User}
\sphinxAtStartPar
The request user id (if required)

\sphinxlineitem{Password}
\sphinxAtStartPar
The request password (if required)

\sphinxlineitem{Timeout}
\sphinxAtStartPar
The downloading timeout limit

\sphinxlineitem{Net}
\sphinxAtStartPar
The default network name.

\sphinxlineitem{Sta}
\sphinxAtStartPar
The default station name.

\sphinxlineitem{Loc}
\sphinxAtStartPar
The default location name.

\sphinxlineitem{Chan}
\sphinxAtStartPar
The default channel name.

\sphinxlineitem{Cache}
\sphinxAtStartPar
The cache directory. In the cache directory are kept all downloaded and preprocessed waveform files
and the file ‘loaded\_signals.json’ containing info

\sphinxlineitem{Stations}
\sphinxAtStartPar
The default request station list

\end{description}\end{quote}
\index{download() (amw.core.signal\_utils.StreamLoader method)@\spxentry{download()}\spxextra{amw.core.signal\_utils.StreamLoader method}}

\begin{fulllineitems}
\phantomsection\label{\detokenize{api_core:amw.core.signal_utils.StreamLoader.download}}
\pysigstartsignatures
\pysiglinewithargsret{\sphinxbfcode{\sphinxupquote{download}}}{\sphinxparam{\DUrole{n}{begin\_time}}\sphinxparamcomma \sphinxparam{\DUrole{n}{end\_time}}\sphinxparamcomma \sphinxparam{\DUrole{n}{event\_id}}\sphinxparamcomma \sphinxparam{\DUrole{n}{new\_file\_name}\DUrole{o}{=}\DUrole{default_value}{None}}}{}
\pysigstopsignatures
\sphinxAtStartPar
Downloads the stream from the seismic data sever with optional caching.
\begin{quote}\begin{description}
\sphinxlineitem{Parameters}\begin{itemize}
\item {} 
\sphinxAtStartPar
\sphinxstyleliteralstrong{\sphinxupquote{new\_file\_name}} (\sphinxstyleliteralemphasis{\sphinxupquote{str}}) \textendash{} The proposed name of the file tobe stored in the cache.
If it is missing the unique random name is generated.

\item {} 
\sphinxAtStartPar
\sphinxstyleliteralstrong{\sphinxupquote{begin\_time}} (\sphinxstyleliteralemphasis{\sphinxupquote{ObsPy.UTCDateTime}}) \textendash{} The begin time of waveforms

\item {} 
\sphinxAtStartPar
\sphinxstyleliteralstrong{\sphinxupquote{end\_time}} (\sphinxstyleliteralemphasis{\sphinxupquote{ObsPy.UTCDateTime}}) \textendash{} The end time of waveforms

\item {} 
\sphinxAtStartPar
\sphinxstyleliteralstrong{\sphinxupquote{event\_id}} (\sphinxstyleliteralemphasis{\sphinxupquote{str}}) \textendash{} The event id, but it can be any string defining the stream request,
which can identify the data in case of repeated inquiry.

\end{itemize}

\sphinxlineitem{Returns}
\sphinxAtStartPar
The requested stream or None if it can not be downloaded

\sphinxlineitem{Return type}
\sphinxAtStartPar
ObsPy.Stream

\end{description}\end{quote}

\end{fulllineitems}

\index{exist\_file() (amw.core.signal\_utils.StreamLoader method)@\spxentry{exist\_file()}\spxextra{amw.core.signal\_utils.StreamLoader method}}

\begin{fulllineitems}
\phantomsection\label{\detokenize{api_core:amw.core.signal_utils.StreamLoader.exist_file}}
\pysigstartsignatures
\pysiglinewithargsret{\sphinxbfcode{\sphinxupquote{exist\_file}}}{\sphinxparam{\DUrole{n}{begin\_time}}\sphinxparamcomma \sphinxparam{\DUrole{n}{end\_time}}\sphinxparamcomma \sphinxparam{\DUrole{n}{event\_id}}}{}
\pysigstopsignatures
\sphinxAtStartPar
The method checks if the requested waveform exists. A few conditions are checked.
First it checks if the cache exists. Then checks if event id exists.
The requested period must include in the existing file period.
The requested station list must include in the existing file station list.
The preprocessor name must be the same.
\begin{quote}\begin{description}
\sphinxlineitem{Parameters}\begin{itemize}
\item {} 
\sphinxAtStartPar
\sphinxstyleliteralstrong{\sphinxupquote{begin\_time}} (\sphinxstyleliteralemphasis{\sphinxupquote{ObsPy.UTCDateTime}}) \textendash{} The requested waveforms begin time

\item {} 
\sphinxAtStartPar
\sphinxstyleliteralstrong{\sphinxupquote{end\_time}} (\sphinxstyleliteralemphasis{\sphinxupquote{ObsPy.UTCDateTime}}) \textendash{} The requested waveforms begin time

\item {} 
\sphinxAtStartPar
\sphinxstyleliteralstrong{\sphinxupquote{event\_id}} (\sphinxstyleliteralemphasis{\sphinxupquote{str}}) \textendash{} The request event id. It can be the event id that the waveforms are associated
or any string that identify the request.

\end{itemize}

\sphinxlineitem{Returns}
\sphinxAtStartPar
The parameters of existed file or None,
if the function can not fit request to existing files list

\sphinxlineitem{Return type}
\sphinxAtStartPar
dict

\end{description}\end{quote}

\end{fulllineitems}

\index{get\_signal() (amw.core.signal\_utils.StreamLoader method)@\spxentry{get\_signal()}\spxextra{amw.core.signal\_utils.StreamLoader method}}

\begin{fulllineitems}
\phantomsection\label{\detokenize{api_core:amw.core.signal_utils.StreamLoader.get_signal}}
\pysigstartsignatures
\pysiglinewithargsret{\sphinxbfcode{\sphinxupquote{get\_signal}}}{\sphinxparam{\DUrole{n}{begin\_time}}\sphinxparamcomma \sphinxparam{\DUrole{n}{end\_time}}\sphinxparamcomma \sphinxparam{\DUrole{n}{event\_id}\DUrole{o}{=}\DUrole{default_value}{None}}\sphinxparamcomma \sphinxparam{\DUrole{n}{stations}\DUrole{o}{=}\DUrole{default_value}{None}}\sphinxparamcomma \sphinxparam{\DUrole{n}{new\_file\_name}\DUrole{o}{=}\DUrole{default_value}{None}}}{}
\pysigstopsignatures
\sphinxAtStartPar
Provides seismic signal waveform based on request.
If matching the request file exist in the cache it reads signal from the file,
otherwise download from the seismic waveforms’ server.
\begin{quote}\begin{description}
\sphinxlineitem{Parameters}\begin{itemize}
\item {} 
\sphinxAtStartPar
\sphinxstyleliteralstrong{\sphinxupquote{begin\_time}} (\sphinxstyleliteralemphasis{\sphinxupquote{ObsPy.UTCDateTime}}) \textendash{} The requested waveforms begin time

\item {} 
\sphinxAtStartPar
\sphinxstyleliteralstrong{\sphinxupquote{end\_time}} (\sphinxstyleliteralemphasis{\sphinxupquote{ObsPy.UTCDateTime}}) \textendash{} The requested waveforms begin time

\item {} 
\sphinxAtStartPar
\sphinxstyleliteralstrong{\sphinxupquote{event\_id}} (\sphinxstyleliteralemphasis{\sphinxupquote{str}}) \textendash{} The request event id. It can be the event id that the waveforms are associated
or any string that identify the request. (optional If missing waveform is only downloaded from the server)

\item {} 
\sphinxAtStartPar
\sphinxstyleliteralstrong{\sphinxupquote{stations}} (\sphinxstyleliteralemphasis{\sphinxupquote{list}}\sphinxstyleliteralemphasis{\sphinxupquote{(}}\sphinxstyleliteralemphasis{\sphinxupquote{str}}\sphinxstyleliteralemphasis{\sphinxupquote{)}}) \textendash{} The request stations list.
(optional) If it is missing the station list from the configuration is checked.

\item {} 
\sphinxAtStartPar
\sphinxstyleliteralstrong{\sphinxupquote{new\_file\_name}} (\sphinxstyleliteralemphasis{\sphinxupquote{str}}) \textendash{} The name of a file in the cache.
(optional) If missing the unique file name is generated.

\end{itemize}

\sphinxlineitem{Returns}
\sphinxAtStartPar
The waveform stream. Return None if it can not (or could not) download waveforms.

\sphinxlineitem{Return type}
\sphinxAtStartPar
ObsPy.Stream

\end{description}\end{quote}

\end{fulllineitems}


\end{fulllineitems}

\index{StreamPreprocessing (class in amw.core.signal\_utils)@\spxentry{StreamPreprocessing}\spxextra{class in amw.core.signal\_utils}}

\begin{fulllineitems}
\phantomsection\label{\detokenize{api_core:amw.core.signal_utils.StreamPreprocessing}}
\pysigstartsignatures
\pysiglinewithargsret{\sphinxbfcode{\sphinxupquote{class\DUrole{w}{ }}}\sphinxcode{\sphinxupquote{amw.core.signal\_utils.}}\sphinxbfcode{\sphinxupquote{StreamPreprocessing}}}{\sphinxparam{\DUrole{n}{name}}}{}
\pysigstopsignatures
\sphinxAtStartPar
The base class of streams preprocessing
\begin{quote}\begin{description}
\sphinxlineitem{Parameters}
\sphinxAtStartPar
\sphinxstyleliteralstrong{\sphinxupquote{name}} (\sphinxstyleliteralemphasis{\sphinxupquote{str}}) \textendash{} The name of the preprocessing

\end{description}\end{quote}

\end{fulllineitems}

\index{get\_inventory() (in module amw.core.signal\_utils)@\spxentry{get\_inventory()}\spxextra{in module amw.core.signal\_utils}}

\begin{fulllineitems}
\phantomsection\label{\detokenize{api_core:amw.core.signal_utils.get_inventory}}
\pysigstartsignatures
\pysiglinewithargsret{\sphinxcode{\sphinxupquote{amw.core.signal\_utils.}}\sphinxbfcode{\sphinxupquote{get\_inventory}}}{\sphinxparam{\DUrole{n}{sta\_name}}\sphinxparamcomma \sphinxparam{\DUrole{n}{date}}\sphinxparamcomma \sphinxparam{\DUrole{n}{inventory}}}{}
\pysigstopsignatures
\sphinxAtStartPar
Extracts inventory for the station.
\begin{quote}\begin{description}
\sphinxlineitem{Parameters}\begin{itemize}
\item {} 
\sphinxAtStartPar
\sphinxstyleliteralstrong{\sphinxupquote{sta\_name}} (\sphinxstyleliteralemphasis{\sphinxupquote{str}}) \textendash{} The station name as the string in the form ‘NN.SSS’,
where ‘NN’ is the network code and ‘SSS’ is the station code.

\item {} 
\sphinxAtStartPar
\sphinxstyleliteralstrong{\sphinxupquote{date}} (\sphinxstyleliteralemphasis{\sphinxupquote{ObsPy.UTCDateTime}}) \textendash{} The date of the inventory

\item {} 
\sphinxAtStartPar
\sphinxstyleliteralstrong{\sphinxupquote{inventory}} (\sphinxstyleliteralemphasis{\sphinxupquote{ObsPy.Inventory}}) \textendash{} The inventory of all stations

\end{itemize}

\sphinxlineitem{Returns}
\sphinxAtStartPar
The inventory of the station

\sphinxlineitem{Return type}
\sphinxAtStartPar
ObsPy.Inventory

\end{description}\end{quote}

\end{fulllineitems}

\index{load\_inventory() (in module amw.core.signal\_utils)@\spxentry{load\_inventory()}\spxextra{in module amw.core.signal\_utils}}

\begin{fulllineitems}
\phantomsection\label{\detokenize{api_core:amw.core.signal_utils.load_inventory}}
\pysigstartsignatures
\pysiglinewithargsret{\sphinxcode{\sphinxupquote{amw.core.signal\_utils.}}\sphinxbfcode{\sphinxupquote{load\_inventory}}}{\sphinxparam{\DUrole{n}{configuration}}}{}
\pysigstopsignatures
\sphinxAtStartPar
Loads inventory from the file. The file name and format are in ‘inventory’ configuration.
If inventory file is missing the inventory is downloaded from the waveform server,
which configuration is in the ‘stream’ sub\sphinxhyphen{}dictionary.
\begin{quote}\begin{description}
\sphinxlineitem{Parameters}
\sphinxAtStartPar
\sphinxstyleliteralstrong{\sphinxupquote{configuration}} (\sphinxstyleliteralemphasis{\sphinxupquote{dict}}) \textendash{} The container of general seismic processing configuration.
The required parameters are kept in the ‘inventory’ sub\sphinxhyphen{}dictionary.

\sphinxlineitem{Returns}
\sphinxAtStartPar
The inventory

\sphinxlineitem{Return type}
\sphinxAtStartPar
ObsPy.Inventory

\end{description}\end{quote}

\sphinxAtStartPar
\sphinxstylestrong{The parameters present in the ‘inventory’ sub\sphinxhyphen{}dictionary:}
\begin{quote}\begin{description}
\sphinxlineitem{File\_name}
\sphinxAtStartPar
The inventory file name. (optional, default is ‘inventory.xml’)

\sphinxlineitem{File\_format}
\sphinxAtStartPar
The format of the inventory file name. (optional, default is ‘STATIONXML’)

\end{description}\end{quote}

\end{fulllineitems}

\index{module@\spxentry{module}!amw.core.utils@\spxentry{amw.core.utils}}\index{amw.core.utils@\spxentry{amw.core.utils}!module@\spxentry{module}}

\subsection{Commonly used utils for seismic data processing be the seismic processing in Python packages}
\label{\detokenize{api_core:commonly-used-utils-for-seismic-data-processing-be-the-seismic-processing-in-python-packages}}\label{\detokenize{api_core:module-amw.core.utils}}\index{ExtremeTraceValues (class in amw.core.utils)@\spxentry{ExtremeTraceValues}\spxextra{class in amw.core.utils}}

\begin{fulllineitems}
\phantomsection\label{\detokenize{api_core:amw.core.utils.ExtremeTraceValues}}
\pysigstartsignatures
\pysiglinewithargsret{\sphinxbfcode{\sphinxupquote{class\DUrole{w}{ }}}\sphinxcode{\sphinxupquote{amw.core.utils.}}\sphinxbfcode{\sphinxupquote{ExtremeTraceValues}}}{\sphinxparam{\DUrole{n}{trace}}\sphinxparamcomma \sphinxparam{\DUrole{n}{begin\_time}\DUrole{o}{=}\DUrole{default_value}{None}}\sphinxparamcomma \sphinxparam{\DUrole{n}{end\_time}\DUrole{o}{=}\DUrole{default_value}{None}}}{}
\pysigstopsignatures
\sphinxAtStartPar
Class that assess the extreme trace values: maximum, minimum, and absolute maximum value
\begin{quote}\begin{description}
\sphinxlineitem{Parameters}\begin{itemize}
\item {} 
\sphinxAtStartPar
\sphinxstyleliteralstrong{\sphinxupquote{trace}} (\sphinxstyleliteralemphasis{\sphinxupquote{ObsPy.Trace}}) \textendash{} The processed trace

\item {} 
\sphinxAtStartPar
\sphinxstyleliteralstrong{\sphinxupquote{begin\_time}} (\sphinxstyleliteralemphasis{\sphinxupquote{ObsPy.UTCDateTime}}) \textendash{} It limits the period, where a process is performed.
If begin\_time is not defined or it is earlier than the beginning of the trace
the process is performed from the beginning of the trace

\item {} 
\sphinxAtStartPar
\sphinxstyleliteralstrong{\sphinxupquote{end\_time}} (\sphinxstyleliteralemphasis{\sphinxupquote{ObsPy.UTCDateTime}}) \textendash{} It limits the period, where a process is performed.
If end\_time is not defined or it is later than the end of the trace,
the process is performed to the end of the trace

\end{itemize}

\end{description}\end{quote}

\sphinxAtStartPar
\sphinxstylestrong{Class variables}
\begin{quote}\begin{description}
\sphinxlineitem{Data}
\sphinxAtStartPar
The optionally cut to time limits data. The data are not a new array but subarray of the Trace data

\sphinxlineitem{Start\_time}
\sphinxAtStartPar
The time of the first data sample.

\sphinxlineitem{End\_time}
\sphinxAtStartPar
The time of the next sample after the last data sample.
It differs from the ObsPy trace end\_time, which points to the last sample of the trace

\sphinxlineitem{Max\_value}
\sphinxAtStartPar
Maximum data value

\sphinxlineitem{Max\_value}
\sphinxAtStartPar
Minimum data value

\sphinxlineitem{Abs\_max}
\sphinxAtStartPar
Absolute maximum value.
abs\_max = max(abs(min\_value), abs(max\_value))

\end{description}\end{quote}

\end{fulllineitems}

\index{IndexTrace (class in amw.core.utils)@\spxentry{IndexTrace}\spxextra{class in amw.core.utils}}

\begin{fulllineitems}
\phantomsection\label{\detokenize{api_core:amw.core.utils.IndexTrace}}
\pysigstartsignatures
\pysiglinewithargsret{\sphinxbfcode{\sphinxupquote{class\DUrole{w}{ }}}\sphinxcode{\sphinxupquote{amw.core.utils.}}\sphinxbfcode{\sphinxupquote{IndexTrace}}}{\sphinxparam{\DUrole{n}{trace}}\sphinxparamcomma \sphinxparam{\DUrole{n}{begin\_time}\DUrole{o}{=}\DUrole{default_value}{None}}\sphinxparamcomma \sphinxparam{\DUrole{n}{end\_time}\DUrole{o}{=}\DUrole{default_value}{None}}}{}
\pysigstopsignatures
\sphinxAtStartPar
Class for operating directly on time limited part of trace data
\begin{quote}\begin{description}
\sphinxlineitem{Parameters}\begin{itemize}
\item {} 
\sphinxAtStartPar
\sphinxstyleliteralstrong{\sphinxupquote{trace}} (\sphinxstyleliteralemphasis{\sphinxupquote{ObsPy.Trace}}) \textendash{} The processed trace

\item {} 
\sphinxAtStartPar
\sphinxstyleliteralstrong{\sphinxupquote{begin\_time}} (\sphinxstyleliteralemphasis{\sphinxupquote{ObsPy.UTCDateTime}}) \textendash{} It limits the period, where a process is performed.
If begin\_time is not defined or it is earlier than the beginning of the trace
the process is performed from the beginning of the trace

\item {} 
\sphinxAtStartPar
\sphinxstyleliteralstrong{\sphinxupquote{end\_time}} (\sphinxstyleliteralemphasis{\sphinxupquote{ObsPy.UTCDateTime}}) \textendash{} It limits the period, where a process is performed.
If end\_time is not defined or it is later than the end of the trace,
the process is performed to the end of the trace

\end{itemize}

\end{description}\end{quote}

\sphinxAtStartPar
\sphinxstylestrong{Class variables}
\begin{quote}\begin{description}
\sphinxlineitem{Start\_time}
\sphinxAtStartPar
The time of the first data sample index

\sphinxlineitem{End\_time}
\sphinxAtStartPar
The time of the next sample after the last data sample.
It differs from the ObsPy trace end\_time, which points to the last sample of the trace

\sphinxlineitem{Begin\_idx}
\sphinxAtStartPar
The first data sample index

\sphinxlineitem{End\_idx}
\sphinxAtStartPar
The last data sample index + 1

\end{description}\end{quote}

\sphinxAtStartPar
Example:

\begin{sphinxVerbatim}[commandchars=\\\{\}]
\PYG{o}{\PYGZgt{}\PYGZgt{}} \PYG{k+kn}{from} \PYG{n+nn}{utils} \PYG{k+kn}{import} \PYG{n}{IndexTrace}
\PYG{o}{\PYGZgt{}\PYGZgt{}} \PYG{k+kn}{from} \PYG{n+nn}{obspy}\PYG{n+nn}{.}\PYG{n+nn}{core}\PYG{n+nn}{.}\PYG{n+nn}{utcdatetime} \PYG{k+kn}{import} \PYG{n}{UTCDateTime}
\PYG{o}{\PYGZgt{}\PYGZgt{}} \PYG{n}{t1} \PYG{o}{=} \PYG{n}{UTCDateTime}\PYG{p}{(}\PYG{l+m+mi}{2024}\PYG{p}{,} \PYG{l+m+mi}{1}\PYG{p}{,} \PYG{l+m+mi}{3}\PYG{p}{,} \PYG{l+m+mi}{8}\PYG{p}{,} \PYG{l+m+mi}{28}\PYG{p}{,} \PYG{l+m+mi}{00}\PYG{p}{)}
\PYG{o}{\PYGZgt{}\PYGZgt{}} \PYG{n}{t2} \PYG{o}{=} \PYG{n}{UTCDateTime}\PYG{p}{(}\PYG{l+m+mi}{2024}\PYG{p}{,} \PYG{l+m+mi}{1}\PYG{p}{,} \PYG{l+m+mi}{3}\PYG{p}{,} \PYG{l+m+mi}{8}\PYG{p}{,} \PYG{l+m+mi}{29}\PYG{p}{,} \PYG{l+m+mi}{00}\PYG{p}{)}
\PYG{o}{\PYGZgt{}\PYGZgt{}} \PYG{n}{st} \PYG{o}{=} \PYG{n}{read}\PYG{p}{(}\PYG{l+s+s1}{\PYGZsq{}}\PYG{l+s+s1}{test.msd}\PYG{l+s+s1}{\PYGZsq{}}\PYG{p}{)}
\PYG{o}{\PYGZgt{}\PYGZgt{}} \PYG{n}{indexes} \PYG{o}{=} \PYG{n}{IndexTrace}\PYG{p}{(}\PYG{n}{st}\PYG{p}{[}\PYG{l+m+mi}{1}\PYG{p}{]}\PYG{p}{,} \PYG{n}{begin\PYGZus{}time}\PYG{o}{=}\PYG{n}{t1}\PYG{p}{,} \PYG{n}{end\PYGZus{}time}\PYG{o}{=}\PYG{n}{t2}
\PYG{o}{\PYGZgt{}\PYGZgt{}} \PYG{k}{for} \PYG{n}{idx} \PYG{o+ow}{in} \PYG{n+nb}{range}\PYG{p}{(}\PYG{n}{indexes}\PYG{o}{.}\PYG{n}{begin\PYGZus{}idx}\PYG{p}{,} \PYG{n}{indexes}\PYG{o}{.}\PYG{n}{end\PYGZus{}idx}\PYG{p}{)}\PYG{p}{:}
\PYG{o}{.}\PYG{o}{.}\PYG{o}{.} \PYG{k}{pass}
\end{sphinxVerbatim}

\end{fulllineitems}

\index{ProcessTrace (class in amw.core.utils)@\spxentry{ProcessTrace}\spxextra{class in amw.core.utils}}

\begin{fulllineitems}
\phantomsection\label{\detokenize{api_core:amw.core.utils.ProcessTrace}}
\pysigstartsignatures
\pysiglinewithargsret{\sphinxbfcode{\sphinxupquote{class\DUrole{w}{ }}}\sphinxcode{\sphinxupquote{amw.core.utils.}}\sphinxbfcode{\sphinxupquote{ProcessTrace}}}{\sphinxparam{\DUrole{n}{trace}}\sphinxparamcomma \sphinxparam{\DUrole{n}{begin\_time}\DUrole{o}{=}\DUrole{default_value}{None}}\sphinxparamcomma \sphinxparam{\DUrole{n}{end\_time}\DUrole{o}{=}\DUrole{default_value}{None}}}{}
\pysigstopsignatures
\sphinxAtStartPar
The base class of the trace processing. Implementations of objects of classes derived from the ProcessTrace
do some processing on traces defined in the derived classes initialization
\begin{quote}\begin{description}
\sphinxlineitem{Parameters}\begin{itemize}
\item {} 
\sphinxAtStartPar
\sphinxstyleliteralstrong{\sphinxupquote{trace}} (\sphinxstyleliteralemphasis{\sphinxupquote{ObsPy.Trace}}) \textendash{} The processed trace

\item {} 
\sphinxAtStartPar
\sphinxstyleliteralstrong{\sphinxupquote{begin\_time}} (\sphinxstyleliteralemphasis{\sphinxupquote{ObsPy.UTCDateTime}}) \textendash{} It limits the period, where a process is performed.
If begin\_time is not defined or it is earlier than the beginning of the trace,
the process is performed from the beginning of the trace

\item {} 
\sphinxAtStartPar
\sphinxstyleliteralstrong{\sphinxupquote{end\_time}} (\sphinxstyleliteralemphasis{\sphinxupquote{ObsPy.UTCDateTime}}) \textendash{} It limits the period, where a process is performed.
If end\_time is not defined or it is later than the end of the trace,
the process is performed to the end of the trace

\end{itemize}

\end{description}\end{quote}

\end{fulllineitems}

\index{get\_focal\_mechanism() (in module amw.core.utils)@\spxentry{get\_focal\_mechanism()}\spxextra{in module amw.core.utils}}

\begin{fulllineitems}
\phantomsection\label{\detokenize{api_core:amw.core.utils.get_focal_mechanism}}
\pysigstartsignatures
\pysiglinewithargsret{\sphinxcode{\sphinxupquote{amw.core.utils.}}\sphinxbfcode{\sphinxupquote{get\_focal\_mechanism}}}{\sphinxparam{\DUrole{n}{event}}\sphinxparamcomma \sphinxparam{\DUrole{n}{inversion\_type}\DUrole{o}{=}\DUrole{default_value}{None}}}{}
\pysigstopsignatures
\sphinxAtStartPar
Function get\_focal\_mechanism extracts the focal mechanism from the event.
If preferred\_focal\_mechanism\_id of the event is set it return the preferred focal mechanism.
Otherwise, it returns the first focal mechanism from the list.
The function is intended to extract the focal mechanism unconditionally and non\sphinxhyphen{}interactively.
Therefore, if preferred\_focal\_mechanism\_id is not set and there are multiple focal mechanisms,
the returned focal mechanism may be random.
\begin{quote}\begin{description}
\sphinxlineitem{Parameters}\begin{itemize}
\item {} 
\sphinxAtStartPar
\sphinxstyleliteralstrong{\sphinxupquote{event}} (\sphinxstyleliteralemphasis{\sphinxupquote{ObsPy.Event}}) \textendash{} The seismic event object

\item {} 
\sphinxAtStartPar
\sphinxstyleliteralstrong{\sphinxupquote{inversion\_type}} (\sphinxstyleliteralemphasis{\sphinxupquote{(}}\sphinxstyleliteralemphasis{\sphinxupquote{str}}\sphinxstyleliteralemphasis{\sphinxupquote{)}}) \textendash{} The name of tensor inversion type.
It must belong to the QuakeML MTInversionType category:
\sphinxcode{\sphinxupquote{\textquotesingle{}general\textquotesingle{}}}, \sphinxcode{\sphinxupquote{\textquotesingle{}zero trace\textquotesingle{}}}, \sphinxcode{\sphinxupquote{\textquotesingle{}double couple\textquotesingle{}}}, or None.

\end{itemize}

\sphinxlineitem{Returns}
\sphinxAtStartPar
The focal mechanism object or None if none focal\_mechanism is defined for the event
or the focal\_mechanism with the defined inversion type does not exist.

\sphinxlineitem{Return type}
\sphinxAtStartPar
ObsPy.FocalMechanism

\end{description}\end{quote}

\end{fulllineitems}

\index{get\_hypocentral\_distance() (in module amw.core.utils)@\spxentry{get\_hypocentral\_distance()}\spxextra{in module amw.core.utils}}

\begin{fulllineitems}
\phantomsection\label{\detokenize{api_core:amw.core.utils.get_hypocentral_distance}}
\pysigstartsignatures
\pysiglinewithargsret{\sphinxcode{\sphinxupquote{amw.core.utils.}}\sphinxbfcode{\sphinxupquote{get\_hypocentral\_distance}}}{\sphinxparam{\DUrole{n}{origin}}\sphinxparamcomma \sphinxparam{\DUrole{n}{station\_inventory}}}{}
\pysigstopsignatures
\sphinxAtStartPar
Function get\_hypocentral\_distance computes the local hypocentral distance in meters
from origin coordinates to station\_name coordinates.
The calculations do not take into account the curvature of the earth.
\begin{quote}\begin{description}
\sphinxlineitem{Parameters}\begin{itemize}
\item {} 
\sphinxAtStartPar
\sphinxstyleliteralstrong{\sphinxupquote{origin}} (\sphinxstyleliteralemphasis{\sphinxupquote{ObsPy.Origin}}) \textendash{} The ObsPy Origin object

\item {} 
\sphinxAtStartPar
\sphinxstyleliteralstrong{\sphinxupquote{station\_inventory}} (\sphinxstyleliteralemphasis{\sphinxupquote{ObsPy.Inventory}}) \textendash{} The station inventory object

\end{itemize}

\sphinxlineitem{Returns}
\sphinxAtStartPar
The hypocentral distance in meters and epicentral distance in degrees

\sphinxlineitem{Return type}
\sphinxAtStartPar
tuple(float, float)

\end{description}\end{quote}

\end{fulllineitems}

\index{get\_magnitude() (in module amw.core.utils)@\spxentry{get\_magnitude()}\spxextra{in module amw.core.utils}}

\begin{fulllineitems}
\phantomsection\label{\detokenize{api_core:amw.core.utils.get_magnitude}}
\pysigstartsignatures
\pysiglinewithargsret{\sphinxcode{\sphinxupquote{amw.core.utils.}}\sphinxbfcode{\sphinxupquote{get\_magnitude}}}{\sphinxparam{\DUrole{n}{event}}\sphinxparamcomma \sphinxparam{\DUrole{n}{magnitude\_type}\DUrole{o}{=}\DUrole{default_value}{None}}}{}
\pysigstopsignatures
\sphinxAtStartPar
Function get\_magnitude extracts the magnitude of the event.
If you want to extract a specific magnitude you can define it as magnitude\_type,
e.g. \sphinxcode{\sphinxupquote{get\_magnitude(event, magnitude\_type=\textquotesingle{}mw\textquotesingle{})}}, otherwise, any magnitude will be extracted.
If the preferred\_magnitude\_id of the event is set it returns the preferred origin.
Otherwise, it returns the first magnitude from the list.
The function is intended to extract the magnitude unconditionally and non\sphinxhyphen{}interactively.
Therefore, if preferred\_magnitude\_id is not set and there are multiple magnitudes,
the returned origin may be random.

\sphinxAtStartPar
If event magnitude does not exist, but station\_name magnitudes exist, the new magnitude is computed
as the mean value of station\_name magnitudes.
\begin{quote}\begin{description}
\sphinxlineitem{Parameters}\begin{itemize}
\item {} 
\sphinxAtStartPar
\sphinxstyleliteralstrong{\sphinxupquote{event}} (\sphinxstyleliteralemphasis{\sphinxupquote{ObsPy.Event}}) \textendash{} The seismic event object

\item {} 
\sphinxAtStartPar
\sphinxstyleliteralstrong{\sphinxupquote{magnitude\_type}} (\sphinxstyleliteralemphasis{\sphinxupquote{str}}) \textendash{} (optional)
Describes the type of magnitude. This is a free\sphinxhyphen{}text. Proposed values are:
* unspecified magnitude (\sphinxcode{\sphinxupquote{\textquotesingle{}M\textquotesingle{}}}) \sphinxhyphen{} function search for exactly unspecified magnitude,
* local magnitude (\sphinxcode{\sphinxupquote{\textquotesingle{}ML\textquotesingle{}}}),
* moment magnitude (\sphinxcode{\sphinxupquote{\textquotesingle{}mw\textquotesingle{}}}),
* energy (\sphinxcode{\sphinxupquote{\textquotesingle{}Energy\textquotesingle{}}}),
* etc.

\end{itemize}

\sphinxlineitem{Returns}
\sphinxAtStartPar
The magnitude object or None if the function cannot find or create the magnitude.
If only station\_name magnitudes exist, the new ObsPy Magnitude object is created,
but it is not appended to the event

\sphinxlineitem{Return type}
\sphinxAtStartPar
ObsPy.Magnitude

\end{description}\end{quote}

\end{fulllineitems}

\index{get\_net\_sta() (in module amw.core.utils)@\spxentry{get\_net\_sta()}\spxextra{in module amw.core.utils}}

\begin{fulllineitems}
\phantomsection\label{\detokenize{api_core:amw.core.utils.get_net_sta}}
\pysigstartsignatures
\pysiglinewithargsret{\sphinxcode{\sphinxupquote{amw.core.utils.}}\sphinxbfcode{\sphinxupquote{get\_net\_sta}}}{\sphinxparam{\DUrole{n}{name}}}{}
\pysigstopsignatures
\sphinxAtStartPar
Function get\_net\_sta extracts network and station\_name codes as strings
\begin{quote}\begin{description}
\sphinxlineitem{Parameters}
\sphinxAtStartPar
\sphinxstyleliteralstrong{\sphinxupquote{name}} (\sphinxstyleliteralemphasis{\sphinxupquote{str}}\sphinxstyleliteralemphasis{\sphinxupquote{ or }}\sphinxstyleliteralemphasis{\sphinxupquote{ObsPy.WaveformStreamID}}) \textendash{} The trace name. It can be the string or the WaveformStreamID object.
The text in the string is in the form ‘NN.SSS.LL.CCC’, where NN is the network code,
SSS is the station\_name code, LL is the location code, and CCC is the channel code.

\sphinxlineitem{Returns}
\sphinxAtStartPar
The tuple of the network code the station\_name code.

\sphinxlineitem{Return type}
\sphinxAtStartPar
tuple(str, str)

\end{description}\end{quote}

\end{fulllineitems}

\index{get\_origin() (in module amw.core.utils)@\spxentry{get\_origin()}\spxextra{in module amw.core.utils}}

\begin{fulllineitems}
\phantomsection\label{\detokenize{api_core:amw.core.utils.get_origin}}
\pysigstartsignatures
\pysiglinewithargsret{\sphinxcode{\sphinxupquote{amw.core.utils.}}\sphinxbfcode{\sphinxupquote{get\_origin}}}{\sphinxparam{\DUrole{n}{event}}}{}
\pysigstopsignatures
\sphinxAtStartPar
Function get\_origin extracts the origin from the event.
If preferred\_origin\_id of the event is set it return the preferred origin.
Otherwise, it returns the first origin from the list.
The function is intended to extract the event origin unconditionally and non\sphinxhyphen{}interactively.
Therefore, if preferred\_origin\_id is not set and there are multiple origins, the returned origin may be random
\begin{quote}\begin{description}
\sphinxlineitem{Parameters}
\sphinxAtStartPar
\sphinxstyleliteralstrong{\sphinxupquote{event}} (\sphinxstyleliteralemphasis{\sphinxupquote{ObsPy.Event}}) \textendash{} The seismic event object

\sphinxlineitem{Returns}
\sphinxAtStartPar
The origin (event location) object or None if none origin is defined for the event.

\sphinxlineitem{Return type}
\sphinxAtStartPar
ObsPy.Origin

\end{description}\end{quote}

\end{fulllineitems}

\index{get\_station\_id() (in module amw.core.utils)@\spxentry{get\_station\_id()}\spxextra{in module amw.core.utils}}

\begin{fulllineitems}
\phantomsection\label{\detokenize{api_core:amw.core.utils.get_station_id}}
\pysigstartsignatures
\pysiglinewithargsret{\sphinxcode{\sphinxupquote{amw.core.utils.}}\sphinxbfcode{\sphinxupquote{get\_station\_id}}}{\sphinxparam{\DUrole{n}{name}}}{}
\pysigstopsignatures
\sphinxAtStartPar
Function get\_station\_id extracts the station\_name name as a WaveformStreamID object
\begin{quote}\begin{description}
\sphinxlineitem{Parameters}
\sphinxAtStartPar
\sphinxstyleliteralstrong{\sphinxupquote{name}} (\sphinxstyleliteralemphasis{\sphinxupquote{str}}\sphinxstyleliteralemphasis{\sphinxupquote{ or }}\sphinxstyleliteralemphasis{\sphinxupquote{ObsPy.WaveformStreamID}}) \textendash{} The trace name. It can be the string or the ObsPy WaveformStreamID object.
The text in the string is in the form ‘NN.SSS.LL.CCC’, where NN is the network code,
SSS is the station\_name code, LL is the location code, and CCC is the channel code.

\sphinxlineitem{Returns}
\sphinxAtStartPar
The waveform stream object containing only the network code and the station\_name code.

\sphinxlineitem{Return type}
\sphinxAtStartPar
ObsPy.WaveformStreamID

\end{description}\end{quote}

\end{fulllineitems}

\index{get\_station\_name() (in module amw.core.utils)@\spxentry{get\_station\_name()}\spxextra{in module amw.core.utils}}

\begin{fulllineitems}
\phantomsection\label{\detokenize{api_core:amw.core.utils.get_station_name}}
\pysigstartsignatures
\pysiglinewithargsret{\sphinxcode{\sphinxupquote{amw.core.utils.}}\sphinxbfcode{\sphinxupquote{get\_station\_name}}}{\sphinxparam{\DUrole{n}{name}}}{}
\pysigstopsignatures
\sphinxAtStartPar
Function get\_station\_name extracts the station\_name name as a string
\begin{quote}\begin{description}
\sphinxlineitem{Parameters}
\sphinxAtStartPar
\sphinxstyleliteralstrong{\sphinxupquote{name}} (\sphinxstyleliteralemphasis{\sphinxupquote{str}}\sphinxstyleliteralemphasis{\sphinxupquote{ or }}\sphinxstyleliteralemphasis{\sphinxupquote{ObsPy.WaveformStreamID}}) \textendash{} The trace name. It can be the string or the WaveformStreamID object.
The text in the string is in the form ‘NN.SSS.LL.CCC’, where NN is the network code,
SSS is the station\_name code, LL is the location code, and CCC is the channel code.

\sphinxlineitem{Returns}
\sphinxAtStartPar
The string in the form ‘NN.STA’, where NN is the network code and SSS is the station\_name code.

\sphinxlineitem{Return type}
\sphinxAtStartPar
str

\end{description}\end{quote}

\end{fulllineitems}

\index{get\_units() (in module amw.core.utils)@\spxentry{get\_units()}\spxextra{in module amw.core.utils}}

\begin{fulllineitems}
\phantomsection\label{\detokenize{api_core:amw.core.utils.get_units}}
\pysigstartsignatures
\pysiglinewithargsret{\sphinxcode{\sphinxupquote{amw.core.utils.}}\sphinxbfcode{\sphinxupquote{get\_units}}}{\sphinxparam{\DUrole{n}{trace}}}{}
\pysigstopsignatures
\sphinxAtStartPar
Return the signal units of the trace
\begin{quote}\begin{description}
\sphinxlineitem{Parameters}
\sphinxAtStartPar
\sphinxstyleliteralstrong{\sphinxupquote{trace}} (\sphinxstyleliteralemphasis{\sphinxupquote{ObsPy.Trace}}) \textendash{} The trace object

\sphinxlineitem{Returns}
\sphinxAtStartPar
The string with units: ‘m/s’, ‘m/s\textasciicircum{}2’, or ‘m’, if the response was removed,
when in the processing\_parameters is the remove\_response process defined,
or ‘counts’ otherwise

\sphinxlineitem{Return type}
\sphinxAtStartPar
str

\end{description}\end{quote}

\end{fulllineitems}

\index{time\_ceil() (in module amw.core.utils)@\spxentry{time\_ceil()}\spxextra{in module amw.core.utils}}

\begin{fulllineitems}
\phantomsection\label{\detokenize{api_core:amw.core.utils.time_ceil}}
\pysigstartsignatures
\pysiglinewithargsret{\sphinxcode{\sphinxupquote{amw.core.utils.}}\sphinxbfcode{\sphinxupquote{time\_ceil}}}{\sphinxparam{\DUrole{n}{time}}\sphinxparamcomma \sphinxparam{\DUrole{n}{step}}}{}
\pysigstopsignatures
\sphinxAtStartPar
Returns the time rounded\sphinxhyphen{}up to the specified accuracy.
\begin{quote}\begin{description}
\sphinxlineitem{Parameters}\begin{itemize}
\item {} 
\sphinxAtStartPar
\sphinxstyleliteralstrong{\sphinxupquote{time}} (\sphinxstyleliteralemphasis{\sphinxupquote{ObsPy.UTCDateTime}}) \textendash{} The time object

\item {} 
\sphinxAtStartPar
\sphinxstyleliteralstrong{\sphinxupquote{step}} (\sphinxstyleliteralemphasis{\sphinxupquote{float}}) \textendash{} The accuracy units in seconds

\end{itemize}

\sphinxlineitem{Returns}
\sphinxAtStartPar
The new rounded\sphinxhyphen{}up time object

\sphinxlineitem{Return type}
\sphinxAtStartPar
ObsPy.UTCDateTime

\end{description}\end{quote}

\sphinxAtStartPar
Example:

\begin{sphinxVerbatim}[commandchars=\\\{\}]
\PYG{o}{\PYGZgt{}\PYGZgt{}} \PYG{k+kn}{from} \PYG{n+nn}{obspy}\PYG{n+nn}{.}\PYG{n+nn}{core}\PYG{n+nn}{.}\PYG{n+nn}{utcdatetime} \PYG{k+kn}{import} \PYG{n}{UTCDateTime}
\PYG{o}{\PYGZgt{}\PYGZgt{}} \PYG{k+kn}{from} \PYG{n+nn}{core}\PYG{n+nn}{.}\PYG{n+nn}{utils} \PYG{k+kn}{import} \PYG{n}{time\PYGZus{}ceil}
\PYG{o}{\PYGZgt{}\PYGZgt{}} \PYG{n}{time} \PYG{o}{=} \PYG{n}{UTCDateTime}\PYG{p}{(}\PYG{l+m+mi}{2024}\PYG{p}{,} \PYG{l+m+mi}{1}\PYG{p}{,} \PYG{l+m+mi}{3}\PYG{p}{,} \PYG{l+m+mi}{8}\PYG{p}{,} \PYG{l+m+mi}{28}\PYG{p}{,} \PYG{l+m+mi}{33}\PYG{p}{,} \PYG{l+m+mi}{245678}\PYG{p}{)}
\PYG{o}{\PYGZgt{}\PYGZgt{}} \PYG{n}{time\PYGZus{}ceil}\PYG{p}{(}\PYG{n}{time}\PYG{p}{,}\PYG{l+m+mf}{1.0}\PYG{p}{)}
\PYG{o}{\PYGZgt{}\PYGZgt{}} \PYG{n}{UTCDateTime}\PYG{p}{(}\PYG{l+m+mi}{2024}\PYG{p}{,} \PYG{l+m+mi}{1}\PYG{p}{,} \PYG{l+m+mi}{3}\PYG{p}{,} \PYG{l+m+mi}{8}\PYG{p}{,} \PYG{l+m+mi}{28}\PYG{p}{,} \PYG{l+m+mi}{34}\PYG{p}{)}
\PYG{o}{\PYGZgt{}\PYGZgt{}} \PYG{n}{time\PYGZus{}ceil}\PYG{p}{(}\PYG{n}{time}\PYG{p}{,}\PYG{l+m+mf}{60.0}\PYG{p}{)}
\PYG{o}{\PYGZgt{}\PYGZgt{}} \PYG{n}{UTCDateTime}\PYG{p}{(}\PYG{l+m+mi}{2024}\PYG{p}{,} \PYG{l+m+mi}{1}\PYG{p}{,} \PYG{l+m+mi}{3}\PYG{p}{,} \PYG{l+m+mi}{8}\PYG{p}{,} \PYG{l+m+mi}{29}\PYG{p}{)}
\PYG{o}{\PYGZgt{}\PYGZgt{}} \PYG{n}{time\PYGZus{}ceil}\PYG{p}{(}\PYG{n}{time}\PYG{p}{,}\PYG{l+m+mf}{0.1}\PYG{p}{)}
\PYG{n}{UTCDateTime}\PYG{p}{(}\PYG{l+m+mi}{2024}\PYG{p}{,} \PYG{l+m+mi}{1}\PYG{p}{,} \PYG{l+m+mi}{3}\PYG{p}{,} \PYG{l+m+mi}{8}\PYG{p}{,} \PYG{l+m+mi}{28}\PYG{p}{,} \PYG{l+m+mi}{33}\PYG{p}{,} \PYG{l+m+mi}{300000}\PYG{p}{)}
\PYG{o}{\PYGZgt{}\PYGZgt{}} \PYG{n}{time\PYGZus{}ceil}\PYG{p}{(}\PYG{n}{time}\PYG{p}{,}\PYG{l+m+mf}{0.01}\PYG{p}{)}
\PYG{o}{\PYGZgt{}\PYGZgt{}} \PYG{n}{UTCDateTime}\PYG{p}{(}\PYG{l+m+mi}{2024}\PYG{p}{,} \PYG{l+m+mi}{1}\PYG{p}{,} \PYG{l+m+mi}{3}\PYG{p}{,} \PYG{l+m+mi}{8}\PYG{p}{,} \PYG{l+m+mi}{28}\PYG{p}{,} \PYG{l+m+mi}{33}\PYG{p}{,} \PYG{l+m+mi}{250000}\PYG{p}{)}
\PYG{o}{\PYGZgt{}\PYGZgt{}} \PYG{n}{time\PYGZus{}ceil}\PYG{p}{(}\PYG{n}{time}\PYG{p}{,}\PYG{l+m+mf}{0.001}\PYG{p}{)}
\PYG{o}{\PYGZgt{}\PYGZgt{}} \PYG{n}{UTCDateTime}\PYG{p}{(}\PYG{l+m+mi}{2024}\PYG{p}{,} \PYG{l+m+mi}{1}\PYG{p}{,} \PYG{l+m+mi}{3}\PYG{p}{,} \PYG{l+m+mi}{8}\PYG{p}{,} \PYG{l+m+mi}{28}\PYG{p}{,} \PYG{l+m+mi}{33}\PYG{p}{,} \PYG{l+m+mi}{246000}\PYG{p}{)}
\end{sphinxVerbatim}

\end{fulllineitems}

\index{time\_ceil\_dist() (in module amw.core.utils)@\spxentry{time\_ceil\_dist()}\spxextra{in module amw.core.utils}}

\begin{fulllineitems}
\phantomsection\label{\detokenize{api_core:amw.core.utils.time_ceil_dist}}
\pysigstartsignatures
\pysiglinewithargsret{\sphinxcode{\sphinxupquote{amw.core.utils.}}\sphinxbfcode{\sphinxupquote{time\_ceil\_dist}}}{\sphinxparam{\DUrole{n}{time}}\sphinxparamcomma \sphinxparam{\DUrole{n}{step}}}{}
\pysigstopsignatures
\sphinxAtStartPar
Returns seconds from the time to the time rounded up to the specified accuracy.
\begin{quote}\begin{description}
\sphinxlineitem{Parameters}\begin{itemize}
\item {} 
\sphinxAtStartPar
\sphinxstyleliteralstrong{\sphinxupquote{time}} (\sphinxstyleliteralemphasis{\sphinxupquote{ObsPy.UTCDateTime}}) \textendash{} The time object

\item {} 
\sphinxAtStartPar
\sphinxstyleliteralstrong{\sphinxupquote{step}} (\sphinxstyleliteralemphasis{\sphinxupquote{float}}) \textendash{} The accuracy units in seconds

\end{itemize}

\sphinxlineitem{Returns}
\sphinxAtStartPar
The period in seconds to the rounded\sphinxhyphen{}up time

\sphinxlineitem{Return type}
\sphinxAtStartPar
float

\end{description}\end{quote}

\sphinxAtStartPar
Example:

\begin{sphinxVerbatim}[commandchars=\\\{\}]
\PYG{o}{\PYGZgt{}\PYGZgt{}} \PYG{k+kn}{from} \PYG{n+nn}{obspy}\PYG{n+nn}{.}\PYG{n+nn}{core}\PYG{n+nn}{.}\PYG{n+nn}{utcdatetime} \PYG{k+kn}{import} \PYG{n}{UTCDateTime}
\PYG{o}{\PYGZgt{}\PYGZgt{}} \PYG{n}{time} \PYG{o}{=} \PYG{n}{UTCDateTime}\PYG{p}{(}\PYG{l+m+mi}{2024}\PYG{p}{,} \PYG{l+m+mi}{1}\PYG{p}{,} \PYG{l+m+mi}{3}\PYG{p}{,} \PYG{l+m+mi}{8}\PYG{p}{,} \PYG{l+m+mi}{28}\PYG{p}{,} \PYG{l+m+mi}{33}\PYG{p}{,} \PYG{l+m+mi}{245678}\PYG{p}{)}
\PYG{o}{\PYGZgt{}\PYGZgt{}} \PYG{n}{time\PYGZus{}ceil\PYGZus{}dist}\PYG{p}{(}\PYG{n}{time}\PYG{p}{,}\PYG{l+m+mf}{0.1}\PYG{p}{)}
\PYG{l+m+mf}{0.054322}
\PYG{o}{\PYGZgt{}\PYGZgt{}} \PYG{n}{time\PYGZus{}ceil\PYGZus{}dist}\PYG{p}{(}\PYG{n}{time}\PYG{p}{,}\PYG{l+m+mf}{1.0}\PYG{p}{)}
\PYG{l+m+mf}{0.754322}
\end{sphinxVerbatim}

\end{fulllineitems}

\index{time\_floor() (in module amw.core.utils)@\spxentry{time\_floor()}\spxextra{in module amw.core.utils}}

\begin{fulllineitems}
\phantomsection\label{\detokenize{api_core:amw.core.utils.time_floor}}
\pysigstartsignatures
\pysiglinewithargsret{\sphinxcode{\sphinxupquote{amw.core.utils.}}\sphinxbfcode{\sphinxupquote{time\_floor}}}{\sphinxparam{\DUrole{n}{time}}\sphinxparamcomma \sphinxparam{\DUrole{n}{step}}}{}
\pysigstopsignatures
\sphinxAtStartPar
Returns the time rounded\sphinxhyphen{}down to the specified accuracy.
\begin{quote}\begin{description}
\sphinxlineitem{Parameters}\begin{itemize}
\item {} 
\sphinxAtStartPar
\sphinxstyleliteralstrong{\sphinxupquote{time}} (\sphinxstyleliteralemphasis{\sphinxupquote{ObsPy.UTCDateTime}}) \textendash{} The time object

\item {} 
\sphinxAtStartPar
\sphinxstyleliteralstrong{\sphinxupquote{step}} (\sphinxstyleliteralemphasis{\sphinxupquote{float}}) \textendash{} The accuracy units in seconds

\end{itemize}

\sphinxlineitem{Returns}
\sphinxAtStartPar
The new rounded\sphinxhyphen{}down time object

\sphinxlineitem{Return type}
\sphinxAtStartPar
ObsPy.UTCDateTime

\end{description}\end{quote}

\sphinxAtStartPar
Example:

\begin{sphinxVerbatim}[commandchars=\\\{\}]
\PYG{o}{\PYGZgt{}\PYGZgt{}} \PYG{k+kn}{from} \PYG{n+nn}{obspy}\PYG{n+nn}{.}\PYG{n+nn}{core}\PYG{n+nn}{.}\PYG{n+nn}{utcdatetime} \PYG{k+kn}{import} \PYG{n}{UTCDateTime}
\PYG{o}{\PYGZgt{}\PYGZgt{}} \PYG{k+kn}{from} \PYG{n+nn}{utils} \PYG{k+kn}{import} \PYG{n}{time\PYGZus{}floor}
\PYG{o}{\PYGZgt{}\PYGZgt{}} \PYG{n}{time} \PYG{o}{=} \PYG{n}{UTCDateTime}\PYG{p}{(}\PYG{l+m+mi}{2024}\PYG{p}{,} \PYG{l+m+mi}{1}\PYG{p}{,} \PYG{l+m+mi}{3}\PYG{p}{,} \PYG{l+m+mi}{8}\PYG{p}{,} \PYG{l+m+mi}{28}\PYG{p}{,} \PYG{l+m+mi}{33}\PYG{p}{,} \PYG{l+m+mi}{245678}\PYG{p}{)}
\PYG{o}{\PYGZgt{}\PYGZgt{}} \PYG{n}{time\PYGZus{}floor}\PYG{p}{(}\PYG{n}{time}\PYG{p}{,}\PYG{l+m+mf}{0.001}\PYG{p}{)}
\PYG{n}{UTCDateTime}\PYG{p}{(}\PYG{l+m+mi}{2024}\PYG{p}{,} \PYG{l+m+mi}{1}\PYG{p}{,} \PYG{l+m+mi}{3}\PYG{p}{,} \PYG{l+m+mi}{8}\PYG{p}{,} \PYG{l+m+mi}{28}\PYG{p}{,} \PYG{l+m+mi}{33}\PYG{p}{,} \PYG{l+m+mi}{245000}\PYG{p}{)}
\PYG{o}{\PYGZgt{}\PYGZgt{}} \PYG{n}{time\PYGZus{}floor}\PYG{p}{(}\PYG{n}{time}\PYG{p}{,}\PYG{l+m+mf}{0.01}\PYG{p}{)}
\PYG{n}{UTCDateTime}\PYG{p}{(}\PYG{l+m+mi}{2024}\PYG{p}{,} \PYG{l+m+mi}{1}\PYG{p}{,} \PYG{l+m+mi}{3}\PYG{p}{,} \PYG{l+m+mi}{8}\PYG{p}{,} \PYG{l+m+mi}{28}\PYG{p}{,} \PYG{l+m+mi}{33}\PYG{p}{,} \PYG{l+m+mi}{240000}\PYG{p}{)}
\PYG{o}{\PYGZgt{}\PYGZgt{}} \PYG{n}{time\PYGZus{}floor}\PYG{p}{(}\PYG{n}{time}\PYG{p}{,}\PYG{l+m+mf}{0.1}\PYG{p}{)}
\PYG{n}{UTCDateTime}\PYG{p}{(}\PYG{l+m+mi}{2024}\PYG{p}{,} \PYG{l+m+mi}{1}\PYG{p}{,} \PYG{l+m+mi}{3}\PYG{p}{,} \PYG{l+m+mi}{8}\PYG{p}{,} \PYG{l+m+mi}{28}\PYG{p}{,} \PYG{l+m+mi}{33}\PYG{p}{,} \PYG{l+m+mi}{200000}\PYG{p}{)}
\PYG{o}{\PYGZgt{}\PYGZgt{}} \PYG{n}{time\PYGZus{}floor}\PYG{p}{(}\PYG{n}{time}\PYG{p}{,}\PYG{l+m+mf}{1.0}\PYG{p}{)}
\PYG{n}{UTCDateTime}\PYG{p}{(}\PYG{l+m+mi}{2024}\PYG{p}{,} \PYG{l+m+mi}{1}\PYG{p}{,} \PYG{l+m+mi}{3}\PYG{p}{,} \PYG{l+m+mi}{8}\PYG{p}{,} \PYG{l+m+mi}{28}\PYG{p}{,} \PYG{l+m+mi}{33}\PYG{p}{)}
\PYG{o}{\PYGZgt{}\PYGZgt{}} \PYG{n}{time\PYGZus{}floor}\PYG{p}{(}\PYG{n}{time}\PYG{p}{,}\PYG{l+m+mf}{60.0}\PYG{p}{)}
\PYG{n}{UTCDateTime}\PYG{p}{(}\PYG{l+m+mi}{2024}\PYG{p}{,} \PYG{l+m+mi}{1}\PYG{p}{,} \PYG{l+m+mi}{3}\PYG{p}{,} \PYG{l+m+mi}{8}\PYG{p}{,} \PYG{l+m+mi}{28}\PYG{p}{)}
\end{sphinxVerbatim}

\end{fulllineitems}

\index{time\_floor\_dist() (in module amw.core.utils)@\spxentry{time\_floor\_dist()}\spxextra{in module amw.core.utils}}

\begin{fulllineitems}
\phantomsection\label{\detokenize{api_core:amw.core.utils.time_floor_dist}}
\pysigstartsignatures
\pysiglinewithargsret{\sphinxcode{\sphinxupquote{amw.core.utils.}}\sphinxbfcode{\sphinxupquote{time\_floor\_dist}}}{\sphinxparam{\DUrole{n}{time}}\sphinxparamcomma \sphinxparam{\DUrole{n}{step}}}{}
\pysigstopsignatures
\sphinxAtStartPar
Returns seconds from the time to the time rounded up to the specified accuracy.
\begin{quote}\begin{description}
\sphinxlineitem{Parameters}\begin{itemize}
\item {} 
\sphinxAtStartPar
\sphinxstyleliteralstrong{\sphinxupquote{time}} (\sphinxstyleliteralemphasis{\sphinxupquote{ObsPy.UTCDateTime}}) \textendash{} The time object

\item {} 
\sphinxAtStartPar
\sphinxstyleliteralstrong{\sphinxupquote{step}} (\sphinxstyleliteralemphasis{\sphinxupquote{float}}) \textendash{} The accuracy units in seconds

\end{itemize}

\sphinxlineitem{Returns}
\sphinxAtStartPar
The period in seconds to the rounded\sphinxhyphen{}down time

\sphinxlineitem{Return type}
\sphinxAtStartPar
float

\end{description}\end{quote}

\sphinxAtStartPar
Example:

\begin{sphinxVerbatim}[commandchars=\\\{\}]
\PYG{o}{\PYGZgt{}\PYGZgt{}} \PYG{k+kn}{from} \PYG{n+nn}{obspy}\PYG{n+nn}{.}\PYG{n+nn}{core}\PYG{n+nn}{.}\PYG{n+nn}{utcdatetime} \PYG{k+kn}{import} \PYG{n}{UTCDateTime}
\PYG{o}{\PYGZgt{}\PYGZgt{}} \PYG{n}{time} \PYG{o}{=} \PYG{n}{UTCDateTime}\PYG{p}{(}\PYG{l+m+mi}{2024}\PYG{p}{,} \PYG{l+m+mi}{1}\PYG{p}{,} \PYG{l+m+mi}{3}\PYG{p}{,} \PYG{l+m+mi}{8}\PYG{p}{,} \PYG{l+m+mi}{28}\PYG{p}{,} \PYG{l+m+mi}{33}\PYG{p}{,} \PYG{l+m+mi}{245678}\PYG{p}{)}
\PYG{o}{\PYGZgt{}\PYGZgt{}} \PYG{n}{time\PYGZus{}floor\PYGZus{}dist}\PYG{p}{(}\PYG{n}{time}\PYG{p}{,}\PYG{l+m+mf}{0.1}\PYG{p}{)}
\PYG{l+m+mf}{0.045678}
\PYG{o}{\PYGZgt{}\PYGZgt{}} \PYG{n}{time\PYGZus{}floor\PYGZus{}dist}\PYG{p}{(}\PYG{n}{time}\PYG{p}{,}\PYG{l+m+mf}{1.0}\PYG{p}{)}
\PYG{l+m+mf}{0.245678}
\end{sphinxVerbatim}

\end{fulllineitems}


\sphinxstepscope


\chapter{Changelogs}
\label{\detokenize{changelog:changelogs}}\label{\detokenize{changelog:changelog}}\label{\detokenize{changelog::doc}}

\section{amw Changelog}
\label{\detokenize{changelog:amw-changelog}}
\sphinxAtStartPar
Earthquake source parameters from P, S, or PS waves displacement spectra
in far, intermediate, and near fields

\sphinxAtStartPar
Copyright (c) 2024\sphinxhyphen{}2024 Jan Wiszniowski \sphinxhref{mailto:jwisz@igf.edu.pl}{jwisz@igf.edu.pl}


\subsection{v0.0.1 \sphinxhyphen{} 2024\sphinxhyphen{}12\sphinxhyphen{}06}
\label{\detokenize{changelog:v0-0-1-2024-12-06}}
\sphinxAtStartPar
Initial Python version.

\sphinxstepscope

\cleardoublepage
\begingroup
\renewcommand\chapter[1]{\endgroup}
\phantomsection


\chapter{References}
\label{\detokenize{bibliography:references}}\label{\detokenize{bibliography::doc}}

\chapter{Indices and tables}
\label{\detokenize{index:indices-and-tables}}\begin{itemize}
\item {} 
\sphinxAtStartPar
\DUrole{xref,std,std-ref}{genindex}

\item {} 
\sphinxAtStartPar
\DUrole{xref,std,std-ref}{modindex}

\item {} 
\sphinxAtStartPar
\DUrole{xref,std,std-ref}{search}

\end{itemize}

\begin{sphinxthebibliography}{1}
\bibitem[1]{bibliography:id6}
\sphinxAtStartPar
Erion\sphinxhyphen{}Vasilis Pikoulis, Olga\sphinxhyphen{}Joan Ktenidou, Emmanouil Z. Psarakis, and Norman A. Abrahamson. Stochastic modeling as a method of arriving at higher frequencies: an application to κ estimation. \sphinxstyleemphasis{Journal of Geophysical Research: Solid Earth}, 125(4):e2019JB018768, 2020. \sphinxhref{https://doi.org/https://doi.org/10.1029/2019JB018768}{doi:https://doi.org/10.1029/2019JB018768}.
\bibitem[2]{bibliography:id2}
\sphinxAtStartPar
John Boatwright. Detailed spectral analysis of two small new york state earthquakes. \sphinxstyleemphasis{Bulletin of the Seismological Society of America}, 68(4):1117\textendash{}1131, 1978. \sphinxhref{https://doi.org/10.1785/BSSA0680041117}{doi:10.1785/BSSA0680041117}.
\bibitem[3]{bibliography:id3}
\sphinxAtStartPar
John Boatwright. A spectral theory for circular seismic sources, simple estimates of source dimension, dynamic stress drop, and radiated seismic energy. \sphinxstyleemphasis{Bulletin of the Seismological Society of America}, 70(1):1\textendash{}27, 1980. \sphinxhref{https://doi.org/10.1785/BSSA0700010001}{doi:10.1785/BSSA0700010001}.
\bibitem[4]{bibliography:id4}
\sphinxAtStartPar
James N. Brune. Tectonic stress and the spectra of seismic shear waves from earthquakes. \sphinxstyleemphasis{Journal of Geophysical Research (1896\sphinxhyphen{}1977)}, 75(26):4997\textendash{}5009, 1970. \sphinxhref{https://doi.org/10.1029/JB075i026p04997}{doi:10.1029/JB075i026p04997}.
\bibitem[5]{bibliography:id5}
\sphinxAtStartPar
James N. Brune. Correction {[}to "Tectonic Stress and the Spectra of Seismic Shear Waves from Earthquakes"{]}. \sphinxstyleemphasis{Journal of Geophysical Research}, 76(5):5002, 1971. \sphinxhref{https://doi.org/10.1029/JB076i020p05002}{doi:10.1029/JB076i020p05002}.
\bibitem[6]{bibliography:id7}
\sphinxAtStartPar
Paweł Wiejacz and Jan Wiszniowski. Moment magnitude determination of local seismic events recorded at selected polish seismic stations. \sphinxstyleemphasis{Acta Geophysica}, 54(1):15\textendash{}32, Mar 2006. URL: \sphinxurl{https://doi.org/10.2478/s11600-006-0003-1}, \sphinxhref{https://doi.org/10.2478/s11600-006-0003-1}{doi:10.2478/s11600\sphinxhyphen{}006\sphinxhyphen{}0003\sphinxhyphen{}1}.
\end{sphinxthebibliography}


\renewcommand{\indexname}{Python Module Index}
\begin{sphinxtheindex}
\let\bigletter\sphinxstyleindexlettergroup
\bigletter{a}
\item\relax\sphinxstyleindexentry{amw.core.signal\_utils}\sphinxstyleindexpageref{api_core:\detokenize{module-amw.core.signal_utils}}
\item\relax\sphinxstyleindexentry{amw.core.utils}\sphinxstyleindexpageref{api_core:\detokenize{module-amw.core.utils}}
\item\relax\sphinxstyleindexentry{amw.mw.double\_phase\_mw}\sphinxstyleindexpageref{api_support:\detokenize{module-amw.mw.double_phase_mw}}
\item\relax\sphinxstyleindexentry{amw.mw.estimation}\sphinxstyleindexpageref{api_support:\detokenize{module-amw.mw.estimation}}
\item\relax\sphinxstyleindexentry{amw.mw.MinimizeInGrid}\sphinxstyleindexpageref{api_support:\detokenize{module-amw.mw.MinimizeInGrid}}
\item\relax\sphinxstyleindexentry{amw.mw.parameters}\sphinxstyleindexpageref{api_support:\detokenize{module-amw.mw.parameters}}
\item\relax\sphinxstyleindexentry{amw.mw.plot}\sphinxstyleindexpageref{api_support:\detokenize{module-amw.mw.plot}}
\item\relax\sphinxstyleindexentry{amw.mw.single\_phase\_mw}\sphinxstyleindexpageref{api_support:\detokenize{module-amw.mw.single_phase_mw}}
\item\relax\sphinxstyleindexentry{amw.mw.source\_models}\sphinxstyleindexpageref{api_support:\detokenize{module-amw.mw.source_models}}
\item\relax\sphinxstyleindexentry{amw.mw.spectral\_mw}\sphinxstyleindexpageref{api_main:\detokenize{module-amw.mw.spectral_mw}}
\item\relax\sphinxstyleindexentry{amw.mw.test\_greens\_function}\sphinxstyleindexpageref{api_main:\detokenize{module-amw.mw.test_greens_function}}
\end{sphinxtheindex}

\renewcommand{\indexname}{Index}
\printindex
\end{document}